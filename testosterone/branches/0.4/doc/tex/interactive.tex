\chapter{Interactive Mode \label{interactive}}

Interactive mode is a front end for scripted mode. There are two main screens,
representing the summary and detail reports described elsewhere. Each is
populated by calling \program{testosterone} in scripted mode in a child process,
and then parsing and formatting the output. There are two additional screens:
One is a primitive pager showing a Python traceback, which is used both for
viewing individual test failures, as well as for error handling in both parent
and child processes. The other is a primitive terminal for interacting with a
\class{Pdb} session in a child process.

You can send a \code{SIGINT} (\code{<ctrl>-C}) at any time to exit
\program{testosterone}.


\section{Summary Screen \label{summary}}

The summary screen shows the summary report as described above, but item names
are indented rather than given in full. Modules are shown in gray, and un-run
\class{TestCase}s in white. \class{TestCase}s with non-passing tests are shown in red, and those
that pass in green.

You may run any subset of the presented tests. The totals for the most recent
test run are shown at the bottom of the screen, in green if all tests pass, red
otherwise. \class{TestCase}s for which there are results but that were not part of the
most recent test run are shown in faded red and green.

\begin{tableii}{l|l}{code}{key}{description}
\lineii{F5}
    {Refresh the list of available \class{TestCase}s without running them.}
\lineii{enter}
    {Run the selected tests and go to the detail screen if there are non-passing
    tests.}
\lineii{q}
    {Exit \program{testosterone}.}
\lineii{right-arrow}
    {Alias for \code{enter}.}
\lineii{space}
    {Run the selected tests and stay on the summary screen.}
\end{tableii}


\section{Detail Screen \label{detail}}

The detail screen shows a list of non-passing tests on the left side, and the
traceback for the currently selected test on the right. Failures are displayed
in red, and errors in yellow. Tests are listed in alphabetical order.

\begin{tableii}{l|l}{code}{key}{description}
\lineii{F5}{Run the tests again.}
\lineii{enter}{Open the traceback for the selected test in an error screen.}
\lineii{left-arrow}{Alias for \code{q}.}
\lineii{q}{Exit back to the summary screen.}
\lineii{right-arrow}{Alias for \code{enter}.}
\lineii{space}{Alias for \code{F5}.}
\end{tableii}


\section{Error Screen \label{error}}

The error screen provides a primitive pager for viewing tracebacks.

\begin{tableii}{l|l}{code}{key}{description}
\lineii{left-arrow}{Alias for \code{q}.}
\lineii{q}{Exit back to the previous screen.}
\end{tableii}


\section{Debugging Screen \label{debugging}}

The debugging screen is a primitive terminal for interacting with a Python
debugger session. When a child process includes the string '\code{(Pdb) }' in
its output, \program{testosterone} enters the debugging screen. When the
debugger exits, \program{testosterone} returns to the previous screen, ignoring
any report output that may have followed the debugging session.

You can easily start debugging from any point in your program or tests by
manually setting a breakpoint:

\begin{verbatim}
import pdb; pdb.set_trace()
\end{verbatim}

\begin{seealso}
\seeurl
    {http://docs.python.org/lib/debugger-commands.html}
    {The Python debugger command reference.}
\end{seealso}