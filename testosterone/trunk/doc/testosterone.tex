.Dd February 20, 2006
.Os
.Dt TESTOSTERONE 1 LOCAL
.\"
.\"
.\"
.\"
.\"
.Sh NAME
.Nm testosterone
.Nd a manly testing interface for Python
.\"
.\"
.\"
.\"
.\"
.Sh SYNOPSIS
.Nm
.Op Ar options
.Ar module
.\"
.\"
.\"
.\"
.\"
.Sh DESCRIPTION

.Nm
is an interface for running tests written with the Python standard library's
unittest module. It delivers summary and detail reports on TestCases discovered
in module-space, via both a command-line and a
.Xr curses 3
interface. The interactive mode is the default, but it depends on the
non-interactive mode. For debugging, static tracebacks and interactive Python
debugger (Pdb) sessions are available in both scripted and interactive modes.
.\"
.\"
.\"
.\"
.\"
.Sh OPTIONS

.Bl -tag -width "--interactive" -compact

.It Fl i
.It Fl -interactive
Use the
.Xr curses 3
interface. This is the default.

.It Fl I
.It Fl -scripted
Use the command-line interface.

.It Fl n
.It Fl -run
.Nm
should run all TestCases that it finds. This only obtains in scripted mode, for
summary reports, where it is the default.

.It Fl N
.It Fl -find
.Nm
should find TestCases but not run them. This only obtains in scripted mode, for
summary reports.

.It Fl x Ar stopwords
.It Fl -stopwords Ar stopwords
.Ar stopwords
is a comma-delimited list of strings that, if they appear in a module's
full dotted name, will prevent that module from being included in the search
for TestCases.

.It Fl t Ar testcase
.It Fl -testcase Ar testcase
.It Fl -TestCase Ar testcase
.Nm
should only run the tests found in
.Ar testcase ,
which is the name of a Python unittest.TestCase class within the module
specified by
.Ar module .
Given this option,
.Nm
will output a detail report for the named TestCase; without it, a summary
report for all TestCases found at or below
.Ar module .
This option only obtains in scripted mode.
.El


.\"
.\"
.\"
.\"
.\"
.Sh SCRIPTED MODE
If the
.Fl -testcase
option is not given,
.Nm
imports
.Ar module ,
and then searches sys.modules for all modules at or below
.Ar module
that do not include any
.Ar stopwords
in their full dotted name.
.Nm
collects TestCase classes that are defined in these modules, and prints a
summary report to the standard output of the format (actually 80 chars wide):

.Bf -literal
    -------------<| testosterone |>-------------
    <header row>
    --------------------------------------------
    <name> <passing> <failures> <errors> <total>
    --------------------------------------------
    TOTALS <passing> <failures> <errors> <total>
.Ef

.Pp
<name> is the full dotted name of a TestCase (this row is repeated for each
TestCase). If the
.Fl -find
flag is set, then no tests are run, and <passing>, <failures>, and <errors> are
each set to a single dash
.Ns ( Sq - ) .
Otherwise, <passing> is given as a percentage, with a terminating percent sign;
the other three are given in absolute terms. There will always be at least one
space between each field, and data rows will be longer than 80 characters iff
the field values exceed the following character lengths:
.Bl -column -offset indent ".Sy field" ".Sy width"
.It Sy field Ta Sy width
.It Li name Ta "  60"
.It Li failures Ta "   4"
.It Li errors Ta "   4"
.It Li total Ta "   4"
.El

Note that in order for your TestCases to be found, you must import their
containing modules within
.Ar module .
.Nm
sets the
.Ev PYTHONTESTING
environment variable to
.Sq testosterone
so that you can avoid defining TestCases or importing testing modules in a
production environment. You can also quarantine your tests in a subpackage, and
give
.Ar module
as the dotted name of this subpackage.

If the
.Fl -testcase
flag is set, then only the named TestCase is run (any
.Fl -find
option is ignored), and
.Nm
delivers a detail report. This report is the usual output of
unittest.TextTestRunner, preceded by the same first banner row as for the
summary report.

For both summary and detail reports,
.Nm
guarantees that no program output will occur after the banner row.
.\"
.\"
.\"
.\"
.\"
.Sh INTERACTIVE MODE
Interactive mode is a front end for scripted mode. There are two main screens,
representing the summary and detail reports described above. Each is populated
by calling
.Nm
in scripted mode in a child process, and then parsing and formatting the output.
There are two additional screens: One is a primitive pager showing a Python
traceback, which is used both for viewing individual test failures, as well as
for error handling in both parent and child processes. The other is a primitive
terminal for interacting with a Pdb session in a child process.

You can send a SIGINT (<ctrl>-C) at any time to exit
.Nm .


.Ss Summary Screen

The summary screen shows the summary report as described above, but item names
are indented rather than given in full. Modules are shown in gray, and un-run
TestCases in white. TestCases with non-passing tests are shown in red, and those
which pass in green.

You may run any subset of the presented tests. The totals for the most recent
test run are shown at the bottom of the screen, in green if all tests pass, red
otherwise. TestCases for which there are results but that were not part of the
most recent test run are shown in faded red and green.

.Bl -hang -width "      "
.It Em F5
Refresh the list of available TestCases without running them.
.It Em enter
Run the selected tests and go to the detail screen if there are non-passing
tests.
.It Em Q
Exit
.Nm .
.It Em right-arrow
Alias for
.Em enter
.It Em space
Run the selected tests and stay on the summary screen.
.El


.Ss Detail Screen

The detail screen shows a list of non-passing tests on the left side, and the
traceback for the currently selected test on the right. Failures are displayed
in red, and errors in yellow. Tests are listed in alphabetical order.

.Bl -hang -width "      "
.It Em F5
Run the tests again.
.It Em enter
Open the traceback for the selected test in an error screen.
.It Em left-arrow
Alias for
.Em Q .
.It Em Q
Exit back to the summary screen.
.It Em right-arrow
Alias for
.Em enter .
.It Em space
Alias for
.Em F5 .
.El


.Ss Error Screen

The error screen provides a primitive pager for viewing tracebacks.

.Bl -hang -width "      "
.It Em left-arrow
Alias for
.Em Q .
.It Em Q
Exit back to the previous screen.
.El


.Ss Debugging Screen

The debugging screen is a primitive terminal for interacting with a Python
debugger session. When a child process includes the string
.Sq "(Pdb) "
in its output,
.Nm
enters the debugging screen. When the debugger exits,
.Nm
returns to the previous screen, ignoring any report output that may have
followed the debugging session.

You can easily start debugging from any point in your program or tests by
manually setting a breakpoint:

.Dl import pdb; pdb.set_trace()

The Python debugger's command reference is online at:

.Dl http://docs.python.org/lib/debugger-commands.html


.\"
.\"
.\"
.\"
.\"
.Sh IMPLEMENTATION NOTES
This program is known to work with the following software:
.Pp
.Bl -dash -offset indent -compact
.It
FreeBSD 4.11
.It
Python 2.4.2
.El
.\"
.\"
.\"
.\"
.\"
.\".Sh FILES
.\"
.\"
.\"
.\"
.\"
.Sh EXAMPLES
Run
.Nm
's
own tests interactively:
.Bl -item -offset indent
.It
$ testosterone testosterone
.El
.\"
.\"
.\"
.\"
.\"
.Sh SEE ALSO
.Xr python 1
.Xr curses 3
.\"
.\"
.\"
.\"
.\"
.\".Sh HISTORY
.\".Bl -hang
.\".It Em 2005-05-01
.\"released version 0.8
.\".El
.\"
.\"
.\"
.\"
.\"
.Sh AUTHORS
.Bl -item -compact
.It
(c) 2005 Chad Whitacre <http://www.zetadev.com/>
.It
This program is beerware. If you like it, buy me a beer someday.
.It
No warranty is expressed or implied.
.El
