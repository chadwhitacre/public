% Complete documentation on the extended LaTeX markup used for Python
% documentation is available in ``Documenting Python'', which is part
% of the standard documentation for Python.  It may be found online
% at:
%
%     http://www.python.org/doc/current/doc/doc.html

\documentclass{manual}

\title{testosterone}

\author{Chad W. L. Whitacre}

% Please at least include a long-lived email address;
% the rest is at your discretion.
\authoraddress{
	Zeta Design \&\ Development \\
	\url{http://www.zetadev.com/software/testosterone/} \\
	Email: \email{\ulink{chad@zetaweb.com}{mailto:chad@zetaweb.com}}
}

%\date{April 30, 1999}		% update before release!
\date\today
				% Use an explicit date so that reformatting
				% doesn't cause a new date to be used.  Setting
				% the date to \today can be used during draft
				% stages to make it easier to handle versions.

\release{0.4}			% release version; this is used to define the
				% \version macro

\makeindex			% tell \index to actually write the .idx file
\makemodindex			% If this contains a lot of module sections.


\begin{document}

\maketitle

\begin{abstract}

\noindent

\program{testosterone} is an interface for running tests written with the Python
standard library's \module{unittest} module. It delivers summary and detail
reports on \class{TestCase}s discovered in module-space, via both a command-line
and a \manpage{curses}{3} interface. The interactive mode is the default, but it
depends on the non-interactive mode. For debugging, static tracebacks and
interactive Python debugger (Pdb) sessions are available in both scripted and
interactive modes.

This software is known to work with \ulink{FreeBSD}{http://www.FreeBSD.org/}
4.11, \ulink{PuTTY}{http://www.chiark.greenend.org.uk/~sgtatham/putty/} 0.58,
and \ulink{Python}{http://www.python.org/} 2.4.2.

Usage:

\begin{verbatim}
$ testosterone [options] module
\end{verbatim}

\program{testosterone} is \copyright 2006 Chad Whitacre. It is offered with
neither warranty nor restrictions.




\end{abstract}

\chapter{Introduction}

httpy exists to bridge your Python application with other HTTP applications. Its
primary use case is as an HTTP origin server for a cluster of Python-based
websites numbering into the hundreds or thousands.

Therefore, httpy shares its design aesthetic with toilet paper: instances of
httpy must be instantly available to solidly perform a single function and then
be cast away without a thought.

With that in mind, here's some httpy zen:

\begin{itemize}
\item
httpy should make simple sites dead simple, and complex sites possible.
\item
Development, deployment, and upgrading should be equally easy, since all will need to happen constantly.
\item
The performance hit of an interpreted server will be offset by easy replication.
\item
Configuration should be kept to a minimum.
\item
Libraries are saner than frameworks.
\end{itemize}

httpy's job is to get HTTP requests into your Python application, and to get HTTP responses from your app back onto the network. Here are some things that are explicitly not httpy's job, along with links to tools that do these jobs well:

\begin{description}

\item[daemonization, complex error logging, uid/gid manipulations]
    {You want \ulink{Dan Bernstein's
    daemontools}{http://cr.yp.to/daemontools.html}. httpy logs everything to the
    standard output, so use
    \ulink{multilog}{http://cr.yp.to/daemontools/multilog.html} to pick up from
    there, and use \ulink{setuidgid}{http://cr.yp.to/daemontools/setuidgid.html}
    to run httpy under a certain account.}

\item[access logging, ssl encryption, virtual hosting, load-balancing]
    {Use an HTTP proxy server such as \ulink{Pound}{http://www.apsis.ch/pound/}.
    You could also do these things with a general-purpose HTTP server such as
    \ulink{Apache}{http://httpd.apache.org/} or
    \ulink{lighttpd}{http://www.lighttpd.net/}.}

\item[caching]
    {Use a caching proxy such as \ulink{Squid}{http://www.squid-cache.org/}, or,
    again, \ulink{Apache}{http://httpd.apache.org/}. Your application should
    also do its own internal caching, of course.}

\item[authentication, authorization, sessioning, storage, templating, etc.]
    {These are your application's responsibility. There are plenty of
    \ulink{Python packages}{http://cheeseshop.python.org/pypi} available to help
    you.}

\end{description}


A future version of httpy is expected to be at least conditionally compliant
with \ulink{HTTP/1.1}{http://www.w3.org/Protocols/rfc2616/rfc2616.html}.
However, the following features are not currently implemented:

\begin{itemize}
\item
Keep-Alive\item
Transfer-Encoding\item
Range requests
\end{itemize}
\chapter{Scripted Mode \label{scripted}}

If the \longprogramopt{testcase} option is not given, \program{testosterone}
imports \var{module}, and then searches \code{sys.modules} for all modules at or
below \var{module} that do not include any \var{stopwords} in their full dotted
name. \program{testosterone} collects \class{TestCase} classes that are defined
in these modules, and prints a summary report to the standard output of the
format:


\begin{verbatim}
-------------------------------<| testosterone |>-------------------------------
<header row>
--------------------------------------------------------------------------------
<name>                                       <passing> <failures> <errors> <all>
--------------------------------------------------------------------------------
TOTALS                                       <passing> <failures> <errors> <all>
\end{verbatim}

\code{<name>} is the full dotted name of a \class{TestCase} (this row is
repeated for each \class{TestCase}). If the \longprogramopt{find} flag is set,
then no tests are run, and \code{<passing>}, \code{<failures>}, and
\code{<errors>} are each set to a single dash (\code{-}). Otherwise,
\code{<passing>} is given as a percentage, with a terminating percent sign; the
other three are given in absolute terms. There will always be at least one space
between each field, and data rows will be longer than 80 characters iff the
field values exceed the following character lengths:

\begin{tableii}{l|l}{}{field}{width}
\lineii{name}{60}
\lineii{failures}{4}
\lineii{errors}{4}
\lineii{all}{4}
\end{tableii}

Note that in order for your \class{TestCase}s to be found, you must import their
containing modules within \var{module}. \program{testosterone} sets the
\envvar{PYTHONTESTING} environment variable to \code{testosterone} so that you
can avoid defining \class{TestCase}s or importing testing modules in a
production environment. You can also quarantine your tests in a subpackage, and
give \var{module} as the dotted name of this subpackage.

If the \longprogramopt{testcase} flag is set, then only the named
\class{TestCase} is run (any \longprogramopt{find} option is ignored), and
\program{testosterone} delivers a detail report. This report is the usual output
of \class{unittest.TextTestRunner}, preceded by the same first banner row as for
the summary report.

For both summary and detail reports, \program{testosterone} guarantees that no
program output will occur after the banner row.
\chapter{Interactive Mode \label{interactive}}

Interactive mode is a front end for scripted mode. There are two main screens,
representing the summary and detail reports described elsewhere. Each is
populated by calling \program{testosterone} in scripted mode in a child process,
and then parsing and formatting the output. There are two additional screens:
One is a primitive pager showing a Python traceback, which is used both for
viewing individual test failures, as well as for error handling in both parent
and child processes. The other is a primitive terminal for interacting with a
\class{Pdb} session in a child process.

You can send a \code{SIGINT} (\code{<ctrl>-C}) at any time to exit
\program{testosterone}.


\section{Summary Screen \label{summary}}

The summary screen shows the summary report as described above, but item names
are indented rather than given in full. Modules are shown in gray, and un-run
\class{TestCase}s in white. \class{TestCase}s with non-passing tests are shown in red, and those
that pass in green.

You may run any subset of the presented tests. The totals for the most recent
test run are shown at the bottom of the screen, in green if all tests pass, red
otherwise. \class{TestCase}s for which there are results but that were not part of the
most recent test run are shown in faded red and green.

\begin{tableii}{l|l}{code}{key}{description}
\lineii{<ctrl>-L}
    {Refresh the list of available \class{TestCase}s without running them.}
\lineii{F5}
    {Run the selected tests and go to the detail screen if there are non-passing
    tests.}
\lineii{enter}
    {alias for \code{F5}}
\lineii{q}
    {Exit \program{testosterone}.}
\lineii{right-arrow}
    {alias for \code{F5}}
\lineii{space}
    {alias for \code{F5}}
\end{tableii}


\section{Detail Screen \label{detail}}

The detail screen shows a list of non-passing tests on the left side, and the
traceback for the currently selected test on the right. Failures are displayed
in red, and errors in yellow. Tests are listed in alphabetical order.

\begin{tableii}{l|l}{code}{key}{description}
\lineii{F5}{Run the tests again.}
\lineii{enter}{Open the traceback for the selected test in an error screen.}
\lineii{left-arrow}{Alias for \code{q}.}
\lineii{q}{Exit back to the summary screen.}
\lineii{right-arrow}{Alias for \code{enter}.}
\lineii{space}{Alias for \code{F5}.}
\end{tableii}


\section{Error Screen \label{error}}

The error screen provides a primitive pager for viewing tracebacks.

\begin{tableii}{l|l}{code}{key}{description}
\lineii{left-arrow}{Alias for \code{q}.}
\lineii{q}{Exit back to the previous screen.}
\end{tableii}


\section{Debugging Screen \label{debugging}}

The debugging screen is a primitive terminal for interacting with a Python
debugger session. When a child process includes the string '\code{(Pdb) }' in
its output, \program{testosterone} enters the debugging screen. When the
debugger exits, \program{testosterone} returns to the previous screen, ignoring
any report output that may have followed the debugging session.

You can easily start debugging from any point in your program or tests by
manually setting a breakpoint:

\begin{verbatim}
import pdb; pdb.set_trace()
\end{verbatim}

\begin{seealso}
\seeurl
    {http://docs.python.org/lib/debugger-commands.html}
    {The Python debugger command reference.}
\end{seealso}




%
%  The ugly "%begin{latexonly}" pseudo-environments are really just to
%  keep LaTeX2HTML quiet during the \renewcommand{} macros; they're
%  not really valuable.
%
%  If you don't want the Module Index, you can remove all of this up
%  until the second \input line.
%
%begin{latexonly}
\renewcommand{\indexname}{Module Index}
%end{latexonly}
\input{mod\jobname.ind}		% Module Index

%begin{latexonly}
\renewcommand{\indexname}{Index}
%end{latexonly}
\input{\jobname.ind}			% Index

\end{document}
