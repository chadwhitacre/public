\chapter{Scripted Mode \label{scripted}}

If the \longprogramopt{testcase} option is not given, \program{testosterone}
imports \var{module}, and then searches \code{sys.modules} for all modules at or
below \var{module} that do not include any \var{stopwords} in their full dotted
name. \program{testosterone} collects \class{TestCase} classes that are defined
in these modules, and prints a summary report to the standard output of the
format:


\begin{verbatim}
-------------------------------<| testosterone |>-------------------------------
<header row>
--------------------------------------------------------------------------------
<name>                                       <passing> <failures> <errors> <all>
--------------------------------------------------------------------------------
TOTALS                                       <passing> <failures> <errors> <all>
\end{verbatim}

\code{<name>} is the full dotted name of a \class{TestCase} (this row is
repeated for each \class{TestCase}). If the \longprogramopt{find} flag is set,
then no tests are run, and \code{<passing>}, \code{<failures>}, and
\code{<errors>} are each set to a single dash (\code{-}). Otherwise,
\code{<passing>} is given as a percentage, with a terminating percent sign; the
other three are given in absolute terms. There will always be at least one space
between each field, and data rows will be longer than 80 characters iff the
field values exceed the following character lengths:

\begin{tableii}{l|l}{}{field}{width}
\lineii{name}{60}
\lineii{failures}{4}
\lineii{errors}{4}
\lineii{all}{4}
\end{tableii}

Note that in order for your \class{TestCase}s to be found, you must import their
containing modules within \var{module}. \program{testosterone} sets the
\envvar{PYTHONTESTING} environment variable to \code{testosterone} so that you
can avoid defining \class{TestCase}s or importing testing modules in a
production environment. You can also quarantine your tests in a subpackage, and
give \var{module} as the dotted name of this subpackage.

If the \longprogramopt{testcase} flag is set, then only the named
\class{TestCase} is run (any \longprogramopt{find} option is ignored), and
\program{testosterone} delivers a detail report. This report is the usual output
of \class{unittest.TextTestRunner}, preceded by the same first banner row as for
the summary report.

For both summary and detail reports, \program{testosterone} guarantees that no
program output will occur after the banner row.