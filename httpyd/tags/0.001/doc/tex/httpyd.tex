% Complete documentation on the extended LaTeX markup used for Python
% documentation is available in ``Documenting Python'', which is part
% of the standard documentation for Python.  It may be found online
% at:
%
%     http://www.python.org/doc/current/doc/doc.html

\documentclass{manual}

\title{httpyd}

\author{Chad W. L. Whitacre}

% Please at least include a long-lived email address;
% the rest is at your discretion.
\authoraddress{
	Zeta Design \&\ Development \\
	\url{http://www.zetadev.com/software/httpyd/} \\
	Email: \email{\ulink{chad@zetaweb.com}{mailto:chad@zetaweb.com}}
}

%\date{January 1, 1970} % update before release!
\date\today
				% Use an explicit date so that reformatting
				% doesn't cause a new date to be used.  Setting
				% the date to \today can be used during draft
				% stages to make it easier to handle versions.

\release{0.8}			% release version; this is used to define the
				% \version macro

\makeindex			% tell \index to actually write the .idx file
\makemodindex			% If this contains a lot of module sections.


\begin{document}

\maketitle

\begin{abstract}

\noindent
httpyd is a sane and robust HTTP server for Python. It serves highly variable
Python-based publication and hybrid publication/application websites.

\end{abstract}

\section{Usage}

\program{httpyd} is an executable that instantiates the \class{Multiple}
responder with the \class{StandAlone} coupler. Out of the box, it serves static
files from the filesystem using \ulink{the \class{Static}
responder}{static.html}. Please see \ulink{the \class{Multiple}
documentation}{multiple.html} for information on extending \program{httpyd}.

\program{httpyd} exposes the following command-line options:


\begin{tableiii}{l|l|l}{var}{Option}{Description}{Default}

\lineiii{\programopt{-a}/\longprogramopt{-address}=\var{address}}
    {The address to which \program{httpyd} should bind. If \var{address} begins
    with a dot or a forward slash, then it is interpreted as an AF_UNIX socket.
    Otherwise, it is interpreted as an AF_INET socket. If \var{address} begins
    with a colon, then the loopback address is assumed.} {\code{:8080}}

\lineiii{\programopt{-d}/\longprogramopt{-daemonize}}
    {If set, \program{httpyd} will double-fork itself.}
    {\code{10}}

\lineiii{\programopt{-m}/\longprogramopt{-mode}=\var{mode}}
    {\var{mode} is one of the strings 'development', 'debugging', 'staging', or
    'deployment'.  The \envvar{HTTPY_MODE} environment variable will be set to
    this value, and will be available to your responders via the
    \class{httpyd.mode} object.} {\code{development}}

\lineiii{\programopt{-t}/\longprogramopt{-threads}=\var{threads}}
    {\program{httpyd} creates a new thread for each request, up to \var{threads}.
    The minimum is 1. No upper limit is enforced.}
    {\code{10}}

\lineiii{\programopt{-u}/\longprogramopt{-user}=\var{user}}
    {After binding to address, \program{httpyd} switches to the uid of
    \var{user}, if given.}
    {}

\end{tableiii}


\program{httpyd} can also take its \var{mode} and \var{threads} parameters from
the environment, in the \envvar{HTTPY_MODE} and \envvar{HTTPY_THREADS}
variables. The command line options override the environment variables.


\section{Runtime Behavior}

\program{httpyd} serves static files from the filesystem. It supports 304 Not
Modified responses in staging and deployment modes.




% These are adding <Image> things to the last page, and I haven't cleaned up the
% inputs for indexing anyway.
%%
%%  The ugly "%begin{latexonly}" pseudo-environments are really just to
%%  keep LaTeX2HTML quiet during the \renewcommand{} macros; they're
%%  not really valuable.
%%
%%  If you don't want the Module Index, you can remove all of this up
%%  until the second \input line.
%%
%%begin{latexonly}
%\renewcommand{\indexname}{Module Index}
%%end{latexonly}
%\input{mod\jobname.ind}		% Module Index
%
%%begin{latexonly}
%\renewcommand{\indexname}{Index}
%%end{latexonly}
%\input{\jobname.ind}			% Index

\end{document}
