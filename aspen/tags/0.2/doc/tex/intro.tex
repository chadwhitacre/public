\chapter{Introduction \label{introduction}}

Aspen is a web server, designed around these two main goals:

\begin{itemize}
\item{To equally support publication, application, and hybrid websites}
\item{To be equally easy for every member of a web team to install and administer}
\end{itemize}


\section{Kinds of Website \label{kinds-of-website}}

Fundamental to Aspen's design is the idea that there are basically two kinds of
websites, publications and applications, differentiated by their organization
and interface models. A \dfn{publication} website organizes information into
individual pages within a hierarchical folder structure that one navigates by
browsing. In an \dfn{application} website, on the other hand, data is not
organized into hierarchical pages but is dealt with via a non-browsing interface
such as a search box.

The HTML version of this documentation is an example of a publication website: a
number of hypertext documents organized into sections. If we weren't using LaTeX
(or if I knew how to use it better), the sections would probably be encoded in
folders. \ulink{Gmail}{http://mail.google.com/mail} is a pure application
website, one which organizes and presents information non-hierarchically. Most
websites, however, are hybrids. That is, within an overall hierarchical
organization you will find both individual pages of information as well as
applications such as a site search feature, or a threaded discussion forum.

Publication websites are actually a subset of application websites, of course.
An application site can use any interface metaphor; a publication is an
application that uses the familiar folder/page metaphor to organize and present
its information. Therefore, every website is fundamentally an application.

Aspen enables the full range of websites: publications, applications, and
hybrids. It uses the filesystem for the hierarchical structure of publication
and hybrid websites, and provides a mechanism for including applications within
that hierarchy.

\section{Many Instances, One Application \label{many-instances-one-application}}

Any given website exists in multiple instances simultaneously. There is the
production version, possibly involving multiple load-balanced website instances.
There may be staging versions. There are probably one or more development
versions, possibly shared, possibly not. Some instances are managed by
professional system administrators. Others may be managed by developers working
on the site's logic, or designers working on the site's user interface. It may
even be the case that a tech-savvy content manager or Internet marketer would
manage a website instance as part of their work. Furthermore, different team
members will no doubt prefer different operating systems: Windows, Mac OS,
Linux, BSD.

Aspen simplifies this complexity. It is robust and configurable enough for
deployment, but simple and portable enough for development. The result is that
the process of building and deploying a website is more efficient with Aspen
than with other webservers.


\section{Basic and Extended Websites \label{basic-and-extended-websites}}

Under Aspen, a website is primarily a collection of files, self-contained within
a single directory, called the \dfn{root} of the website (cf. \ulink{Apache's
DocumentRoot
directive}{http://httpd.apache.org/docs/1.3/mod/core.html#documentroot}). URLs
map directly to the filesystem, so given a root of
\file{/usr/local/www/example.com}, a request for \file{/foo.html} would serve a
file at \file{/usr/local/www/example.com/foo.html}. If all you want to do is
serve static files, then this is all you need to know.

To extend a website, you use a UNIX-style userland located within a directory
under the website root named \dfn{__} (that's two underscores), also called the
website's \dfn{magic directory}. The existence and contents of this directory
are safe from prying eyes, because Aspen will respond to any requests mapping to
the magic directory with a \ulink{\code{404 Not
Found}}{http://www.w3.org/Protocols/rfc2616/rfc2616-sec10.html#sec10.4.5}.
