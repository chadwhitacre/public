\chapter{User Interface \label{interface}}

Users interface with Aspen through three mechanisms: the command line, a
configuration file, and the environment. Where a program parameter is set in
more than one of these contexts, they take precedence in the order given here.
For example, a \var{mode} option on the command line will override any
\var{mode} setting in the config file or in the environment.


\section{Command Line \label{command-line}}

Usage:

\begin{verbatim}
aspen [options] [command]
\end{verbatim}

Aspen takes one optional positional argument, \var{command}, which must be one
of: \code{start}, \code{stop}, \code{restart}, or \code{runfg}. The default is
\code{runfg}, which causes Aspen to run in the foreground, sending any access
log messages to stdout, and any error log messages to stderr.

\code{start}, \code{stop}, and \code{restart} control Aspen as a daemon, via a
pid file at \file{__/var/aspen.pid}. When run as a daemon, access log messages
are sent to a file at \file{__/var/log/access-x.log}, and error messages to
\file{__/var/log/error-x.log}, where \code{x} is a timestamp of the file's
creation time. These log files are rotated when they reach 1MB in size.


Aspen's command-line options are a subset of the options available in the config
file:

\begin{tableiii}{l|l|l}{var}{Option}{Description}{Default}

\lineiii{\programopt{-a}/\longprogramopt{-address}=\var{address}}
    {The address to which Aspen should bind. If \var{address} begins
    with a dot or a forward slash, then it is interpreted as an AF_UNIX socket.
    Otherwise, it is interpreted as an AF_INET socket. If \var{address} begins
    with a colon, then the loopback address is assumed.} {\code{:8080}}

\lineiii{\programopt{-l}/\longprogramopt{-log_filter}=\var{log_filter}}
    {A subsystem filter to apply to the error log, per the logging module.}
    {\code{}}

\lineiii{\programopt{-m}/\longprogramopt{-mode}=\var{mode}}
    {One of \code{development}, \code{debugging}, \code{staging}, or
    \code{deployment}. See mode.py.} {\code{development}}

\lineiii{\programopt{-r}/\longprogramopt{-root}=\var{root}}
    {The directory containing the website for Aspen to serve.}
    {\code{.}}

\lineiii{\programopt{-v}/\longprogramopt{-log_level}=\var{log_level}}
    {The error log level. Valid options per the logging module
    are (case-insensitive): \code{notset}, \code{debug}, \code{info},
    \code{warning}, \code{error}, \code{critical}.}{\code{warning}}

\end{tableiii}


\section{Configuration File \label{config-file}}

This section describes the general Aspen configuration file at
\file{__/etc/aspen.conf}. Additional configuration files are described in the
Extending Aspen chapter. \file{aspen.conf} is in \file{.ini}-style format per
the ConfigParser module. There are five sections recognized: \code{DEFAULT},
\code{debugging}, \code{development}, \code{staging}, and \code{production}. Any
of the below settings can be given in any section, except for \var{mode}, which
can only occur in \code{DEFAULT}. However, only two sections will be used at any
given time: \code{DEFAULT}, and the section corresponding to the current
deployment mode (see The Environment for more on mode).

\begin{tableiii}{l|l|l}{var}{Option}{Description}{Default}

\lineiii{address}{The address to which Aspen should bind. If \var{address}
begins with a dot or a forward slash, then it is interpreted as an AF_UNIX
socket. Otherwise, it is interpreted as an AF_INET socket. If \var{address}
begins with a colon, then the loopback address is assumed.}{\code{:8080}}

\lineiii{defaults}{A comma-separated list of names to look for when a directory
is requested.}{\code{index.html, index.htm, index.py}}

\lineiii{group}{A groupname or gid to which, if given, Aspen will attempt to
switch after binding to the socket.}{}

\lineiii{log_access}{Whether or not to maintain an access log. Valid options
are (case-insensitive): \code{yes}, \code{no}, \code{none}, \code{true},
\code{false}, \code{0}, \code{1}. The access log will be in Apache's Combined
Log Format.}{\code{no}}

\lineiii{log_format}{The format of error log messages, per the logging
module.}{\code{\%(levelname)s:\%(name)s:\%(message)s}}

\lineiii{log_level}{The error log level. Valid options per the logging module
are (case-insensitive): \code{notset}, \code{debug}, \code{info},
\code{warning}, \code{error}, \code{critical}.}{\code{warning}}

\lineiii{log_filter}{A subsystem filter to apply to the error log, per the
logging module.}{}

\lineiii{mode}{One of \code{debugging}, \code{development}, \code{staging}, or
\code{production}. (Naturally, this option only obtains in the DEFAULT
section.)}{\code{development}}

\lineiii{threads}{The number of threads to maintain in the request-servicing
thread pool.}{\code{10}}

\lineiii{user}{A username or uid to which, if given, Aspen will attempt to
switch after binding to the socket.}{}

\end{tableiii}



see also: Extending Aspen, ConfigParser, mode, logging

\section{The Environment \label{environment}}

Aspen incorporates the \module{mode} module, which uses the \envvar{PYTHONMODE}
environment variable to model the application life-cycle through four deployment
modes: \code{debugging}, \code{development}, \code{staging}, and
\code{production}. This module is available to your applications at
\module{aspen.mode}. For more information, see the \module{mode} documentation.

see also: mode.py
