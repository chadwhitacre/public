\chapter{User Interface \label{interface}}

\section{Command Line \label{command-line}}

\program{aspen} exposes the following command-line options:

\begin{tableiii}{l|l|l}{var}{Option}{Description}{Default}

\lineiii{\programopt{-a}/\longprogramopt{-address}=\var{address}}
    {The address to which \program{aspen} should bind. If \var{address} begins
    with a dot or a forward slash, then it is interpreted as an AF_UNIX socket.
    Otherwise, it is interpreted as an AF_INET socket. If \var{address} begins
    with a colon, then the loopback address is assumed.} {\code{:8080}}

\lineiii{\programopt{-m}/\longprogramopt{-mode}=\var{mode}}
    {\var{mode} is one of the strings 'development', 'debugging', 'staging', or
    'deployment'.  The \envvar{HTTPY_MODE} environment variable will be set to
    this value, and will be available to your responders via the
    \class{aspen.mode} object.} {\code{development}}

\end{tableiii}


\section{Configuration File \label{config-file}}

This section describes the general Aspen configuration file at
\file{__/etc/aspen.conf}. Additional configuration files are described in the
Extending Aspen chapter.


\section{The Environment \label{environment}}

PYTHONMODE

see also: mode.py
