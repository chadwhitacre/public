\chapter{Tutorial \label{tutorial}}

Once you have installed Aspen, here are some quick walk-throughs to get your
feet wet. They are written sequentially.


\section{"Greetings, program!" \label{tutorial-greetings-program}}

Make a new directory named \file{aspentut}. Create a file in \file{aspentut}
named \file{index.html}, with the following contents:

\begin{verbatim}
Greetings, program!
\end{verbatim}

At the command line in the \file{aspentut} directory, type \code{aspen}. You
should get output like this:

\begin{verbatim}
$ aspen
aspen started on port 8080
\end{verbatim}

Now open a web browser and hit \code{http://localhost:8080/}. You should see
"Greetings, program!" in your browser. Congratulations!


\section{A Python Script \label{tutorial-pyscript}}

Aspen supports pseudo-CGI-style programming with regular python scripts. In your
\file{aspentut} directory, create a file named \file{foo.py} with the following
contents:

\begin{verbatim}
response.body = "Greetings, program!"
\end{verbatim}

Assuming that \program{aspen} is still running, hit
\code{http://localhost:8080/foo.py} in your browser. You should again see
"Greetings, program!"


\section{Your First Handler \label{tutorial-handler}}

Aspen uses \dfn{handlers} to process your \file{index.html} and \file{foo.py}
files. Now we are going to write our own handler.

First, create a directory under \file{aspentut} named \file{__} (that's two
underscores). This is aspentut's \dfn{magic directory}, and it is where you
configure and extend your website. Now create two directories under the magic
directory: \file{etc} and \file{src}. Your directory structure should now look
like this:

\begin{verbatim}
aspentut
aspentut/__
aspentut/__/etc
aspentut/__/src
\end{verbatim}

In \file{__/src}, create a file named \file{handy.py} with the following
contents:

\begin{verbatim}
def handle(environ, start_response):
    return environ['aspen.fp'].name
\end{verbatim}

And in \file{__/etc}, create a file named \file{handlers.conf} with these
contents:

\begin{verbatim}
fnmatch aspen.rules.fnmatch

[handy:handle]
fnmatch *.asp
\end{verbatim}

What we have done is we have defined a new handler, and wired it up to be used
for any request for a file with the extension \file{.asp}. So now let's create
such a file at \file{aspentut/handled.asp} and give it the following contents:

\begin{verbatim}
"Greetings, program?"
\end{verbatim}

Now hit \code{http://localhost:8080/handled.asp}. You should see the pathname of
the file being served. That's because \code{environ['aspen.fp']} points to an
open \class{file} object of that file.

If you are familiar with the WSGI specification, you will recognize that
\function{handy.handle} is very nearly a WSGI callable. Aspen plugins all speak
a slight superset of WSGI. Also notice that the rules for when a certain handler
is invoked are themselves extensible: in addition to writing handlers, you can
write your own rules.


\section{What You've Learned \label{tutorial-learned}}

In this brief tutorial we've introduced these key facts about Aspen:

\begin{itemize}
\item{Aspen websites use the filesystem for site hierarchy.}
\item{Aspen websites are extended and configured via a "magic directory."}
\item{Aspen extensions are slightly extended WSGI callables.}
\item{Aspen configuration happens through plain-text configuration files.}
\end{itemize}

Aspen also supports wiring up arbitrary WSGI apps at certain paths, and
maintaining a global WSGI middleware stack. If this all fits your style of
development, then check out the reference documentation that follows for the
full story.
