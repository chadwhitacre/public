% Complete documentation on the extended LaTeX markup used for Python
% documentation is available in ``Documenting Python'', which is part
% of the standard documentation for Python.  It may be found online
% at:
%
%     http://www.python.org/doc/current/doc/doc.html

\documentclass{manual}

\title{aspen}

\author{Chad W. L. Whitacre}

% Please at least include a long-lived email address;
% the rest is at your discretion.
\authoraddress{
	Zeta Design \&\ Development \\
	\url{http://www.zetadev.com/software/aspen/} \\
	Email: \email{\ulink{chad@zetaweb.com}{mailto:chad@zetaweb.com}}
}

%\date{January 1, 1970} % update before release!
\date\today
				% Use an explicit date so that reformatting
				% doesn't cause a new date to be used.  Setting
				% the date to \today can be used during draft
				% stages to make it easier to handle versions.

\release{0.2}			% release version; this is used to define the
				% \version macro

\makeindex			% tell \index to actually write the .idx file
\makemodindex			% If this contains a lot of module sections.


\begin{document}

\maketitle

\begin{abstract}

\noindent
Aspen is a web server. It serves highly extensible Python-based publication,
application, and hybrid websites.

\end{abstract}

\chapter{Introduction \label{introduction}}

Aspen is designed around the idea that there are basically two kinds of
websites, publications and applications, differentiated by their organization
and interface models. A \dfn{publication} website organizes information into
individual pages within a hierarchical folder structure that one navigates by
browsing. In an \dfn{application} website, on the other hand, data is not
organized into hierarchical pages but is dealt with via a non-browsing interface
such as a search box.

The HTML version of this documentation is an example of a publication website: a
number of hypertext documents organized into sections. If we weren't using LaTeX
(or if I knew how to use it better), the sections would probably be encoded in
folders. \ulink{Gmail}{http://mail.google.com/mail} is a pure application
website, one which organizes and presents information non-hierarchically. Most
websites, however, are hybrids. That is, within an overall hierarchical
organization you will find both individual pages of information as well as
applications such as a site search feature, or a threaded discussion forum.

Publication websites are actually a subset of application websites, of course.
An application site can use any interface metaphor; a publication is an
application that uses the familiar folder/page metaphor to organize and present
its information. Therefore, every website is fundamentally an application.

Aspen enables the full range of websites: publications, applications, and
hybrids. It uses the filesystem for the hierarchical structure of publication
and hybrid websites, and provides a mechanism for including applications within
that hierarchy.

An Aspen website is a collection of files, self-contained within a single
directory, called the \dfn{root} of the website (cf. \ulink{Apache's
\code{DocumentRoot}
directive}{http://httpd.apache.org/docs/1.3/mod/core.html#documentroot}). In
general, URLs map directly to the filesystem. That is, given a root of:

\begin{verbatim}
/usr/local/www/example.com
\end{verbatim}

A request for \file{/foo.html} would serve a file at:

\begin{verbatim}
/usr/local/www/example.com/foo.html
\end{verbatim}

If all you want to do is serve static files, then that's most of what you need
to know.

To extend an Aspen website, you use a \UNIX{}-style userland located within a
directory under the website root named \dfn{__} (that's two underscores), also
called the website's \dfn{magic directory}. The existence and contents of this
directory are safe from prying eyes, because Aspen will respond to any requests
mapping to the magic directory with a \ulink{\code{404 Not
Found}}{http://www.w3.org/Protocols/rfc2616/rfc2616-sec10.html#sec10.4.5}.

\chapter{Extending Aspen}

Aspen uses Python's WSGI specification for its extension architecture. There are
three categories of extension:

\begin{tableii}{l|l}{}{Category}{Explanation}
\lineii{applications}{applications are connected to directories within the site
    hierarchy; only one app touches any given request}
\lineii{handlers}{handlers are tied to individual resources (i.e., files) based
    on extensible rules; only one handler touches any given request}
\lineii{middleware}{one or many middleware applications may be specified;
    all middleware generally touches every request}
\end{tableii}

All extensions are WSGI callables, connected to the above entry points with
three configuration files in \file{__/etc}:

\begin{itemize}
\item{\file{apps.conf}}
\item{\file{handlers.conf}}
\item{\file{middleware.conf}}
\end{itemize}

Where called for in these files, objects are specified in a notation derived
from setuptools' entry_points feature: a dotted module name, followed by a colon
and a dotted identifier naming an object within the module. This is referred to
below as \dfn{colon notation}. The following example would import the \code{bar}
object from \code{example.package.foo}, and use its \code{baz} attribute (a WSGI
callable):

\begin{verbatim}
example.package.foo:bar.baz
\end{verbatim}

Wherever in these files a name in colon notation points to a class, that class
is instantiated with the current \class{Website} instance as its sole positional
argument, and the instance is used as the named object.

The comment character for these files is \#, and comments can be included
in-line. Blank lines are ignored, as is initial and trailing whitespace
per-line. Where section names are called for, they are given in brackets.


\section{Applications: Path-based Extension \label{apps}}

In Aspen, an \dfn{application} or \dfn{app} refers to a WSGI application that is
connected to a particular directory. Apps are set up in \file{__/etc/apps.conf}.

The \file{__/etc/apps.conf} file contains a newline-separated list of
white-space-separated path name/object name pairs. The path names refer to
URL-space, and are translated literally to the filesystem. If the trailing slash
is given, then requests for that directory will first be redirected to the
trailing slash before being handed off to the application. If no trailing slash
is given, the application will also get requests without the slash. When
choosing an application to service a request, the most specific pathname matches
first.

Object names are in colon notation, and they name WSGI callables.

Aspen will (over)write a file called \file{README.aspen} in each directory
mentioned in \file{apps.conf}, containing the relevant line from
\file{apps.conf}. If the directory does not exist, it is created.


\subsection{Example apps.conf}

\begin{verbatim}
/foo        example.apps:foo    # will get both /foo and /foo/
/bar/       example.apps:bar    # /bar will redirect to /bar/
/bar/baz    example.apps:baz    # will 'steal' some of /bar's requests
\end{verbatim}


\section{Handlers: Resource-based Extension \label{handlers}}

Aspen \dfn{handlers} are WSGI applications that are associated with files on the
filesystem according to arbitrary rules. This enables ASP/PHP-style web
development, where URLs map literally to the filesystem, and the response is
generated by somehow processing a filesystem resource.

The \file{__/etc/handlers.conf} file begins with an anonymous "rules" section,
which is a newline-separated list of white-space-separated rule name/object name
pairs. Rule names can be any string without whitespace. Each object name (in
colon notation) specifies a \dfn{rule}, a callable taking a Python file object
and an arbitrary predicate string, and returning \class{True} or \class{False}.

Following the rule specification are sections specifying \dfn{handlers}, which
as mentioned above are WSGI callables. When called, handlers receive the
following additional objects in \code{environ}:

\begin{tableii}{l|l}{var}{key}{value}
\lineii{aspen.fp}{the filesystem resource as a \class{file} object, positioned at zero}
\lineii{aspen.website}{the \class{Website} instance}
\end{tableii}

The name of each section specifies a handler (a WSGI callable) in colon
notation. The body of each section is a newline-separated list of conditions
under which this handler is to be called. Fundamentally, these conditions are
made up of a rule name as defined at the beginning of the file, and an arbitrary
predicate string (which can include whitespace) that is meaningful to the
matching rule callable. If no predicate is given, then the rule callable will
receive \class{None} for its predicate argument. Rules must be explicitly
specified at the beginning of the file before being available within handler
sections. After the first condition in a handler section, additional condition
lines must begin with one of \code{AND}, \code{OR}, or \code{NOT}. These
case-insensitive tokens specify how conditions are to be combined in evaluating
whether to use this handler.

On each request, handlers are considered in the order given, and the first
matching handler is used. Only one handler is used for any given request.

Note that if the file \file{__/etc/handlers.conf} exists at all, the defaults
(see the example below) disappear, and you must respecify any of the default
rules in your own file if you want them.


\subsection{Example handlers.conf}
This is Aspen's default handler configuration:

\begin{verbatim}
fnmatch     aspen.rules:fnmatch
hashbang    aspen.rules:hashbang
mime-type   aspen.rules:mimetype


[aspen.handlers:HTTP404]
fnmatch *.py[cod]           # hide any compiled Python scripts


[aspen.handlers:pyscript]
    fnmatch     *.py        # exec python scripts ...
OR  hashbang                # ... and anything starting with #!


[aspen.handlers:Simplate]
mime-type text/html         # run html files through the Simplates engine


[aspen.handlers:static]
fnmatch *                   # anything else, serve it statically
\end{verbatim}


\section{Middleware: Global Extension \label{middleware}}

Aspen allows for a full WSGI middleware stack, configured via the
\file{__/etc/middleware.conf} file. This is simply a newline-separated list of
middleware specifiers in colon notation. The first-mentioned middleware will be
the outer-most in the stack (i.e., closest to the browser).

Aspen adds an \class{httpy.Responder} instance to the bottom of the stack.

\subsection{Example middleware.conf}

\begin{verbatim}
example.foo:bar # called as is
example.baz:Buz # will be instantiated
\end{verbatim}



% These are adding <Image> things to the last page, and I haven't cleaned up the
% inputs for indexing anyway.
%%
%%  The ugly "%begin{latexonly}" pseudo-environments are really just to
%%  keep LaTeX2HTML quiet during the \renewcommand{} macros; they're
%%  not really valuable.
%%
%%  If you don't want the Module Index, you can remove all of this up
%%  until the second \input line.
%%
%%begin{latexonly}
%\renewcommand{\indexname}{Module Index}
%%end{latexonly}
%\input{mod\jobname.ind}		% Module Index
%
%%begin{latexonly}
%\renewcommand{\indexname}{Index}
%%end{latexonly}
%\input{\jobname.ind}			% Index

\end{document}
