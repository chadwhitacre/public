\chapter{Extending Aspen}

Aspen is extensible with the Python programming language. There are three
categories of extension:

\begin{tableii}{l|l}{}{Category}{Explanation}
\lineii{hooks}{one or many global hooks may be specified; all hooks touch every
    request}
\lineii{applications}{applications are connected to directories within the site
    hierarchy; only one app touches any given request}
\lineii{handlers}{handlers are tied to resources (i.e., files) based on
    extensible rules; only one handler touches any given request}
\end{tableii}

The basic pattern is that all extensions are Python callables with a certain
signature. You connect these callables to the various entry points with three
configuration files in \file{__/etc}:

\begin{itemize}
\item{\file{hooks.conf}}
\item{\file{handlers.conf}}
\item{\file{apps.conf}}
\end{itemize}

Where called for in these files, objects are specified in dot notation. If a
given name has no dot, it is imported as a module/package. If it does contain a
dot, then the last part becomes the \code{import} target, and the remaining
dotted portion becomes the \code{from} target. E.g.:

\begin{verbatim}
example.foo.bar => from example.foo import bar
\end{verbatim}

The comment character for these files is \#, and comments can be included
in-line. Blank lines are ignored, as is initial and trailing whitespace
per-line.

\section{Global Extension through Hooks}

Aspen defines four points at which you may insert callables conforming to a
specified signature. The hooks, their positional arguments, and return values
are as follows:

\begin{tableiii}{l|l|l}{}{Hook Name}{Arguments}{Return Value}
\lineiii{startup}{\var{website}}{undefined (ignored)}
\lineiii{request}{\var{website}, \var{request}}{\class{httpy.Request}}
\lineiii{response}{\var{website}, \var{request}, \var{response}}{\class{httpy.Response}}
\lineiii{exception}{\var{website}, \var{request}, \var{response}}{\class{httpy.Response}}
\end{tableiii}

The \file{__/etc/hooks.conf} file gives you fine-grained control over any Python
callables that should be invoked at each hook. The file contains a list of
Python dotted names, split into 5 sections: an initial anonymous section, and
four named sections. All sections are optional.

For each object named in any initial anonymous section, Aspen will look for
callable attributes with names corresponding to each hook. Callables specified
in this manner are called in forward order for the startup and request hooks,
and in reverse order for the response and exception hooks. Modules need not
define a callable for every hook.

Names for named sections are given in brackets, and correspond to the hook names
given above. The dotted names within each section must point to callable
objects; they will be called in the order specified.

\subsection{Example hooks.conf \label{hooks.conf}}

\begin{verbatim}
foo.hooks   # will look for {exception,request,response,startup}
bar.hooks   #   in these modules/objects

[startup]
  example.hooks.startup

[exception]
  OtherExample.ErrorHandling.hook
\end{verbatim}

\section{Resource-based Extension through Handlers}

Aspen provides an extensible mechanism for supporting ASP/PHP-style web
development, where URLs map literally to the filesystem, and the response is
generated by somehow processing a filesystem resource.

The \file{__/etc/handlers.conf} file begins with a newline-separated list of
white-space-separated rule name/object name pairs. Rule names can be any string
without whitespace. Each object name must specify a Python class, instance,
module/package, or function. In each case we are looking for a \dfn{rule}, a
callable that takes a Python file object and an arbitrary predicate string, and
returns \class{True} or \class{False}. Here is how each case is treated:

\begin{tableii}{l|l}{}{Object Type}{Behavior}
\lineii{class}{instantiated with the website instance as a positional
               argument; a \var{rule} attribute is the callable}
\lineii{instance, module, or package}{a \var{rule} attribute is the callable}
\lineii{function}{the function itself is the callable}
\end{tableii}

Following the rule specification are sections specifying \dfn{handlers}, which
are Python callables that, given the following keyword arguments, return an
\class{httpy.Response} object:

\begin{tableii}{l|l}{var}{Argument}{Description}
\lineii{fp}{the filesystem resource as a file object, positioned at zero}
\lineii{request}{the current \class{httpy.Request} object}
\lineii{response}{a newly-created \class{httpy.Response} object}
\lineii{website}{the website instance}
\end{tableii}

Handlers are specified in brackets as the name of each section. Following each
such name is a newline-separated list of conditions under which this handler is
to be called. Fundamentally, these conditions are made up of a rule name as
defined at the beginning of the file, and an arbitrary predicate string that is
meaningful to the matching rule callable. If no predicate is given, then the
rule callable will receive \class{None} for its predicate argument. Rules must
be explicitly specified at the beginning of the file before being available
within handler sections.


After the first condition in a handler section, additional condition lines must
begin with one of \code{AND}, \code{OR}, or \code{NOT}. These case-insensitive
tokens specify how conditions are to be combined in evaluating whether to use
this handler. On each request, handlers are considered in the order given, and
the first matching handler is used. Only one handler is used for any given
request.



\subsection{Example handlers.conf \label{handlers.conf}}
This is Aspen's default handler configuration:

\begin{verbatim}
fnmatch     aspen.rules.fnmatch
hashbang    aspen.rules.hashbang
mime-type   aspen.rules.mimetype


[aspen.handlers.HTTP404]
fnmatch *.py[cod]           # hide any compiled Python scripts


[aspen.handlers.pyscript]
    fnmatch     *.py        # exec python scripts ...
OR  hashbang                # ... and anything with a hashbang


[aspen.handlers.Simplate]
mime-type text/html         # run html files through the Simplates engine


[aspen.handlers.static]
fnmatch *                   # anything else, serve it statically
\end{verbatim}

Note that if the file \file{__/etc/handlers.conf} exists at all, these defaults
disappear, and you must respecify these rules in your own file if you want them.

\section{Path-based Extension through Applications}

Whereas handlers allow for file-based extensions, the application mechanism is
for directory-based extensions.

The \file{__/etc/apps.conf} file contains a newline-separated list of
white-space-separated path name/object name pairs. The path names refer to
URL-space, but must be reflected on the filesystem (or \class{AppsConfError}
will be raised). If the trailing slash is given, then requests for that
directory will first be redirected to the trailing slash before being handed off
to the application. If no trailing slash is given, the application will also get
requests without the slash. Applications match in the order specified.

Each object name must specify a Python class, instance, module, or
function in dotted notation. In each case we are looking for a callable
that takes a request object and returns a response object. Here's how
each is treated:

\begin{tableii}{l|l}{}{Object Type}{Behavior}
\lineii{class}{instantiated with the website instance as a positional
               argument; a \var{respond} attribute is the callable}
\lineii{instance, module, or package}{a \var{respond} attribute is the callable}
\lineii{function}{the function itself is the callable}
\end{tableii}


\subsection{Example apps.conf \label{apps.conf}}

\begin{verbatim}
/foo        example.apps.foo    # will get both /foo and /foo/
/bar/       example.apps.bar    # /bar will redirect to /bar/
/bar        example.apps.Bar    # will never be called
/bar/baz    example.apps.baz    # also never called
\end{verbatim}

If it does not already exist, Aspen will place a file called \file{README.aspen}
in each directory mentioned in \file{apps.conf}, containing the relevant line
from \file{apps.conf}. If the directory does not exist, we raise
\class{AppsConfError}.
