\section{Mail Commands}



\subsection{AUTO (AUTOcompletion of email addresses)}

 The AUTO command is used by clients which want to request a list of email
recipients whose names or email addresses match a partial string supplied by
the client.  This string is the only parameter passed to this command.  The
command will return ERROR if no user is logged in or if no address book could
be found; otherwise, it returns LISTING_FOLLOWS followed by zero or more
candidate recipients.



\subsection{IGAB (Initialize Global Address Book)}

 This command creates, or re-creates, a database of Internet e-mail addresses
using the vCard information in the Global Address Book room.  This procedure
is normally run internally when the server determines it necessary, but is
also provided as a server command to be used as a troubleshooting/maintenenance
tool.  Only a system Aide can run the command.  It returns OK on success or
ERROR on failure.



\subsection{ISME (find out if an e-mail address IS ME)}

 This is a quickie shortcut command to find out if a given e-mail address
belongs to the user currently logged in.  Its sole argument is an address to
parse.  The supplied address may be in any format (local, IGnet, or Internet).
The command returns OK if the address belongs to the user, ERROR otherwise.



\subsection{QDIR (Query global DIRectory)}

 Look up an internet address in the global directory.  Any logged-in user may
call QDIR with one parameter, the Internet e-mail address to look up.  QDIR
returns OK followed by a Citadel address if there is a match, otherwise it
returns ERROR+NOT_LOGGED_IN.



\subsection{*SMTP (utility commands for the SMTP gateway)}

 This command, accessible only by Aides, supports several utility operations
which examine or manipulate Citadel's SMTP support.  The first command argument
is a subcommand telling the server what to do.  The following subcommands are
supported:

      SMTP mx|hostname             (display all MX hosts for 'hostname')
      SMTP runqueue                (attempt immediate delivery of all messages
                                    in the outbound SMTP queue, ignoring any
                                    retry times stored there)



