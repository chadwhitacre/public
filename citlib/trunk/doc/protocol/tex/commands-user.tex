\section{User Commands}



\subsection{*AGUP/ASUP (Administrative Get/Set User Parameters)}

 These commands are only executable by Aides and by server extensions running
at system-level.  They are used to get/set any and all parameters relating to
a user account.  AGUP requires only one argument: the name of the user in
question.  SGUP requires all of the parameters to be set.  The parameters are
as follows, and are common to both commands:

 0 - User name
 1 - Password
 2 - Flags (see citadel.h)
 3 - Times called
 4 - Messages posted
 5 - Access level
 6 - User number
 7 - Timestamp of last call
 8 - Purge time (in days) for this user (or 0 to use system default)

 Upon success, AGUP returns OK followed by all these parameters, and ASUP
simply returns OK.  If the client has insufficient access to perform the
requested operation, ERROR+HIGHER_ACCESS_REQUIRED is returned.  If the
requested user does not exist, ERROR+NO_SUCH_USER is returned.



\subsection{*CHEK (CHEcK various things)}

 When logging in, there are various things that need to be checked.   This
command will return ERROR+NOT_LOGGED_IN if no user is logged in.  Otherwise
it returns OK and the following parameters:

 0: Number of new private messages in Mail>
 1: Nonzero if the user needs to register
 2: (Relevant to Aides only) Nonzero if new users require validation
 3: The user's preferred Internet e-mail address



\subsection{CREU (CREate new User account)}

 This command creates a new user account AND DOES NOT LOG IT IN.  The first
argument to this command will be the name of the account.  No case conversion
is done on the name.  Note that the new account is installed with a default
configuration, and no password.  The second argument is optional, and will be
an initial password for the user.  This command returns OK if the account was
created, ERROR + HIGHER_ACCESS_REQUIRED if the user is not an Aide, ERROR +
USERNAME_REQUIRED if no username was specified, or ERROR + ALREADY_EXISTS if
another user already exists with this name.

 Please note that CREU is intended to be used for activities in which a
system administrator is creating user accounts.  For self-service user
account creation, use the NEWU command.



\subsection{EBIO (Enter BIOgraphy)}

 Transmit to the server a free-form text file containing a little bit of
information about the user for other users to browse.  This is typically
referred to as a 'bio' online.  EBIO returns SEND_LISTING if it succeeds,
after which the client is expected to transmit the file, or any of the usual
ERROR codes if it fails.



\subsection{*GETU (GET User configuration)}

 This command retrieves the screen dimensions and user options for the
currently logged in account.  ERROR + NOT_LOGGED_IN will be returned if no
user is logged in, of course.  Otherwise, OK will be returned, followed by
four parameters.  The first parameter is the user's screen width, the second
parameter is the user's screen height, and the third parameter is a bag of
bits with the following meanings:

 \#define US_LASTOLD	16		/* Print last old message with new  */
 \#define US_EXPERT	32		/* Experienced user		    */
 \#define US_UNLISTED	64		/* Unlisted userlog entry           */
 \#define US_NOPROMPT	128		/* Don't prompt after each message  */
 \#define US_DISAPPEAR	512		/* Use "disappearing msg prompts"   */
 \#define US_PAGINATOR	2048		/* Pause after each screen of text  */

 There are other bits, too, but they can't be changed by the user (see below).



\subsection{GNUR (Get Next Unvalidated User)}

 This command shows the name of a user that needs to be validated.  If there
are no unvalidated users, OK is returned.  Otherwise, MORE_DATA is returned
along with the name of the first unvalidated user the server finds.  All of
the usual ERROR codes may be returned as well (for example, if the user is
not an Aide and cannot validate users).

 A typical "Validate New Users" command would keep executing this command,
and then validating each user it returns, until it returns OK when all new
users have been validated.



\subsection{*GREG (Get REGistration for user)}

 This command retrieves the registration info for a user, whose name is the
command's sole argument.  All the usual error messages can be returned.  If
the command succeeds, LISTING_FOLLOWS is returned, followed by the user's name
(retrieved from the userlog, with the right upper and lower case etc.)  The
contents of the listing contains one field per line, followed by the usual
000 on the last line.

 The following lines are defined.  Others WILL be added in the futre, so all
software should be written to read the lines it knows about and then ignore
all remaining lines:

 Line 1:  User number
 Line 2:  Password
 Line 3:  Real name
 Line 4:  Street address or PO Box
 Line 5:  City/town/village/etc.
 Line 6:  State/province/etc.
 Line 7:  ZIP Code
 Line 8:  Telephone number
 Line 9:  Access level
 Line 10: Internet e-mail address
 Line 11: Country

 Users without Aide privileges may retrieve their own registration using
this command.  This can be accomplished either by passing the user's own
name as the argument, or the string "_SELF_".  The command will always
succeed when used in this manner, unless no user is logged in.



\subsection{*HCHG (Hostname CHanGe)}

 HCHG is a command, usable by any user, that allows a user to change their RWHO
host value.  This will mask a client's originating hostname from normal
users; access level 6 and higher can see, in an extended wholist, the actual
hostname the user originates from.

 The format of an HCHG command is:

 HCHG <name>

 If a HCHG command is successful, the value OK (200) is returned.



\subsection{LBIO (List users who have BIOs on file)}

 This command is self-explanatory.  Any user who has used EBIO to place a bio
on file is listed.  LBIO almost always returns LISTING_FOLLOWS followed by
this listing, unless it experiences an internal error in which case ERROR
is returned.



\subsection{*LIST (user LISTing)}

 This is a simple user listing.  It always succeeds, returning
LISTING_FOLLOWS followed by zero or more user records, 000 terminated.  The
fields on each line are as follows:

 1. User name
 2. Access level
 3. User number
 4. Date/time of last login (Unix format)
 5. Times called
 6. Messages posted
 7. Password (listed only if the user requesting the list is an Aide)

 Unlisted entries will also be listed to Aides logged into the server, but
not to ordinary users.

 The LIST command accepts an optional single argument, which is a simple,
case-insensitive search string.  If this argument is present, only usernames
in which the search string is present will be returned.  It is a simple
substring search, not a regular expression search.  If this string is empty
or not present, all users will be returned.



\subsection{LOUT (LogOUT)}

 Log out the user without closing the server connection.  It always returns
OK even if no user is logged in.



\subsection{NEWU (create NEW User account)}

 This command creates a new user account AND LOGS IT IN.  The argument to
this command will be the name of the account.  No case conversion is done
on the name.  Note that the new account is installed with a default
configuration, and no password, so the client should immediately prompt the
user for a password and install it with the SETP command as soon as this
command completes.  This command returns OK if the account was created and
logged in, ERROR + ALREADY_EXISTS if another user already exists with this
name, ERROR + NOT_HERE if self-service account creation is disabled,
ERROR + MAX_SESSIONS_EXCEEDED if too many users are logged in, ERROR +
USERNAME_REQUIRED if a username was not provided, or ERROR + ILELGAL_VALUE
if the username provided is invalid.  If OK, it will also return the same
parameters that PASS returns.

 Please note that the NEWU command should only be used for self-service
user account creation.  For administratively creating user accounts, please
use the CREU command.



\subsection{PASS (send PASSword)}

 The second step in logging in a user.  This command takes one argument: the
password for the user we are attempting to log in.  If the password doesn't
match the correct password for the user we specified for the USER command,
ERROR + PASSWORD_REQUIRED is returned.  If a USER command has not been
executed yet, ERROR + USERNAME_REQUIRED is returned.  If a user is already
logged in, ERROR + ALREADY_LOGGED_IN is returned.  If the password is
correct, OK is returned and the user is now logged in... and most of the
other server commands can now be executed.  Along with OK, the following
parameters are returned:

\begin{tableii}{l|l}{var}{Position}{Description}

\lineii{0}
    {The user's name (in case the client wants the right upperlower casing)}

\lineii{1}
    {The user's current access level}

\lineii{2}
    {Times called}

\lineii{3}
    {Messages posted}

\lineii{4}
    {Various flags (see citadel.h)}

\lineii{5}
    {User number}

\lineii{6}
    {Time of last call (UNIX timestamp)}

\end{tableii}



\subsection{QUSR (Query for a USeR)}

 This command is used to check to see if a particular user exists.  The only
argument to this command is the name of the user being searched for.  If
the user exists, OK is returned, along with the name of the user in the userlog
(so the client software can learn the correct upper/lower casing of the name
if necessary).  If the user does not exist, ERROR+NO_SUCH_USER is returned.
No login or current room is required to utilize this command.



\subsection{RBIO (Read BIOgraphy)}

 Receive from the server a named user's bio.  This command should be passed
a single argument - the name of the user whose bio is requested.  RBIO returns
LISTING_FOLLOWS plus the bio file if the user exists and has a bio on file.
The return has the following parameters:  the user name, user number, access
level, date of last call, times called, and messages posted.  This command
returns ERROR+NO_SUCH_USER if the named user does not exist.

 RBIO no longer considers a user with no bio on file to be an error condition.
It now returns a message saying the user has no bio on file as the text of the
bio.  This allows newer servers to operate with older clients.



\subsection{RCHG (Roomname CHanGe)}

 RCHG is a command, usable by any user, that allows a user to change their RWHO
room value.  This will mask a client's roomname from normal users; access
level 6 and higher can see, in an extended wholist, the actual room the user
is in.

 The format of an RCHG command is:

 RCHG <name>

 If a RCHG command is successful, the value OK (200) is returned.



\subsection{*REGI (send REGIstration)}

 Clients will use this command to transmit a user's registration info.  If
no user is logged in, ERROR+NOT_LOGGED_IN is returned.  Otherwise,
SEND_LISTING is returned, and the server will expect the following information
(terminated by 000 on a line by itself):

 Line 1:  Real name
 Line 2:  Street address or PO Box
 Line 3:  City/town/village/etc.
 Line 4:  State/province/etc.
 Line 5:  ZIP Code
 Line 6:  Telephone number
 Line 7:  e-mail address
 Line 8:  Country



\subsection{*RWHO (Read WHO's online)}

 Displays a list of all users connected to the server.  No error codes are
ever returned.  LISTING_FOLLOWS will be returned, followed by zero or more
lines containing the following three fields:

 0 - Session ID.  Citadel fills this with the pid of a server program.
 1 - User name.
 2 - The name of the room the user is currently in.  This field might not
be displayed (for example, if the user is in a private room) or it might
contain other information (such as the name of a file the user is
downloading).
 3 - (server v4.03 and above) The name of the host the client is connecting
from, or "localhost" if the client is local.
 4 - (server v4.04 and above) Description of the client software being used
 5 - The last time, locally to the server, that a command was received from
     this client (Note: NOOP's don't count)
 6 - The last command received from a client. (NOOP's don't count)
 7 - Session flags.  These are: + (spoofed address), - (STEALTH mode), *
     (posting) and . (idle).
 8 - Actual user name, if user name is masqueraded and viewer is an Aide.
 9 - Actual room name, if room name is masqueraded and viewer is an Aide.
 10 - Actual host name, if host name is masqueraded and viewer is an Aide.
 11 - Nonzero if the session is a logged-in user, zero otherwise.

 The listing is terminated, as always, with the string "000" on a line by
itself.



\subsection{SETP (SET new Password)}

 This command sets a new password for the currently logged in user.  The
argument to this command will be the new password.  The command always
returns OK, unless the client is not logged in, in which case it will return
ERROR + NOT_LOGGED_IN, or if the user is an auto-login user, in which case
it will return ERROR + NOT_HERE.



\subsection{*SETU (SET User configuration)}

 This command does the opposite of SETU: it takes the screen dimensions and
user options (which were probably obtained with a GETU command, and perhaps
modified by the user) and writes them to the user account.  This command
should be passed three parameters: the screen width, the screen height, and
the option bits (see above).  It returns ERROR + NOT_LOGGED_IN if no user is
logged in, and ERROR + ILLEGAL_VALUE if the parameters are incorrect.

 Note that there exist bits here which are not listed in this document.  Some
are flags that can only be set by Aides or the system administrator.  SETU
will ignore attempts to toggle these bits.  There also may be more user
settable bits added at a later date.  To maintain later downward compatibility,
the following procedure is suggested:

 1. Execute GETU to read the current flags
 2. Toggle the bits that we know we can toggle
 3. Execute SETU to write the flags

 If we are passed a bit whose meaning we don't know, it's best to leave it
alone, and pass it right back to the server.  That way we can use an old
client on a server that uses an unknown bit without accidentally clearing
it every time we set the user's configuration.



\subsection{STEL (enter STEaLth mode)}

 When in "stealth mode," a user will not show up in the "Who is online"
listing (the RWHO server command).  Only Aides may use stealth mode.  The
STEL command accepts one argument: a 1 indicating that the user wishes to
enter stealth mode, or a 0 indicating that the user wishes to exit stealth
mode.  STEL returns OK if the command succeeded, ERROR+NOT_LOGGED_IN if no
user is logged in, or ERROR+HIGHER_ACCESS_REQUIRED if the user is not an Aide;
followed by a 1 or 0 indicating the new state.

 If any value other than 1 or 0 is sent by the client, the server simply
replies with 1 or 0 to indicate the current state without changing it.

The STEL command also makes it so a user does not show up in the chat room
/who.



\subsection{*UCHG (Username CHanGe)}

 UCHG is an aide-level command which allows an aide to effectively change their
username.  If this value is blank, the user goes into stealth mode (see
STEL).  Posts
will show up as being from the real username in this mode, however.  In
addition, the RWHO listing will include both the spoofed and real usernames.

 The format of an UCHG command is:

 UCHG <name>

 If a UCHG command is successful, the value OK (200) is returned.



\subsection{USER (send USER name)}

 The first step in logging in a user.  This command takes one argument: the
name of the user to be logged in.  If the user exists, a MORE_DATA return
code will be sent, which means the client should execute PASS as the next
command.  If the user does not exist, ERROR + NO_SUCH_USER is returned.



\subsection{VALI (VALIdate user)}

 This command is used to validate users.  Obviously, it can only be executed
by users with Aide level access.  It should be passed two parameters: the
name of the user to validate, and the desired access level

 If the command succeeds, OK is returned.  The user's access level is changed
and the "need validation" bit is cleared.  If the command fails for any
reason, ERROR, ERROR+NO_SUCH_USER, or ERROR+HIGHER_ACCESS_REQUIRED will be
returned.


