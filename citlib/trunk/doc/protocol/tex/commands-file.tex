\section{File Commands}



\subsection{CLOS (CLOSe the download file)}

 This command is used to close the download file.  It returns OK if the
file was successfully closed, or ERROR if there wasn't any file open in the
first place.



\subsection{*DELF (DELete a File)}

 This command deletes a file from the room's directory, if there is one.  The
name of the file to delete is the only parameter to be supplied.  Wildcards
are not acceptable, and any slashes in the filename will be converted to
underscores, to prevent unauthorized access to neighboring directories.  The
possible return codes are:

 OK                            -  Command succeeded.  The file was deleted.
 ERROR+NOT_LOGGED_IN           -  Not logged in.
 ERROR+HIGHER_ACCESS_REQUIRED  -  Not an Aide or Room Aide.
 ERROR+NOT_HERE                -  There is no directory in this room.
 ERROR+FILE_NOT_FOUND          -  Requested file was not found.



\subsection{MOVF (MOVe a File)}

 This command is similar to DELF, except that it moves a file (and its
associated file description) to another room.  It should be passed two
parameters: the name of the file to move, and the name of the room to move
the file to.  All of the same return codes as DELF may be returned, and also
one additional one: ERROR+NO_SUCH_ROOM, which means that the target room
does not exist.  ERROR+NOT_HERE could also mean that the target room does
not have a directory.



\subsection{NDOP (Network Download OPen file)}

 Open a network spool file for downloading.  The client must have already
identified itself as a network session using the NETP command.  If the command
returns OK, the client may begin receiving IGnet/Open spool data using
a series of READ commands.  When a CLOS command is issued, the spooled data
is deleted from the server and may not be read again.  If the client has not
authenticated itself with a NETP command, ERROR+HIGHER_ACCESS_REQUIRED will
be returned.



\subsection{NETF (NETwork send a File)}

 This command is similar to MOVF, except that it attempts to send a file over
the network to another system.  It should be passed two parameters: the name
of the file to send, and the node name of the system to send it to.  All of
the same return codes as MOVF may be returned, except for ERROR+NO_SUCH_ROOM.
Instead, ERROR+NO_SUCH_SYSTEM may be returned if the name of the target
system is invalid.

 The name of the originating room will be sent along with the file.  Most
implementations will look for a room with the same name at the receiving end
and attempt to place the file there, otherwise it goes into a bit bucket room
for miscellaneous files.  This is, however, beyond the scope of this document;
see elsewhere for more details.



\subsection{NUOP (Network Upload OPen file)}

 Open a network spool file for uploading.  The client must have already
identified itself as a network session using the NETP command.  If the command
returns OK, the client may begin transmitting IGnet/Open spool data using
a series of WRIT commands.  When a UCLS command is issued, the spooled data
is entered into the server if the argument to UCLS is 1 or discarded if the
argument to UCLS is 0.  If the client has not authenticated itself with a
NETP command, ERROR+HIGHER_ACCESS_REQUIRED will be returned.



\subsection{*OIMG (Open an IMaGe file)}

 Open an image (graphics) file for downloading.  Once opened, the file can be
read as if it were a download file.  This implies that an image and a download
cannot be opened at the same time.  OIMG returns the same result codes as OPEN.

 All images will be in GIF (Graphics Interchange Format).  In the case of
Citadel, the server will convert the supplied filename to all lower case,
append the characters ".gif" to the filename, and look for it in the "images"
subdirectory.  As with the MESG command, there are several "well known"
images which are likely to exist on most servers:

 hello        - "Welcome" graphics to be displayed alongside MESG "hello"
 goodbye      - Logoff banner graphics to be displayed alongside MESG "goodbye"
 background   - Background image (usually tiled) for graphical clients

 The following "special" image names are defined in Citadel server version
5.00 and above:

 _userpic_    - Picture of a user (send the username as the second argument)
 _floorpic_   - A graphical floor label (send the floor number as the second
                argument).  Clients which request a floor picture will display
                the picture *instead* of the floor name.
 _roompic_    - A graphic associated with the *current* room.  Clients which
                request a room picture will display the picture in *addition*
                to the room name (i.e. it's used for a room banner, as
                opposed to the floor picture's use in a floor listing).



\subsection{*OPEN (OPEN a file for download)}

 This command is used to open a file for downloading.  Only one download
file may be open at a time.  The only argument to this command is the name
of the file to be opened.  The user should already be in the room where the
file resides.  Possible return codes are:

 ERROR+NOT_LOGGED_IN
 ERROR+NOT_HERE                (no directory in this room)
 ERROR+FILE_NOT_FOUND          (could not open the file)
 ERROR                         (misc errors)
 OK                            (file is open)

 If the file is successfully opened, OK will be returned, along with the
size (in bytes) of the file, the time of last modification (if applicable),
the filename (if known), and the MIME type of the file (if known).



\subsection{READ (READ from the download file)}

 Two arguments are passed to this command.  The first is the starting position
in the download file, and the second is the total number of bytes to be
read.  If the operation can be performed, BINARY_FOLLOWS will be returned,
along with the number of bytes to follow.  Then, immediately following the
newline, will be that many bytes of binary data.  The client *must* read
exactly that number of bytes, otherwise the client and server will get out
of sync.

 If the operation cannot be performed, any of the usual error codes will be
returned.



\subsection{UCLS (CLoSe the Upload file)}

 Close the file opened with UOPN.  An argument of "1" should be passed to
this command to close and save the file; otherwise, the transfer will be
considered aborted and the file will be deleted.  This command returns OK
if the operation succeeded or ERROR if it did not.



\subsection{*UIMG (Upload an IMaGe file)}

 UIMG is complemenary to OIMG; it is used to upload an image to the server.
The first parameter supplied to UIMG should be 0 if the client is only checking
for permission to upload, or 1 if the client is actually attempting to begin
the upload operation.  The second argument is the name of the file to be
transmitted.  In Citadel, the filename is converted to all lower case,
appended with the characters ".gif", and stored in the "images" directory.

 UIMG returns OK if the client has permission to perform the requested upload,
or ERROR+HIGHER_ACCESS_REQUIRED otherwise.  If the client requested to begin
the operation (first parameter set to 1), an upload file is opened, and the
client should begin writing to it with WRIT commands, then close it with a
UCLS command.

 The supplied filename should be one of:

 ->  _userpic_   (Server will attempt to write to the user's online photo)
 ->  Any of the "well known" filenames described in the writeup for the
     OIMG command.



\subsection{*UOPN (OPeN a file for Uploading)}

 This command is similar to OPEN, except that this one is used when the
client wishes to upload a file to the server.  The first argument is the name
of the file to create, and the second argument is a one-line comment
describing the contents of the file.  Only one upload file may be open at a
time.  Possible return codes are:

 ERROR+NOT_LOGGED_IN
 ERROR+NOT_HERE               (no directory in this room)
 ERROR+FILE_NOT_FOUND         (a name must be specified)
 ERROR                        (miscellaneous errors)
 ERROR+ALREADY_EXISTS         (a file with the same name already exists)
 OK

 If OK is returned, the command has succeeded and writes may be performed.



\subsection{WRIT (WRITe to the upload file)}

 If an upload file is open, this command may be used to write to it.  The
argument passed to this command is the number of bytes the client wishes to
transmit.  An ERROR code will be returned if the operation cannot be
performed.

 If the operation can be performed, SEND_BINARY will be returned, followed
by the number of bytes the server is expecting.  The client must then transmit
exactly that number of bytes.  Note that in the current implementation, the
number of bytes the server is expecting will always be the number of bytes
the client requested to transmit, but the client software should never assume
that this will always happen, in case changes are made later.



