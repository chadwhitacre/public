\section{Network Commands}


\subsection{GNET/SNET (Get/Set NETwork configuration for this room)}

 These commands get/set the network configuration for the current room.  Aide
or Room Aide privileges are required, otherwise an ERROR code is returned.
If the command succeeds, LISTING_FOLLOWS or SEND_LISTING is returned.  The
network configuration for a specific room includes neighbor nodes with whom
the room is shared, and mailing list recipients.  The format of the network
configuration is described in the file "netconfigs.txt".



\subsection{NDOP (Network Download OPen file)}

 Open a network spool file for downloading.  The client must have already
identified itself as a network session using the NETP command.  If the command
returns OK, the client may begin receiving IGnet/Open spool data using
a series of READ commands.  When a CLOS command is issued, the spooled data
is deleted from the server and may not be read again.  If the client has not
authenticated itself with a NETP command, ERROR+HIGHER_ACCESS_REQUIRED will
be returned.



\subsection{NETF (NETwork send a File)}

 This command is similar to MOVF, except that it attempts to send a file over
the network to another system.  It should be passed two parameters: the name
of the file to send, and the node name of the system to send it to.  All of
the same return codes as MOVF may be returned, except for ERROR+NO_SUCH_ROOM.
Instead, ERROR+NO_SUCH_SYSTEM may be returned if the name of the target
system is invalid.

 The name of the originating room will be sent along with the file.  Most
implementations will look for a room with the same name at the receiving end
and attempt to place the file there, otherwise it goes into a bit bucket room
for miscellaneous files.  This is, however, beyond the scope of this document;
see elsewhere for more details.



\subsection{NETP (authenticate as network session with connection NET Password)}

 This command is used by client software to identify itself as a transport
session for Citadel site-to-site networking.  It should be called with
two arguments: the node name of the calling system, and the "shared secret"
password for that connection.  If the authentication succeeds, NETP will
return OK, otherwise, it returns ERROR.



\subsection{NSYN (Network SYNchronize room)}

 This command can be used to synchronize the contents of a room on the
network.  It is only usable by Aides.  It accepts one argument: the name of
a network node (which must be a valid one).

 When NSYN is run, the *entire* contents of the current room will be spooled
to the specified node, without regard to whether any of the messages have
already undergone network processing.  It is up to the receiving node to
check for duplicates (the Citadel networker does handle this) and avoid
posting them twice.

 The command returns OK upon success or ERROR if the user is not an Aide.



\subsection{NUOP (Network Upload OPen file)}

 Open a network spool file for uploading.  The client must have already
identified itself as a network session using the NETP command.  If the command
returns OK, the client may begin transmitting IGnet/Open spool data using
a series of WRIT commands.  When a UCLS command is issued, the spooled data
is entered into the server if the argument to UCLS is 1 or discarded if the
argument to UCLS is 0.  If the client has not authenticated itself with a
NETP command, ERROR+HIGHER_ACCESS_REQUIRED will be returned.



