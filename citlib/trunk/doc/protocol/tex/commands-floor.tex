\section{Floor Commands}



\subsection{CFLR (Create a new FLooR)}

 This command is used to create a new floor.  It should be passed two
arguments: the name of the new floor to be created, and a 1 or 0 depending
on whether the client is actually creating a floor or merely checking to
see if it has permission to create the floor.   The user must be logged in
and have Aide privileges to create a floor.

 If the command succeeds, it will return OK followed by the floor number
associated with the new floor.  Otherwise, it will return ERROR (plus perhaps
HIGHER_ACCESS_REQUIRED, ALREADY_EXISTS, or INVALID_FLOOR_OPERATION)
followed by a description of why the command failed.



\subsection{EFLR (Edit a FLooR)}

 Edit the parameters of a floor.  The client may pass one or more parameters
to this command:

 1. The number of the floor to be edited
 2. The desired new name

 More parameters may be added in the future.  Any parameters not passed to
the server will remain unchanged.  A minimal command would be EFLR and a
floor number -- which would do nothing.  EFLR plus the floor number plus a
floor name would change the floor's name.

 If the command succeeds, it will return OK.  Otherwise it will return
ERROR (plus perhaps HIGHER_ACCESS_REQUIRED or INVALID_FLOOR_OPERATION)



\subsection{KFLR (Kill a FLooR)}

 This command is used to delete a floor.  It should be passed two
argument: the *number* of the floor to be deleted, and a 1 or 0 depending
on whether the client is actually deleting the floor or merely checking to
see if it has permission to delete the floor.  The user must be logged in
and have Aide privileges to delete a floor.

 Floors that contain rooms may not be deleted.  If there are rooms on a floor,
they must be either deleted or moved to different floors first.  This implies
that the Main Floor (floor 0) can never be deleted, since Lobby>, Mail>, and
Aide> all reside on the Main Floor and cannot be deleted.

 If the command succeeds, it will return OK.  Otherwise it will return
ERROR (plus perhaps HIGHER_ACCESS_REQUIRED or INVALID_FLOOR_OPERATION)
followed by a description of why the command failed.



\subsection{*LFLR (List all known FLooRs)}

 On systems supporting floors, this command lists all known floors.  The
command accepts no parameters.  It will return ERROR+NOT_LOGGED_IN if no
user is logged in.  Otherwise it returns LISTING_FOLLOWS and a list of
the available floors, each line consisting of three fields:

\begin{tableii}{l|l}{var}{Position}{Description}

\lineii{1}{The floor number associated with the floor}

\lineii{2}{The name of the floor}

\lineii{3}{Reference count (number of rooms on this floor)}

\end{tableii}



