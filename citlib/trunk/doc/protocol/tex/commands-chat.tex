\section{Chat Commands}



\subsection{*CHAT (enter CHAT mode)}

 This command functions differently from every other command in the system.  It
is used to implement multi-user chat.  For this to function, a new transfer
mode, called START_CHAT_MODE, is implemented.  If a client does not support
chat mode, it should never send a CHAT command!

 In chat mode, messages may arrive asynchronously from the server at any
time.  The client may send messages at any time.  This allows the arrival of
messages without the client having to poll for them.  Arriving messages will
be of the form  "user|message", where the "user" portion is, of course, the
name of the user sending the message, and "message" is the message text.

 Chat mode ends when the server says it ends.  The server will signal the end
of chat mode by transmitting "000" on a line by itself.  When the client reads
this line, it must immediately exit from chat mode without sending any
further traffic to the server.  The next transmission sent to the server
will be a regular server command.

 The Citadel server understands the following commands:
 /quit   -   Exit from chat mode (causes the server to do an 000 end)
 /who    -   List users currently in chat
 /whobbs -   List users currently in chat and elsewhere
 /me     -   Do an irc-style action.
 /join   -   Join a new "room" in which all messages are only heard by
             people in that room.
 /msg    -   /msg <user> <msg> will send the msg to <user> only.
 /help   -   Print help information
 NOOP    -   Do nothing (silently)

 Any other non-empty string is treated as message text and will be broadcast
to other users currently in chat.



