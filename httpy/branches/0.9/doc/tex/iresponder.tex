\subsection{Writing a Responder \label{iresponder}}

In \module{httpy}, a \dfn{responder} is simply an object with a
\function{respond} callable. For example, a responder can be a module with a
\function{respond} function, or a class instance with a \method{respond} method.
Responders have two optional data attributes, which, if not present, will be
added by whatever coupler is in use. For responders that are classes, this API
addition takes place prior to instantiation, so the information is available to
the constructor.

\begin{classdesc*}{IResponder}

\begin{funcdesc}{respond}{request} Responds to a single HTTP request.
\var{request} is a \class{Request} object. The \code{raise} statement is used to
end the transaction. If a \class{Response} object is raised, it will be
converted to an HTTP Response message and written out to the wire. All other
\class{Exception}s will result in a \ulink{\code{500 Internal Server
Error}}{http://www.w3.org/Protocols/rfc2616/rfc2616-sec10.html#sec10.5.1}
response.\end{funcdesc}

\begin{funcdesc}{stop}{} This optional method is called when the application
terminates normally. \end{funcdesc}

\begin{datadesc}{root} The location of the responder on the local
filesystem.\end{datadesc}

\begin{datadesc}{uri}The location of the responder on the network.\end{datadesc}

\end{classdesc*}

Responders are validated using \ulink{Zope's interface
machinery}{http://www.zope.org/Products/ZopeInterface}. Your responder may
provide this interface without explicitly declaring so. If a responder does not
have a \function{respond} callable, or if \function{respond} does not accept
exactly one argument, then an \class{Exception} is raised.

Responders must be able to respond to multiple (non-concurrent) requests.


\subsubsection{An Example}

Here is an example showing the general feel of a responder. This example assumes
a \class{logic} module with API for getting and setting data based on a URI path
and a POST body. The suggestion here is that these might return a commonly
formatted data structure, which would then be used to populate a common
template. Notice the authorization check before setting data.

\begin{verbatim}
import auth
import logic
import templating

from httpy import Response

def respond(request):

    if request.method == 'GET':
        result = logic.get_data(request.path)
    elif request.method == 'POST':
        if not auth.check(request):
            raise Response(403)
        result = logic.set_data(request.path, request.raw_body)
    else:
        raise Response(501)

    template = templating.get_template(request.path)
    body = template.render(result)
    raise Response(200, body)

\end{verbatim}


This is a contrived example, basically following \ulink{the popular
Model-View-Controller pattern}{http://c2.com/cgi/wiki?ModelViewController}: the
\module{auth} and \module{logic} modules are the model, the \module{templating}
module provides the view, and the responder is the controller. However, this
pattern is not enforced in any way, and the bottom line is that you've got all
of Python to play with in writing your responders.
