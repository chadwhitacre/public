\section{The \class{mode} Object \label{mode}}

Websites and web applications go through a life-cycle involving development,
debugging, staging, and deployment. It is often desirable to alter behavior
throughout the application based on the current stage in this life-cycle, for
example, to use a different database connection string in deployment than in
development or staging.

\module{httpy} models this common requirement via a \class{mode} singleton,
which has the following members:

\begin{memberdesc}[boolean]{IS_DEVELOPMENT}
\memberline[boolean]{IS_DEBUGGING}
\memberline[boolean]{IS_STAGING}
\memberline[boolean]{IS_DEPLOYMENT}
    These constants are boolean values based on the current mode, and only one
    will be \class{True} at any given time. Abbrevations and alternate casings
    are allowed; e.g., \constant{IS_DEV} and \constant{is_develo} are both
    aliases for \constant{IS_DEVELOPMENT}. \constant{IS_DE} is an
    \class{AttributeError}, however.
\end{memberdesc}

\begin{methoddesc}{__repr__}{}
    Returns the current mode as a lowercase string. This will always be one of
    \code{development}, \code{debugging}, \code{staging}, or \code{deployment}.
\end{methoddesc}

\begin{methoddesc}{__str__}{}
    Alias for \method{__repr__}.
\end{methoddesc}

\begin{memberdesc}{default}
    Contains the default mode as a lowercase string. Out of the box, this is
    \code{development}.
\end{memberdesc}


\module{httpy}'s current mode is determined by the environment variable
\envvar{HTTPY_MODE}. Since the mode of an application instance is generally only
defined at start-up, this API is intended to be read-only. However, \class{mode}
checks the environment on each call or attribute access, so if you must change
the mode on the fly, you can.

Other parts of the \module{httpy} package alter their behavior according to the
current mode. Here is a reference:

\begin{description}

\item[\code{development}]
    Internal server errors generate a traceback in the browser.

    On \UNIX{} systems, the \class{StandAlone} coupler monitors the filesystem
    counterparts of all loaded modules, and restarts itself if the modification
    time of any of these files changes.

\item[\code{debugging}]
    Equivalent to \code{development}. Additionally, non-\class{Response}
    \class{Exception}s trigger post-mortem debugging via
    \ulink{\module{pdb}}{http://docs.python.org/lib/module-pdb.html}.

\item[\code{staging}]
    Equivalent to \code{deployment}.

\item[\code{deployment}]
    The \class{Static} responder supports the \ulink{304 Not
    Modified}{http://www.w3.org/Protocols/rfc2616/rfc2616-sec10.html#sec10.3.5}
    response.

\end{description}
