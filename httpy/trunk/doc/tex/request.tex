\section{\class{Request} Objects \label{request}}

\subsection{Raw API}

Instances of \class{httpy.Request} store the raw HTTP request in the following
attributes:

\begin{datadesc}{raw}
The entire \ulink{HTTP Request message}{http://www.w3.org/Protocols/rfc2616/rfc2616-sec5.html} exactly as it was found on the wire.
\end{datadesc}

\begin{datadesc}{raw_line}
The raw
\ulink{Request-Line}{http://www.w3.org/Protocols/rfc2616/rfc2616-sec5.html#sec5.1},
 not including the trailing line break.
\end{datadesc}

\begin{datadesc}{raw_headers}
The raw message headers, not including the trailing line breaks.
\end{datadesc}

\begin{datadesc}{raw_body}
The raw message body.
\end{datadesc}



\subsection{Derived API}

Furthermore, \class{Request} instances provide a very minimal derivative API:

\begin{datadesc}{method}
The \ulink{HTTP method}{http://www.w3.org/Protocols/rfc2616/rfc2616-sec9.html} from the Request-Line.
\end{datadesc}

\begin{datadesc}{uri}
The \ulink{Request-URI}{http://www.w3.org/Protocols/rfc2616/rfc2616-sec5.html#sec5.1.2} as a dictionary. The keys of this dictionary are taken from the
names used in the standard library's
\ulink{\module{urlparse}}{http://docs.python.org/lib/module-urlparse.html},
namely: \emph{scheme://netloc/path;parameters?query\#fragment}.
\end{datadesc}

\begin{datadesc}{path}
The \emph{path} component of the Request-URI.
\end{datadesc}

\begin{datadesc}{headers}
the headers as an instance of the standard library's
\ulink{\module{email.Message.Message}}{http://docs.python.org/lib/module-email.Message.html}.
\end{datadesc}


\subsection{Differences under CGI/FastCGI}

Whereas the \class{StandAlone} coupler builds \class{Request} objects directly
from the raw request coming off the wire, the \class{CGI} and \class{FastCGI}
couplers must resort to reconstructing \class{Request}s from the environment and
standard input. The resulting object has the same API as above, but since some
attributes could have subtly different meanings, we provide the following
reference. Where called for, \code{\e r\e n} line breaks are used.

\begin{datadesc}{raw}
The concatenation of raw_line, raw_headers, and raw_body as described below.
\end{datadesc}

\begin{datadesc}{raw_line}
The Request-URI is reconstructed from the \envvar{SCRIPT_NAME},
\envvar{PATH_INFO}, and \envvar{QUERY_STRING} environment variables. This is
combined with the \envvar{REQUEST_METHOD} and \envvar{SERVER_PROTOCOL} variables
to approximate the Request-Line.
\end{datadesc}

\begin{datadesc}{raw_headers}
The headers are reconstructed by taking all environment variables beginning with
"HTTP_" and replacing all underscores with dashes.
\end{datadesc}

\begin{datadesc}{raw_body}
Read from the standard input.
\end{datadesc}

\begin{datadesc}{method}
Equivalent to the \envvar{REQUEST_METHOD} environment variable.
\end{datadesc}

\begin{datadesc}{uri}
The dictionary is based on \var{raw_line} as derived above.
\end{datadesc}

\begin{datadesc}{path}
Equivalent to the \envvar{PATH_INFO} environment variable.
\end{datadesc}

\begin{datadesc}{headers}
The \class{Message} object is constructed from \var{raw_headers} as described
above.
\end{datadesc}


\begin{seealso}
    \seetitle[http://hoohoo.ncsa.uiuc.edu/cgi/interface.html]{The CGI
    Specification}{ The reference for CGI, including use of environment
    variables and standard input.}
\end{seealso}
