% Complete documentation on the extended LaTeX markup used for Python
% documentation is available in ``Documenting Python'', which is part
% of the standard documentation for Python.  It may be found online
% at:
%
%     http://www.python.org/doc/current/doc/doc.html

\documentclass{manual}

\title{httpy}

\author{Chad W. L. Whitacre}

% Please at least include a long-lived email address;
% the rest is at your discretion.
\authoraddress{
	Zeta Design \&\ Development \\
	\url{http://www.zetaweb.com/} \\
	Email: \email{\ulink{chad@zetaweb.com}{mailto:chad@zetaweb.com}}
}

%\date{April 30, 1999}		% update before release!
\date\today
				% Use an explicit date so that reformatting
				% doesn't cause a new date to be used.  Setting
				% the date to \today can be used during draft
				% stages to make it easier to handle versions.

\release{0.5a}			% release version; this is used to define the
				% \version macro

\makeindex			% tell \index to actually write the .idx file
\makemodindex			% If this contains a lot of module sections.


\begin{document}

\maketitle

\begin{abstract}

\noindent
httpy is a sane Python webserver. Simple to configure, and with a
straightforward API, it provides a stable and satisfying HTTP bridge for your
Python web applications.

\end{abstract}

\chapter{Introduction}

httpy exists to bridge your Python application with other HTTP applications. Its
primary use case is as an HTTP origin server for a cluster of Python-based
websites numbering into the hundreds or thousands.

Therefore, httpy shares its design aesthetic with toilet paper: instances of
httpy must be instantly available to solidly perform a single function and then
be cast away without a thought.

With that in mind, here's some httpy zen:

\begin{itemize}
\item
httpy should make simple sites dead simple, and complex sites possible.
\item
Development, deployment, and upgrading should be equally easy, since all will need to happen constantly.
\item
The performance hit of an interpreted server will be offset by easy replication.
\item
Configuration should be kept to a minimum.
\item
Libraries are saner than frameworks.
\end{itemize}

httpy's job is to get HTTP requests into your Python application, and to get HTTP responses from your app back onto the network. Here are some things that are explicitly not httpy's job, along with links to tools that do these jobs well:

\begin{description}

\item[daemonization, complex error logging, uid/gid manipulations]
    {You want \ulink{Dan Bernstein's
    daemontools}{http://cr.yp.to/daemontools.html}. httpy logs everything to the
    standard output, so use
    \ulink{multilog}{http://cr.yp.to/daemontools/multilog.html} to pick up from
    there, and use \ulink{setuidgid}{http://cr.yp.to/daemontools/setuidgid.html}
    to run httpy under a certain account.}

\item[access logging, ssl encryption, virtual hosting, load-balancing]
    {Use an HTTP proxy server such as \ulink{Pound}{http://www.apsis.ch/pound/}.
    You could also do these things with a general-purpose HTTP server such as
    \ulink{Apache}{http://httpd.apache.org/} or
    \ulink{lighttpd}{http://www.lighttpd.net/}.}

\item[caching]
    {Use a caching proxy such as \ulink{Squid}{http://www.squid-cache.org/}, or,
    again, \ulink{Apache}{http://httpd.apache.org/}. Your application should
    also do its own internal caching, of course.}

\item[authentication, authorization, sessioning, storage, templating, etc.]
    {These are your application's responsibility. There are plenty of
    \ulink{Python packages}{http://cheeseshop.python.org/pypi} available to help
    you.}

\end{description}


A future version of httpy is expected to be at least conditionally compliant
with \ulink{HTTP/1.1}{http://www.w3.org/Protocols/rfc2616/rfc2616.html}.
However, the following features are not currently implemented:

\begin{itemize}
\item
Keep-Alive\item
Transfer-Encoding\item
Range requests
\end{itemize}
\chapter{Installation \label{installation}}

Aspen can be installed using either \module{distutils} or \module{setuptools}.
That is, you can either download a tarball, unpack it, and run:

\begin{verbatim}
$ python setup.py install
\end{verbatim}

Or you can run:

\begin{verbatim}
$ easy_install http://aspen.googlecode.com/svn/tags/0.3/
\end{verbatim}

\chapter{Configuration}

httpy exposes the following configuration parameters:


\begin{tableiii}{l|l|l}{var}{Parameter}{Description}{Default}

\lineiii{apps}
    {A colon-separated list of paths that should be considered applications. The
    paths should be given as if they were in URL-space, or from the filesystem's
    perspective, as if they were rooted in the website's filesystem root. So
    they should begin with a slash but should not include the filesystem path to
    the root of the website. The website root is always implicitly considered an
    application.}
    {[no explicit applications]}

\lineiii{ip}
    {The IP address that httpy should listen on. This must either be a valid
    IPv4 address or a null value, in which case httpy will listen on all
    available addresses.}
    {[listen on all addresses]}

\lineiii{mode}
    {Either of the strings deployment or development. This affects various
    aspects of httpy's behavior.}
    {deployment}

\lineiii{port}
    {The TCP port that httpy should bind to. This must be an integer between 0
    and 65535, inclusive.}
    {8080}

\lineiii{root}
    {The path to a directory on the filesystem which will serve as the root of
    the website.}
    {. [i.e., the current working directory]}

\lineiii{verbosity}
    {An integer from 0 to 99, inclusive, indicating how much information to
    log.}
    {1}

\end{tableiii}

The defaults may be overriden in three ways: via command line options, through a
configuration file, and by setting environment variables. This is also the order
in which they take precedence, such that configuration file settings override
environment variables, etc. The subsections below refer to the parameter names
as given in the first column of the above table.




\section{Command Line Options}

httpy uses the Python standard library's optparse module to represent its
command line arguments. The parameter names may be used for long-named
arguments, or their initial letter may be used for short-named arguments. So for
example, httpy could be started on a custom port, with a custom website root,
using the following command:

\begin{verbatim}
% httpy -p9000 --root ~/www
\end{verbatim}

In addition, the path to any configuration file is specified with the
\programopt{-f}/\longprogramopt{file} argument.

\begin{seealso}
\seetitle [http://www.python.org/doc/lib/module-optparse.html] {optparse}
          {httpy uses the standard library's optparse module.}
\end{seealso}




\section{Configuration File Settings}

httpy uses the standard library's RawConfigParser class to represent its
configuration file. This means that the file is in the style of a Windows INI
file, with section headers in brackets, and name=value pairs on successive
lines. Use the names as given above. The number and names of any sections are
irrelevant, but RawConfigParser requires that there be at least one section. As
an example, a configuration file with the following contents would set the
verbosity of an httpy instance:

\begin{verbatim}
[main]
verbosity = 99
\end{verbatim}

\begin{seealso}
\seetitle [http://www.python.org/doc/lib/RawConfigParser-objects.html] {RawConfigParser}
          {httpy uses the standard library's RawConfigParser class.}
\end{seealso}




\section{Environment Variables}

The names of the environment variables recognized by httpy are derived by
uppercasing the above names, and prepending them with HTTPY_. So, for example,
to configure a server so that all instances of httpy on that server are run in
development mode instead of deployment mode, one would simply set the
environment variable \envvar{HTTPY_MODE} to \constant{development}.

\chapter{Customization}

httpy is built around the idea that there are two kinds of websites,
publications and applications, which are differentiated by their organization
and interface models. A \dfn{publication} website organizes information into
individual pages within a hierarchical folder structure that one navigates by
browsing. In an \dfn{application} website, on the other hand, data is not
organized into hierarchical pages but is dealt with via a non-browsing interface
such as a search box.

The HTML version of this documentation is an example of a publication website: a
number of hypertext documents organized into sections. If we weren't using LaTeX
(or if I knew how to use it better), the sections would probably be encoded in
folders. \ulink{Gmail}{http://mail.google.com/mail} is a pure application
website, one which organizes and presents information non-hierarchically. Most
websites, however, are hybrids. That is, within an overall hierarchical
organization you will find both individual pages of information as well as
applications such as a site search feature, or a threaded discussion forum.

Now, in point of fact, publication websites are a subset of application
websites. An application site can use any interface metaphor; a publication is
an application that uses the familiar folder/page metaphor to organize and
present its information. Therefore, every website is fundamentally an
application. httpy's default application implements the basic publication
scenario: serving static files straight off the filesystem.

httpy enables you to build custom publication, application, and hybrid websites.
It privileges the filesystem in supporting a publication or hybrid website's
hierarchical organization. To support applications, httpy looks for a certain
filesystem layout with files that define certain Python objects, which it
incorporates into the HTTP request/response process. These are described below.



\section{Filesystem Layout}

When httpy starts up, it walks the tree rooted in your website's filesystem
root, and it gathers all paths to directories below the root that have a
subdirectory named '__' -- that's two underscores. The directory is said to be
an \dfn{app} or \dfn{application}, and the subdirectory is that application's
\dfn{magic directory}. \footnote{You can override this default behavior. Please
see the section on Configuration.}

For example, consider this filesystem layout:

\begin{verbatim}
/usr/local/www/
/usr/local/www/about/
/usr/local/www/news/
/usr/local/www/news/__
\end{verbatim}

Assuming that \file{/usr/local/www/} is configured as your website root, httpy
will by default find one app: /news. This app's magic directory will be
\file{/usr/local/www/__}.

On each request, httpy looks for an application that matches the beginning of
the requested URI path. So in the above example, requests for \file{/news},
\file{/news/archive}, \file{/news/post}, etc., would be handled by the
\code{/news} app. All other requests would be handled by the default httpy
application, because there is no app configured for the site root.

If you manually configure the \var{apps} setting, and a given app does not have
a magic directory, httpy will raise \class{ConfigError} and exits.



\section{Expected Python Objects}

When httpy starts up, it attempts to import a module or package named
\module{app} from each application's magic directory. Each \module{app}
module/package must define a class named \class{Application}.
\class{Application}'s constructor must take a single argument, an instance of
\class{AppConfig}. \class{Application} must have at least one other method,
named \code{process}, which also takes a single argument, an instance of
\class{Request}. If any of these conditions are not met, httpy raises
\class{ConfigError} and exits. Therefore, here is the minimal legal
\file{app.py}:

\begin{verbatim}
class Application:

    def __init__(self, config):
        pass

    def process(self, request):
        pass

\end{verbatim}

\class{Application.process} is expected to end the request by raising a
\class{Response} or other \class{Exception}. Therefore, here is the minimal
successful \file{app.py}:

\begin{verbatim}
from httpy.Response import Response

class Application:

    def __init__(self, config):
        pass

    def process(self, request):
        response = Response(200)
        response.headers['content-type'] = 'text/plain'
        response.body = "Greetings, program!"
        raise response

\end{verbatim}



\section{Site-Specific Libraries}

If a website root has a magic directory, httpy will look when it starts up for a
subdirectory of this magic directory named \file{site-packages}. It will prepend
this path to \class{sys.path}, with the effect that each website can maintain a
local Python library in \file{<site-root>/__/site-packages}. This library is
global to the applications within a website, however, and only the root's magic
directory is searched for \file{site-packages}.



\chapter{Package Contents/API \label{api}}

The \module{httpy} package defines classes to represent HTTP Request and
Response messages. It also defines interfaces specifying how to build responders
and couplers, specimens of which are to be found in the \module{responders} and
\module{couplers} subpackages. Finally, the \class{mode} singleton provides an
object-oriented API for the \envvar{HTTPY_MODE} environment variable, and the
\module{utils} subpackage collects some other possibly useful tools.


\begin{classdesc}{Request}{IRequest}
Constructs a new \class{Request} object. \var{IRequest} is a (probably implicit)
provider of the \class{IRequest} interface. This provision is validated, and the
desired API is transferred from the \class{IRequest} provider to the new object
instance.

Your responder's \method{respond} method will be given instances of this class,
so you will be using it constantly on that basis. However, you would probably
only need to instantiate it directly if you were writing a new coupler.
\end{classdesc}


\begin{classdesc}{Response}{\optional{code} \optional{, body} \optional{,
    headers}}
Constructs a new \class{Response} object. If given, \var{code} must be
an integer; the default is
\ulink{200}{http://www.w3.org/Protocols/rfc2616/rfc2616-sec10.html#sec10.2.1}
(see \citetitle[http://www.w3.org/Protocols/rfc2616/rfc2616-sec10.html]{the HTTP
spec} for other values that will be meaningful to most HTTP clients). \var{body}
may be a string or an iterator over strings. \var{headers} must be a dictionary.

\var{body} is second rather than \var{headers} because one more often wants to
specify a body without headers than vice versa. Also note that
\mailheader{Content-Type} defaults to \code{text/html} for responses where
\var{code} is between 200 and 299, inclusive, but to \code{text/plain} for
non-2xx responses.

This class is likewise central to \module{httpy} programming: its instances are
the payload for \module{httpy}'s overloaded \code{raise} statement. These will
be caught, validated, flattened, and sent out to the wire by whatever coupler is
in use.
\end{classdesc}


\section{\class{Request} Objects \label{request}}

HTTP is a stateless protocol. Any state that exists is encoded somewhere in the
Request: either in the Request-Line, the headers, or the body. httpy doesn't
pretend to know the best way for you to encode state in your app's requests.
Instead, it gives you the raw HTTP message, and a very minimal secondary API.
There are a few utilities to help you with this -- parse_body, parse_cookie,
parse_query. But you are free to parse the Request however you like.


\subsection{Raw API}

Instances of \class{httpy.Request} store the raw HTTP request in the following
attributes:

\begin{datadesc}{raw}
The entire \ulink{HTTP Request message}{http://www.w3.org/Protocols/rfc2616/rfc2616-sec5.html} exactly as it was found on the wire.
\end{datadesc}

\begin{datadesc}{raw_line}
The raw
\ulink{Request-Line}{http://www.w3.org/Protocols/rfc2616/rfc2616-sec5.html#sec5.1},
 not including the trailing line break.
\end{datadesc}

\begin{datadesc}{raw_headers}
The raw message headers, not including the trailing line breaks.
\end{datadesc}

\begin{datadesc}{raw_body}
The raw message body.
\end{datadesc}



\subsection{Derived API}

Besides providing access to the raw Request message, \class{Request} instances
provide a very minimal derivative API:

\begin{datadesc}{method}
The \ulink{HTTP method}{http://www.w3.org/Protocols/rfc2616/rfc2616-sec9.html} from the Request-Line.
\end{datadesc}

\begin{datadesc}{uri}
The \ulink{Request-URI}{http://www.w3.org/Protocols/rfc2616/rfc2616-sec5.html#sec5.1.2} as a dictionary. The keys of this dictionary are taken from the
names used in the standard library's
\ulink{\module{urlparse}}{http://docs.python.org/lib/module-urlparse.html},
namely: \emph{scheme://netloc/path;parameters?query\#fragment}.
\end{datadesc}

\begin{datadesc}{path}
The \emph{path} component of the Request-URI.
\end{datadesc}

\begin{datadesc}{headers}
the headers as an instance of the standard library's
\ulink{\module{email.Message.Message}}{http://docs.python.org/lib/module-email.Message.html}.
\end{datadesc}


\subsection{Differences under CGI/FastCGI}

Whereas the \class{StandAlone} coupler builds \class{Request} objects directly
from the raw request coming off the wire, the \class{CGI} and \class{FastCGI}
couplers must resort to reconstructing \class{Request}s from the environment and
standard input. The resulting object has the same API as above, but since some
attributes could have subtly different meanings, we provide the following
reference. Where called for, \code{\e r\e n} line breaks are used.

\begin{datadesc}{raw}
The concatenation of raw_line, raw_headers, and raw_body as described below.
\end{datadesc}

\begin{datadesc}{raw_line}
The Request-URI is reconstructed from the \envvar{SCRIPT_NAME},
\envvar{PATH_INFO}, and \envvar{QUERY_STRING} environment variables. This is
combined with the \envvar{REQUEST_METHOD} and \envvar{SERVER_PROTOCOL} variables
to approximate the Request-Line.
\end{datadesc}

\begin{datadesc}{raw_headers}
The headers are reconstructed by taking all environment variables beginning with
"HTTP_" and replacing all underscores with dashes.
\end{datadesc}

\begin{datadesc}{raw_body}
Read from the standard input.
\end{datadesc}

\begin{datadesc}{method}
Equivalent to the \envvar{REQUEST_METHOD} environment variable.
\end{datadesc}

\begin{datadesc}{uri}
The dictionary is based on \var{raw_line} as derived above.
\end{datadesc}

\begin{datadesc}{path}
Equivalent to the \envvar{PATH_INFO} environment variable.
\end{datadesc}

\begin{datadesc}{headers}
The \class{Message} object is constructed from \var{raw_headers} as described
above.
\end{datadesc}


\begin{seealso}
    \seetitle[http://hoohoo.ncsa.uiuc.edu/cgi/interface.html]{The CGI
    Specification}{ The reference for CGI, including use of environment
    variables and standard input.}
\end{seealso}

\section{\class{Response} Objects \label{response}}

Instances of \class{httpy.Response} have the following attributes. Note that
values are only validated in the constructor, so it is currently possible to
raise a malformed \class{Response} by setting instance attributes
post-instantiation.

\begin{datadesc}{code}
The \ulink{HTTP code}{http://www.w3.org/Protocols/rfc2616/rfc2616-sec10.html} as an integer.
\end{datadesc}

\begin{datadesc}{body}
The message body as a string, an iterator over strings, or an
\class{httpy.utils.Path} object.
\end{datadesc}

\begin{datadesc}{headers}
The message headers as a dictionary.
\end{datadesc}

\section{Responders \label{responders}}

There are basically two kinds of websites, publications and applications,
differentiated by their organization and interface models. A \dfn{publication}
website organizes information into individual pages within a hierarchical folder
structure that one navigates by browsing. In an \dfn{application} website, on
the other hand, data is not organized into hierarchical pages but is dealt with
via a non-browsing interface such as a search box.

The HTML version of this documentation is an example of a publication website: a
number of hypertext documents organized into sections. If we weren't using LaTeX
(or if I knew how to use it better), the sections would probably be encoded in
folders. \ulink{Gmail}{http://mail.google.com/mail} is a pure application
website, one which organizes and presents information non-hierarchically. Most
websites, however, are hybrids. That is, within an overall hierarchical
organization you will find both individual pages of information as well as
applications such as a site search feature, or a threaded discussion forum.

Publication websites are actually a subset of application websites, of course.
An application site can use any interface metaphor; a publication is an
application that uses the familiar folder/page metaphor to organize and present
its information. Therefore, every website is fundamentally an application.

\module{httpy} enables you to build custom hybrid, publication, and application
websites, and its bundled responders include an implementation of each.



\begin{classdesc}{Multiple}{\optional{root}} Constructs a new \class{Multiple}
responder object. \var{root} is the filesystem path of the directory to use as
the publishing root. If not given or \class{None}, then it is set to the current
working directory. If \var{root} does not point to a directory, \exception
{ValueError} is raised. This class is defined in the
\module{httpy.responders.multiple} module.\end{classdesc}

\begin{classdesc}{Static}{\optional{root}\optional{,
defaults}} Constructs a new \class{Static} responder object. \var{root} has the
same meaning as in \class{Multiple}. \var{defaults} is a sequence of names to
use when looking for a default resource. If it is omitted or \class{None}, it is
set to \code{('index.html', 'index.htm')}, after checking the class attribute as
with \var{root}. This class is defined in the \module{httpy.responders.static}
module.\end{classdesc}

\begin{classdesc*}{XMLRPC} The XMLRPC responder is not
designed to be instantiated directly, but rather used as a mixin. This class is
defined in the \module{httpy.responders.xmlrpc} module. \end{classdesc*}


\chapter{the \class{Multiple} responder \label{multiple}}

httpy is built around the idea that there are two kinds of websites,
publications and applications, which are differentiated by their organization
and interface models. A \dfn{publication} website organizes information into
individual pages within a hierarchical folder structure that one navigates by
browsing. In an \dfn{application} website, on the other hand, data is not
organized into hierarchical pages but is dealt with via a non-browsing interface
such as a search box.

The HTML version of this documentation is an example of a publication website: a
number of hypertext documents organized into sections. If we weren't using LaTeX
(or if I knew how to use it better), the sections would probably be encoded in
folders. \ulink{Gmail}{http://mail.google.com/mail} is a pure application
website, one which organizes and presents information non-hierarchically. Most
websites, however, are hybrids. That is, within an overall hierarchical
organization you will find both individual pages of information as well as
applications such as a site search feature, or a threaded discussion forum.

Now, in point of fact, publication websites are a subset of application
websites. An application site can use any interface metaphor; a publication is
an application that uses the familiar folder/page metaphor to organize and
present its information. Therefore, every website is fundamentally an
application. httpy's default application implements the basic publication
scenario: serving static files straight off the filesystem.

httpy enables you to build custom publication, application, and hybrid websites.
It privileges the filesystem in supporting a publication or hybrid website's
hierarchical organization. To support applications, httpy looks for a certain
filesystem layout with files that define certain Python objects, which it
incorporates into the HTTP request/response process. These are described below.



\section{Filesystem Layout}

When httpy starts up, it walks the tree rooted in your website's filesystem
root, and it gathers all paths to directories below the root that have a
subdirectory named '__' -- that's two underscores. The directory is said to be
an \dfn{app} or \dfn{application}, and the subdirectory is that application's
\dfn{magic directory}. \footnote{You can override this default behavior. Please
see the section on Configuration.}

For example, consider this filesystem layout:

\begin{verbatim}
/usr/local/www/
/usr/local/www/about/
/usr/local/www/news/
/usr/local/www/news/__
\end{verbatim}

Assuming that \file{/usr/local/www/} is configured as your website root, httpy
will by default find one app: /news. This app's magic directory will be
\file{/usr/local/www/__}.

On each request, httpy looks for an application that matches the beginning of
the requested URI path. So in the above example, requests for \file{/news},
\file{/news/archive}, \file{/news/post}, etc., would be handled by the
\code{/news} app. All other requests would be handled by the default httpy
application, because there is no app configured for the site root.

If you manually configure the \var{apps} setting, and a given app does not have
a magic directory, httpy will raise \class{ConfigError} and exits.



\section{Expected Python Objects}

When httpy starts up, it attempts to import a module or package named
\module{app} from each application's magic directory. Each \module{app}
module/package must define a class named \class{Application}.
\class{Application}'s constructor must take a single argument, an instance of
\class{AppConfig}. \class{Application} must have at least one other method,
named \code{respond}, which also takes a single argument, an instance of
\class{Request}. If any of these conditions are not met, httpy raises
\class{ConfigError} and exits. Therefore, here is the minimal legal
\file{app.py}:

\begin{verbatim}
class Application:

    def __init__(self, config):
        pass

    def respond(self, request):
        pass

\end{verbatim}

\class{Application.respond} is expected to end the request by raising a
\class{Response} or other \class{Exception}. Therefore, here is the minimal
successful \file{app.py}:

\begin{verbatim}
from httpy.Response import Response

class Application:

    def __init__(self, config):
        pass

    def respond(self, request):
        response = Response(200)
        response.headers['content-type'] = 'text/plain'
        response.body = "Greetings, program!"
        raise response

\end{verbatim}



\section{Site-Specific Libraries}

If a website root has a magic directory, httpy will look when it starts up for a
subdirectory of this magic directory named \file{site-packages}. It will prepend
this path to \class{sys.path}, with the effect that each website can maintain a
local Python library in \file{<site-root>/__/site-packages}. This library is
global to the applications within a website, however, and only the root's magic
directory is searched for \file{site-packages}.



\subsection{\class{Static} Responders \label{static}}

The \class{Static} responder implements the basic publication application:
serving public files straight off the filesystem. This responder is designed to
be used either by itself or as a mixin. It provides one attributes and two
methods:

\begin{memberdesc}[string]{root}
The filesystem path of the directory to use as the publishing root.
\end{memberdesc}

\begin{memberdesc}[tuple]{defaults}
A tuple of names to interpret as default resources. The first-named is chosen
first.
\end{memberdesc}


\begin{methoddesc}{respond}{request}
This is a pass-through for \method{serve_static}, and can safely be overriden.
\end{methoddesc}

\begin{methoddesc}{serve_static}{request} Serves a static resource from the
filesystem. \var{request} is a \class{Request} object. The URI is translated
using the \function{httpy.utils.translate} function, and the
\mailheader{Content-Type} is set by the standard library's
\ulink{\module{mimetypes}}{http://docs.python.org/lib/module-mimetypes.html} module. In
\code{staging} and \code{deployment} modes, \method{serve_static} supports the
\ulink{\code{304 Not
Modified}}{http://www.w3.org/Protocols/rfc2616/rfc2616-sec10.html#sec10.3.5}
response.
\end{methoddesc}


\subsection{\class{XMLRPC} Responders \label{xmlrpc}}

\ulink{XMLRPC}{http://www.xmlrpc.com/} is a language-neutral protocol for
distributed computing, using XML for serialization and HTTP for transport. The
Python standard library includes \ulink{a well-written XMLRPC
library}{http://docs.python.org/lib/module-xmlrpclib.html}, which forms the
basis for this class. By mixing the \class{XMLRPC} responder into your class,
you can easily make your instance's methods available to clients over the
network. Instances inherit one attribute and two methods:

\begin{memberdesc}[tuple]{protected} A tuple of names of methods on \class{self}
which should never be served via XMLRPC. This class attribute on \class{XMLRPC}
is set to the empty tuple.\end{memberdesc}

\begin{methoddesc}{respond}{request}
This is a pass-through for \method{serve_xmlrpc}. It can safely be overriden, but
in most cases need not be.
\end{methoddesc}

\begin{methoddesc}{serve_xmlrpc}{request}
Proxies methods on \class{self} via XMLRPC. \var{request} is a \class{Request}
object. Assuming \var{request} is an XMLRPC request for method \var{name}, the
following conditions trigger an XMLRPC <fault> response with <faultCode> 404:

\begin{enumerate}

\item \var{name} starts with an underscore.

\item \var{name} is \code{respond} or \code{serve_xmlrpc}.

\item \var{name} is named in \code{protected}.

\item The instance does not have a method named \var{name}.

\end{enumerate}
\end{methoddesc}


\subsubsection{An Example}

In a Python shell, create an XMLRPC server like so:

\begin{verbatim}
>>> import httpy
>>> class Responder(httpy.responders.XMLRPC):
...   protected = ['private']
...   def private(self):
...     return 'leave me alone!'
...   def ping(self):
...     return 'pong'
...
>>> responder = Responder()
>>> coupler = httpy.couplers.StandAlone(responder)
>>> coupler.go()
httpy.server     INFO     httpy started on port 8080
\end{verbatim}

Then, in a second shell, you can talk to your server like this:

\begin{verbatim}
>>> import xmlrpclib
>>> server = xmlrpclib.ServerProxy('http://localhost:8080/')
>>> server.ping()
'pong'
>>> server.private()
Traceback (most recent call last):
...
xmlrpclib.Fault: <Fault 404: "method 'private' not found">
>>>
\end{verbatim}

\subsection{Writing a Responder \label{iresponder}}

In \module{httpy}, a \dfn{responder} is simply an object with a
\function{respond} callable. For example, a responder can be a module with a
\function{respond} function, or a class instance with a \method{respond} method.
Responders have two optional data attributes, which, if not present, will be
added by whatever coupler is in use. For responders that are classes, this API
addition takes place prior to instantiation, so the information is available to
the constructor.

\begin{classdesc*}{IResponder}

\begin{funcdesc}{respond}{request} Responds to a single HTTP request.
\var{request} is a \class{Request} object. The transaction is ended by either
returning or raising a \class{Response} object, which is converted to an HTTP
Response message and written out to the wire. All other \class{Exception}s
result in a \ulink{\code{500 Internal Server
Error}}{http://www.w3.org/Protocols/rfc2616/rfc2616-sec10.html#sec10.5.1}
response.\end{funcdesc}

\begin{funcdesc}{stop}{} This optional method is called when the application
terminates normally. \end{funcdesc}

\begin{datadesc}{root} The location of the responder on the local
filesystem.\end{datadesc}

\begin{datadesc}{uri}The location of the responder on the network.\end{datadesc}

\end{classdesc*}

Responders are validated using \ulink{Zope's interface
machinery}{http://www.zope.org/Products/ZopeInterface}. Your responder may
provide this interface without explicitly declaring so. If a responder does not
have a \function{respond} callable, or if \function{respond} does not accept
exactly one argument, then an \class{Exception} is raised.

Responders must be able to respond to multiple (non-concurrent) requests.


\subsubsection{An Example}

Here is an example showing the general feel of a responder. This example assumes
a \class{logic} module with API for getting and setting data based on a URI path
and a POST body. The suggestion here is that these might return a commonly
formatted data structure, which would then be used to populate a common
template. Notice the authorization check before setting data.

\begin{verbatim}
import auth
import logic
import templating

from httpy import Response

def respond(request):

    if request.method == 'GET':
        result = logic.get_data(request.path)
    elif request.method == 'POST':
        if not auth.check(request):
            raise Response(403)
        result = logic.set_data(request.path, request.raw_body)
    else:
        raise Response(501)

    template = templating.get_template(request.path)
    body = template.render(result)
    return Response(200, body)

\end{verbatim}


This is a contrived example, basically following \ulink{the popular
Model-View-Controller pattern}{http://c2.com/cgi/wiki?ModelViewController}: the
\module{auth} and \module{logic} modules are the model, the \module{templating}
module provides the view, and the responder is the controller. However, this
pattern is not enforced in any way, and the bottom line is that you've got all
of Python to play with in writing your responders.


\section{Couplers \label{couplers}}

The action in \module{httpy} is in the responders, but the couplers are the
workhorses that put responders on the network. \module{httpy} includes three
couplers out of the box. There are multiple Python webservers available, and a
couple Python FastCGI libraries too. I've tried to pick the single most mature
option in each category, and build the couplers on top of those. These decisions
are purely pragmatic.

\begin{classdesc}{CGI}{responder} Constructs a new \class{CGI} coupler object.
This class is defined in the \module{httpy.couplers.cgi} module.
\end{classdesc}

\begin{classdesc}{FastCGI}{responder} Constructs a new \class{FastCGI} coupler
object. This class is defined in the \module{httpy.couplers.fastcgi} module.
\end{classdesc}

\begin{classdesc}{StandAlone}{responder\optional{, argv}} Constructs a new
\class{StandAlone} coupler object. \var{argv} is a sequence of arguments. If
omitted or \class{None}, \code{sys.argv} is used. See \ulink{the chapter on the
\program{httpy} executable}{manual.html} for the available options. This class
is defined in the \module{httpy.couplers.standalone} module. \end{classdesc}

Each coupler takes a (possibly implicit) provider of the \class{IResponder}
interface as its first argument, and each provides the following method:

\begin{methoddesc}{go}{} For \class{CGI} instances, this responds to a single
request. For \class{FastCGI} and \class{StandAlone}, this enters a blocking
loop.\end{methoddesc}

As an example, here is what a CGI script looks like with \module{httpy}:

\begin{verbatim}
#!/usr/local/bin/python
"""This is a CGI script.
"""
from httpy.couplers.cgi import CGI
import myresponder

coupled = CGI(myresponder)
coupled.go()
\end{verbatim}

Those wishing to implement new couplers are invited to consult the source code,
and to contact the author.
\section{The \class{mode} Object \label{mode}}

Websites and web applications go through a life-cycle involving development,
debugging, staging, and deployment. It is often desirable to alter behavior
throughout the application based on the current stage in this life-cycle, for
example, to use a different database connection string in deployment than in
development or staging.

\module{httpy} models this common requirement via a \class{mode} singleton,
which has the following members:

\begin{memberdesc}[boolean]{IS_DEVELOPMENT}
\memberline[boolean]{IS_DEBUGGING}
\memberline[boolean]{IS_STAGING}
\memberline[boolean]{IS_DEPLOYMENT}
    These constants are boolean values based on the current mode, and only one
    will be \class{True} at any given time. Abbrevations and alternate casings
    are allowed; e.g., \constant{IS_DEV} and \constant{is_develo} are both
    aliases for \constant{IS_DEVELOPMENT}. \constant{IS_DE} is an
    \class{AttributeError}, however.
\end{memberdesc}

\begin{methoddesc}{__repr__}{}
    Returns the current mode as a lowercase string. This will always be one of
    \code{development}, \code{debugging}, \code{staging}, or \code{deployment}.
\end{methoddesc}

\begin{methoddesc}{__str__}{}
    Alias for \method{__repr__}.
\end{methoddesc}

\begin{memberdesc}{default}
    Contains the default mode as a lowercase string. Out of the box, this is
    \code{development}.
\end{memberdesc}


\module{httpy}'s current mode is determined by the environment variable
\envvar{HTTPY_MODE}. Since the mode of an application instance is generally only
defined at start-up, this API is intended to be read-only. However, \class{mode}
checks the environment on each call or attribute access, so if you must change
the mode on the fly, you can.

Other parts of the \module{httpy} package alter their behavior according to the
current mode. Here is a reference:

\begin{description}

\item[\code{development}]
    Internal server errors generate a traceback in the browser.

    On \UNIX{} systems, the \class{StandAlone} coupler monitors the filesystem
    counterparts of all loaded modules, and restarts itself if the modification
    time of any of these files changes.

\item[\code{debugging}]
    Equivalent to \code{development}. Additionally, non-\class{Response}
    \class{Exception}s trigger post-mortem debugging via
    \ulink{\module{pdb}}{http://docs.python.org/lib/module-pdb.html}.

\item[\code{staging}]
    Equivalent to \code{deployment}.

\item[\code{deployment}]
    The \class{Static} responder supports \ulink{304}{http://www.w3.org/Protocols/rfc2616/rfc2616-sec10.html#sec10.3.5}s.

\end{description}

\section{Utilities \label{utilities}}

The \module{httpy.utilities} subpackage collects several tools which may be of
value to those building web applications with \module{httpy}. The \class{Path}
and \function{translate} objects are actually used by \module{httpy}'s own
\class{Static} responder. The \function{parse_*} functions are not used in the
base module, but all but the simplest web applications will need these or
parallel methods to extract information from the \class{Request} object.


\begin{classdesc}{Path}{path}
Constructs a new \class{Path} object. \var{path} is a path to a file on the
filesystem. When a \class{Response}.\code{body} is a \class{Path}, the
third-party
\ulink{\class{sendfile}}{http://tautology.org/software/python-modules/sendfile}
module will be used if available to write the file out to the wire.
\end{classdesc}



\begin{funcdesc}{parse_body}{request}
Given a \class{Request} instance, returns an instance of the standard library's
\ulink{\module{cgi.FieldStorage}}{http://docs.python.org/lib/node471.html} class
representing the request body.
\end{funcdesc}

\begin{funcdesc}{parse_cookie}{request}
Given a \class{Request} instance, returns an instance of the standard library's
\ulink{\module{Cookie.SimpleCookie}}{http://docs.python.org/lib/module-Cookie.html} class
representing the request's cookie.
\end{funcdesc}

\begin{funcdesc}{parse_query}{request}
Given a \class{Request} instance, returns an instance of the standard library's
\ulink{\module{cgi.FieldStorage}}{http://docs.python.org/lib/node471.html} class
representing the request's querystring.
\end{funcdesc}



\begin{funcdesc}{translate}{uri_path, fs_root\optional{, defaults}\optional{, raw}}
Translates a URI path to the filesystem.

\var{uri_path} is the path component of a Request-URI (i.e.,
\class{Request}.\code{path}). \var{fs_root} is the filesystem path of the
directory in which the URI path should be rooted. \var{defaults}, if given, is a
sequence of names that should be considered default resources. If not given,
\var{defaults} is empty. If \var{raw} is given and it evaluates to \class{True},
then \code{translate()} ignores \var{defaults} and performs no validation.

If \var{raw} is \class{False}, then validation proceeds according to this
rubric:

\begin{enumerate}

\item If the translated path points to a directory, then the URI must end with a
slash, or
\ulink{301}{http://www.w3.org/Protocols/rfc2616/rfc2616-sec10.html#sec10.3.2} is
raised.

\item If the translated path points to a directory, and no default resource is
named or available, then
\ulink{403}{http://www.w3.org/Protocols/rfc2616/rfc2616-sec10.html#sec10.4.4} is
raised.

\item If the translated path does not point to a directory, then it must point
to a valid file, or
\ulink{404}{http://www.w3.org/Protocols/rfc2616/rfc2616-sec10.html#sec10.4.5} is
raised.

\end{enumerate}

\code{translate()} returns the filesystem path of the requested resource.
\end{funcdesc}


\chapter{Examples}

\section{basic site}

\section{root app}

\section{non-root app}

\section{custom contig}







%
%  The ugly "%begin{latexonly}" pseudo-environments are really just to
%  keep LaTeX2HTML quiet during the \renewcommand{} macros; they're
%  not really valuable.
%
%  If you don't want the Module Index, you can remove all of this up
%  until the second \input line.
%
%begin{latexonly}
\renewcommand{\indexname}{Module Index}
%end{latexonly}
\input{mod\jobname.ind}		% Module Index

%begin{latexonly}
\renewcommand{\indexname}{Index}
%end{latexonly}
\input{\jobname.ind}			% Index

\end{document}
