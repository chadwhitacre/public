% Complete documentation on the extended LaTeX markup used for Python
% documentation is available in ``Documenting Python'', which is part
% of the standard documentation for Python.  It may be found online
% at:
%
%     http://www.python.org/doc/current/doc/doc.html

\documentclass{manual}

\title{httpy}

\author{Chad W. L. Whitacre}

% Please at least include a long-lived email address;
% the rest is at your discretion.
\authoraddress{
	Zeta Design \&\ Development \\
	\url{http://www.zetaweb.com/} \\
	Email: \email{\ulink{chad@zetaweb.com}{mailto:chad@zetaweb.com}}
}

%\date{April 30, 1999}		% update before release!
\date\today
				% Use an explicit date so that reformatting
				% doesn't cause a new date to be used.  Setting
				% the date to \today can be used during draft
				% stages to make it easier to handle versions.

\release{0.5a}			% release version; this is used to define the
				% \version macro

\makeindex			% tell \index to actually write the .idx file
\makemodindex			% If this contains a lot of module sections.


\begin{document}

\maketitle

\begin{abstract}

\noindent
httpy is a sane Python webserver. Simple to configure, and with a
straightforward API, it provides a stable and satisfying HTTP bridge for your
Python web applications.

\end{abstract}

\chapter{Introduction}

httpy exists to bridge your Python application with other HTTP applications. Its
primary use case is as an HTTP origin server for a cluster of Python-based
websites numbering into the hundreds or thousands.

Therefore, httpy shares its design aesthetic with toilet paper: instances of
httpy must be instantly available to solidly perform a single function and then
be cast away without a thought.

With that in mind, here's some httpy zen:

\begin{itemize}
\item
httpy should make simple sites dead simple, and complex sites possible.
\item
Development, deployment, and upgrading should be equally easy, since all will need to happen constantly.
\item
The performance hit of an interpreted server will be offset by easy replication.
\item
Configuration should be kept to a minimum.
\item
Libraries are saner than frameworks.
\end{itemize}

httpy's job is to get HTTP requests into your Python application, and to get HTTP responses from your app back onto the network. Here are some things that are explicitly not httpy's job, along with links to tools that do these jobs well:

\begin{description}

\item[daemonization, complex error logging, uid/gid manipulations]
    {You want \ulink{Dan Bernstein's
    daemontools}{http://cr.yp.to/daemontools.html}. httpy logs everything to the
    standard output, so use
    \ulink{multilog}{http://cr.yp.to/daemontools/multilog.html} to pick up from
    there, and use \ulink{setuidgid}{http://cr.yp.to/daemontools/setuidgid.html}
    to run httpy under a certain account.}

\item[access logging, ssl encryption, virtual hosting, load-balancing]
    {Use an HTTP proxy server such as \ulink{Pound}{http://www.apsis.ch/pound/}.
    You could also do these things with a general-purpose HTTP server such as
    \ulink{Apache}{http://httpd.apache.org/} or
    \ulink{lighttpd}{http://www.lighttpd.net/}.}

\item[caching]
    {Use a caching proxy such as \ulink{Squid}{http://www.squid-cache.org/}, or,
    again, \ulink{Apache}{http://httpd.apache.org/}. Your application should
    also do its own internal caching, of course.}

\item[authentication, authorization, sessioning, storage, templating, etc.]
    {These are your application's responsibility. There are plenty of
    \ulink{Python packages}{http://cheeseshop.python.org/pypi} available to help
    you.}

\end{description}


A future version of httpy is expected to be at least conditionally compliant
with \ulink{HTTP/1.1}{http://www.w3.org/Protocols/rfc2616/rfc2616.html}.
However, the following features are not currently implemented:

\begin{itemize}
\item
Keep-Alive\item
Transfer-Encoding\item
Range requests
\end{itemize}
\chapter{Installation}

On a \UNIX-derived system, install httpy with the following procedure:

\begin{verbatim}
% fetch http://www.zetadev.com/software/httpy/0.3/httpy-0.3.tbz
% tar jxf httpy-0.3.tbz
% cd httpy-0.3
% make install
\end{verbatim}

This will install the httpy package into your system's python, and will install
the executable and manual page under /usr/local. To use a specific python
installation, change the installation location, or otherwise configure the
installation, edit the Makefile.

To only install the python package, use:

\begin{verbatim}
% python setup.py install
\end{verbatim}

In particular, Windows users are currently limited to this method, because I
haven't yet found a good way to install an executable on Windows. I am open to
suggestions.

\chapter{Configuration}

httpy exposes the following configuration parameters:


\begin{tableiii}{l|l|l}{var}{Parameter}{Description}{Default}

\lineiii{apps}
    {A colon-separated list of paths that should be considered applications. The
    paths should be given as if they were in URL-space, or from the filesystem's
    perspective, as if they were rooted in the website's filesystem root. So
    they should begin with a slash but should not include the filesystem path to
    the root of the website. The website root is always implicitly considered an
    application.}
    {[no explicit applications]}

\lineiii{ip}
    {The IP address that httpy should listen on. This must either be a valid
    IPv4 address or a null value, in which case httpy will listen on all
    available addresses.}
    {[listen on all addresses]}

\lineiii{mode}
    {Either of the strings deployment or development. This affects various
    aspects of httpy's behavior.}
    {deployment}

\lineiii{port}
    {The TCP port that httpy should bind to. This must be an integer between 0
    and 65535, inclusive.}
    {8080}

\lineiii{root}
    {The path to a directory on the filesystem which will serve as the root of
    the website.}
    {. [i.e., the current working directory]}

\lineiii{verbosity}
    {An integer from 0 to 99, inclusive, indicating how much information to
    log.}
    {1}

\end{tableiii}

The defaults may be overriden in three ways: via command line options, through a
configuration file, and by setting environment variables. This is also the order
in which they take precedence, such that configuration file settings override
environment variables, etc. The subsections below refer to the parameter names
as given in the first column of the above table.




\section{Command Line Options}

httpy uses the Python standard library's optparse module to represent its
command line arguments. The parameter names may be used for long-named
arguments, or their initial letter may be used for short-named arguments. So for
example, httpy could be started on a custom port, with a custom website root,
using the following command:

\begin{verbatim}
% httpy -p9000 --root ~/www
\end{verbatim}

In addition, the path to any configuration file is specified with the
\programopt{-f}/\longprogramopt{file} argument.

\begin{seealso}
\seetitle [http://www.python.org/doc/lib/module-optparse.html] {optparse}
          {httpy uses the standard library's optparse module.}
\end{seealso}




\section{Configuration File Settings}

httpy uses the standard library's RawConfigParser class to represent its
configuration file. This means that the file is in the style of a Windows INI
file, with section headers in brackets, and name=value pairs on successive
lines. Use the names as given above. The number and names of any sections are
irrelevant, but RawConfigParser requires that there be at least one section. As
an example, a configuration file with the following contents would set the
verbosity of an httpy instance:

\begin{verbatim}
[main]
verbosity = 99
\end{verbatim}

\begin{seealso}
\seetitle [http://www.python.org/doc/lib/RawConfigParser-objects.html] {RawConfigParser}
          {httpy uses the standard library's RawConfigParser class.}
\end{seealso}




\section{Environment Variables}

The names of the environment variables recognized by httpy are derived by
uppercasing the above names, and prepending them with HTTPY_. So, for example,
to configure a server so that all instances of httpy on that server are run in
development mode instead of deployment mode, one would simply set the
environment variable \envvar{HTTPY_MODE} to \constant{development}.

\chapter{API}

There are two kinds of websites: publications and applications. They are
differentiated by their organization and interface models. In a
\dfn{publication}, information is organized hierarchically into folders that one
navigates by browsing. An \dfn{application} website, on the other hand,
organizes information non-hierarchically, and presents a non-browsing interface
such as a search box.

The HTML version of this documentation is an example of a publication website: a
few hypertext documents organized into sections. If we weren't using \LaTeX, the
sections would probably be encoded in folders.
\ulink{Gmail}{http://mail.google.com/mail} would be an example of a pure
application website. Most websites, however, are hybrids. For example, some
information might be organized hierarchically, but at points in that hierarchy,
one may meet an application, such as a site search feature, or a threaded
discussion forum.

httpy enables you to build publication, application, and hybrid websites. It
relies on the underlying filesystem to support a publication website's
hierarchical organization. For application websites or sections of websites,
httpy looks for certain Python files in a certain subdirectory of any
directories that you explicitly configured as applications. This is the
\strong{filesystem} aspect of its API.

Once your Python application has successfully made it into httpy's process
space, then httpy's \strong{Python} API comes into play.

Both aspects are described below.

\section{Filesystem}

\section{Python}

\begin{classdesc}{Server}{config}

foo

\end{classdesc}



\chapter{Examples}

\section{basic site}

\section{root app}

\section{non-root app}

\section{custom contig}







%
%  The ugly "%begin{latexonly}" pseudo-environments are really just to
%  keep LaTeX2HTML quiet during the \renewcommand{} macros; they're
%  not really valuable.
%
%  If you don't want the Module Index, you can remove all of this up
%  until the second \input line.
%
%begin{latexonly}
\renewcommand{\indexname}{Module Index}
%end{latexonly}
\input{mod\jobname.ind}		% Module Index

%begin{latexonly}
\renewcommand{\indexname}{Index}
%end{latexonly}
\input{\jobname.ind}			% Index

\end{document}
