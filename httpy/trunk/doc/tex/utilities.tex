\section{Utilities \label{utils}}

The \module{httpy.utils} subpackage collects several tools which may be of value
to those building web applications with \module{httpy}. The \class{Path} and
\function{translate} objects are actually used by \module{httpy}'s own
\class{Static} responder. The \function{parse_*} functions are not used in the
base package. However, all but the simplest web applications will need these or
similar methods to extract information from the \class{Request} object.


\begin{classdesc}{Path}{path}
Constructs a new \class{Path} object. \var{path} is a path to a file on the
filesystem. When a \class{Response}.\code{body} is a \class{Path}, the
third-party
\ulink{\class{sendfile}}{http://tautology.org/software/python-modules/sendfile}
module will be used if available to write the file out to the wire.
\end{classdesc}



\begin{funcdesc}{parse_body}{request}
Given a \class{Request} instance, returns an instance of the standard library's
\ulink{\module{cgi.FieldStorage}}{http://docs.python.org/lib/node471.html} class
representing the request body.
\end{funcdesc}

\begin{funcdesc}{parse_cookie}{request}
Given a \class{Request} instance, returns an instance of the standard library's
\ulink{\module{Cookie.SimpleCookie}}{http://docs.python.org/lib/module-Cookie.html} class
representing the request's cookie.
\end{funcdesc}

\begin{funcdesc}{parse_query}{request}
Given a \class{Request} instance, returns an instance of the standard library's
\ulink{\module{cgi.FieldStorage}}{http://docs.python.org/lib/node471.html} class
representing the request's querystring.
\end{funcdesc}



\begin{funcdesc}{translate}{uri_path, fs_root\optional{, defaults}\optional{, raw}}
Translates a URI path to the filesystem.

\var{uri_path} is the path component of a Request-URI (i.e.,
\class{Request}.\code{path}). \var{fs_root} is the filesystem path of the
directory in which the URI path should be rooted. \var{defaults}, if given, is a
sequence of names that should be considered default resources. If not given,
\var{defaults} is empty. If \var{raw} is given and it evaluates to \class{True},
then \code{translate()} ignores \var{defaults} and performs no validation.

If \var{raw} is \class{False}, then validation proceeds according to this
rubric:

\begin{enumerate}

\item If the translated path points to a directory, then the URI must end with a
slash, or
\ulink{301}{http://www.w3.org/Protocols/rfc2616/rfc2616-sec10.html#sec10.3.2} is
raised.

\item If the translated path points to a directory, and no default resource is
named or available, then
\ulink{403}{http://www.w3.org/Protocols/rfc2616/rfc2616-sec10.html#sec10.4.4} is
raised.

\item If the translated path does not point to a directory, then it must point
to a valid file, or
\ulink{404}{http://www.w3.org/Protocols/rfc2616/rfc2616-sec10.html#sec10.4.5} is
raised.

\end{enumerate}

\code{translate()} returns the filesystem path of the requested resource.
\end{funcdesc}
