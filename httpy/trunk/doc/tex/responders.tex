\section{Responders \label{responders}}

There are basically two kinds of websites, publications and applications,
differentiated by their organization and interface models. A \dfn{publication}
website organizes information into individual pages within a hierarchical folder
structure that one navigates by browsing. In an \dfn{application} website, on
the other hand, data is not organized into hierarchical pages but is dealt with
via a non-browsing interface such as a search box.

The HTML version of this documentation is an example of a publication website: a
number of hypertext documents organized into sections. If we weren't using LaTeX
(or if I knew how to use it better), the sections would probably be encoded in
folders. \ulink{Gmail}{http://mail.google.com/mail} is a pure application
website, one which organizes and presents information non-hierarchically. Most
websites, however, are hybrids. That is, within an overall hierarchical
organization you will find both individual pages of information as well as
applications such as a site search feature, or a threaded discussion forum.

Publication websites are actually a subset of application websites, of course.
An application site can use any interface metaphor; a publication is an
application that uses the familiar folder/page metaphor to organize and present
its information. Therefore, every website is fundamentally an application.

\module{httpy} enables you to build custom hybrid, publication, and application
websites, and its bundled responders include an implementation of each.




\begin{classdesc}{Multiple}{\optional{root}} Constructs a new \class{Multiple}
responder object. \var{root} is the filesystem path of the directory to use as
the publishing root. If not given or \class{None}, then the value in the class
attribute of the same name is used if it does not evaluate to \class{False}. If
it does (and out of the box, it does), then \var{root} is set to the current
working directory. If \var{root} does not point to a directory, \exception
{ValueError} is raised.\end{classdesc}

\begin{classdesc}{Static}{\optional{root}\optional{, defaults}} Constructs a new
\class{Static} responder object. \var{root} has the same meaning as in
\class{Multiple}. \var{defaults} is a sequence of names to use when looking for
a default resource. If it is omitted or \class{None}, it is set to
\code{('index.html', 'index.htm')}, after checking the class attribute as with
\var{root}. \end{classdesc}

\begin{classdesc*}{XMLRPC}
The XMLRPC responder is not designed to be instantiated directly, but rather
used as a mixin.
\end{classdesc*}


\chapter{the \class{Multiple} responder \label{multiple}}

httpy is built around the idea that there are two kinds of websites,
publications and applications, which are differentiated by their organization
and interface models. A \dfn{publication} website organizes information into
individual pages within a hierarchical folder structure that one navigates by
browsing. In an \dfn{application} website, on the other hand, data is not
organized into hierarchical pages but is dealt with via a non-browsing interface
such as a search box.

The HTML version of this documentation is an example of a publication website: a
number of hypertext documents organized into sections. If we weren't using LaTeX
(or if I knew how to use it better), the sections would probably be encoded in
folders. \ulink{Gmail}{http://mail.google.com/mail} is a pure application
website, one which organizes and presents information non-hierarchically. Most
websites, however, are hybrids. That is, within an overall hierarchical
organization you will find both individual pages of information as well as
applications such as a site search feature, or a threaded discussion forum.

Now, in point of fact, publication websites are a subset of application
websites. An application site can use any interface metaphor; a publication is
an application that uses the familiar folder/page metaphor to organize and
present its information. Therefore, every website is fundamentally an
application. httpy's default application implements the basic publication
scenario: serving static files straight off the filesystem.

httpy enables you to build custom publication, application, and hybrid websites.
It privileges the filesystem in supporting a publication or hybrid website's
hierarchical organization. To support applications, httpy looks for a certain
filesystem layout with files that define certain Python objects, which it
incorporates into the HTTP request/response process. These are described below.



\section{Filesystem Layout}

When httpy starts up, it walks the tree rooted in your website's filesystem
root, and it gathers all paths to directories below the root that have a
subdirectory named '__' -- that's two underscores. The directory is said to be
an \dfn{app} or \dfn{application}, and the subdirectory is that application's
\dfn{magic directory}. \footnote{You can override this default behavior. Please
see the section on Configuration.}

For example, consider this filesystem layout:

\begin{verbatim}
/usr/local/www/
/usr/local/www/about/
/usr/local/www/news/
/usr/local/www/news/__
\end{verbatim}

Assuming that \file{/usr/local/www/} is configured as your website root, httpy
will by default find one app: /news. This app's magic directory will be
\file{/usr/local/www/__}.

On each request, httpy looks for an application that matches the beginning of
the requested URI path. So in the above example, requests for \file{/news},
\file{/news/archive}, \file{/news/post}, etc., would be handled by the
\code{/news} app. All other requests would be handled by the default httpy
application, because there is no app configured for the site root.

If you manually configure the \var{apps} setting, and a given app does not have
a magic directory, httpy will raise \class{ConfigError} and exits.



\section{Expected Python Objects}

When httpy starts up, it attempts to import a module or package named
\module{app} from each application's magic directory. Each \module{app}
module/package must define a class named \class{Application}.
\class{Application}'s constructor must take a single argument, an instance of
\class{AppConfig}. \class{Application} must have at least one other method,
named \code{respond}, which also takes a single argument, an instance of
\class{Request}. If any of these conditions are not met, httpy raises
\class{ConfigError} and exits. Therefore, here is the minimal legal
\file{app.py}:

\begin{verbatim}
class Application:

    def __init__(self, config):
        pass

    def respond(self, request):
        pass

\end{verbatim}

\class{Application.respond} is expected to end the request by raising a
\class{Response} or other \class{Exception}. Therefore, here is the minimal
successful \file{app.py}:

\begin{verbatim}
from httpy.Response import Response

class Application:

    def __init__(self, config):
        pass

    def respond(self, request):
        response = Response(200)
        response.headers['content-type'] = 'text/plain'
        response.body = "Greetings, program!"
        raise response

\end{verbatim}



\section{Site-Specific Libraries}

If a website root has a magic directory, httpy will look when it starts up for a
subdirectory of this magic directory named \file{site-packages}. It will prepend
this path to \class{sys.path}, with the effect that each website can maintain a
local Python library in \file{<site-root>/__/site-packages}. This library is
global to the applications within a website, however, and only the root's magic
directory is searched for \file{site-packages}.



\subsection{\class{Static} Responders \label{static}}

The \class{Static} responder implements the basic publication application:
serving public files straight off the filesystem. This responder is designed to
be used either by itself or as a mixin. It provides one attributes and two
methods:

\begin{memberdesc}[string]{root}
The filesystem path of the directory to use as the publishing root.
\end{memberdesc}

\begin{memberdesc}[tuple]{defaults}
A tuple of names to interpret as default resources. The first-named is chosen
first.
\end{memberdesc}


\begin{methoddesc}{respond}{request}
This is a pass-through for \method{serve_static}, and can safely be overriden.
\end{methoddesc}

\begin{methoddesc}{serve_static}{request} Serves a static resource from the
filesystem. \var{request} is a \class{Request} object. The URI is translated
using the \function{httpy.utils.translate} function, and the
\mailheader{Content-Type} is set by the standard library's
\ulink{\module{mimetypes}}{http://docs.python.org/lib/module-mimetypes.html} module. In
\code{staging} and \code{deployment} modes, \method{serve_static} supports the
\ulink{\code{304 Not
Modified}}{http://www.w3.org/Protocols/rfc2616/rfc2616-sec10.html#sec10.3.5}
response.
\end{methoddesc}


\subsection{\class{XMLRPC} Responders \label{xmlrpc}}

\ulink{XMLRPC}{http://www.xmlrpc.com/} is a language-neutral protocol for
distributed computing, using XML for serialization and HTTP for transport. The
Python standard library includes \ulink{a well-written XMLRPC
library}{http://docs.python.org/lib/module-xmlrpclib.html}, which forms the
basis for this class. By mixing the \class{XMLRPC} responder into your class,
you can easily make your instance's methods available to clients over the
network. Instances inherit one attribute and two methods:

\begin{memberdesc}[tuple]{protected} A tuple of names of methods on \class{self}
which should never be served via XMLRPC. This class attribute on \class{XMLRPC}
is set to the empty tuple.\end{memberdesc}

\begin{methoddesc}{respond}{request}
This is a pass-through for \method{serve_xmlrpc}. It can safely be overriden, but
in most cases need not be.
\end{methoddesc}

\begin{methoddesc}{serve_xmlrpc}{request}
Proxies methods on \class{self} via XMLRPC. \var{request} is a \class{Request}
object. Assuming \var{request} is an XMLRPC request for method \var{name}, the
following conditions trigger an XMLRPC <fault> response with <faultCode> 404:

\begin{enumerate}

\item \var{name} starts with an underscore.

\item \var{name} is \code{respond} or \code{serve_xmlrpc}.

\item \var{name} is named in \code{protected}.

\item The instance does not have a method named \var{name}.

\end{enumerate}
\end{methoddesc}


\subsubsection{An Example}

In a Python shell, create an XMLRPC server like so:

\begin{verbatim}
>>> import httpy
>>> class Responder(httpy.responders.XMLRPC):
...   protected = ['private']
...   def private(self):
...     return 'leave me alone!'
...   def ping(self):
...     return 'pong'
...
>>> responder = Responder()
>>> coupler = httpy.couplers.StandAlone(responder)
>>> coupler.go()
httpy.server     INFO     httpy started on port 8080
\end{verbatim}

Then, in a second shell, you can talk to your server like this:

\begin{verbatim}
>>> import xmlrpclib
>>> server = xmlrpclib.ServerProxy('http://localhost:8080/')
>>> server.ping()
'pong'
>>> server.private()
Traceback (most recent call last):
...
xmlrpclib.Fault: <Fault 404: "method 'private' not found">
>>>
\end{verbatim}

\subsection{Writing a Responder \label{writing-a-responder}}

In \module{httpy}, a \dfn{responder} is simply an object with a
\function{respond} routine. For example, a responder can be a module with a
\function{respond} function, or a class instance with a \method{respond} method.
Responders have two optional data attributes, which, if not present, will be
added by whatever coupler is in use. For responders that are classes, this API
addition takes place prior to instantiation, so the information is available to
the constructor. See the coupler documentation for more information.


\begin{funcdesc}{respond}{request} Responds to a single HTTP request.
\var{request} is a \class{Request} object. The \code{raise} statement is used to
end the transaction. If a \class{Response} object is raised, it will be
converted to an HTTP Response message and written out to the wire. All other
\class{Exception}s will result in a \ulink{\code{500 Internal Server
Error}}{http://www.w3.org/Protocols/rfc2616/rfc2616-sec10.html#sec10.5.1}
response.\end{funcdesc}

\begin{datadesc}{root} The location of the responder on the local
filesystem.\end{datadesc}

\begin{datadesc}{uri}The location of the responder on the network.\end{datadesc}




\subsubsection{An Example}

Here is an example showing the general feel of a responder. This example assumes
a \class{logic} module with API for getting and setting data based on a URI path
and a POST body. The suggestion here is that these might return a commonly
formatted data structure, which would then be used to populate a common
template. Notice the authorization check before setting data.

\begin{verbatim}
import auth
import logic
import templating

from httpy import Response

def respond(request):

    if request.method == 'GET':
        result = logic.get_data(request.path)
    elif request.method == 'POST':
        if not auth.check(request):
            raise Response(403)
        result = logic.set_data(request.path, request.body)
    else:
        raise Response(501)

    template = templating.get_template(request.path)
    body = template.render(result)
    raise Response(200, body)

\end{verbatim}


This is a contrived example, basically following \ulink{the popular
Model-View-Controller pattern}{http://c2.com/cgi/wiki?ModelViewController}: the
\module{auth} and \module{logic} modules are the model, the \module{templating}
module provides the view, and the responder is the controller. However, this
pattern is not enforced in any way, and the bottom line is that you've got all
of Python to play with in writing your responders.

