\section{Couplers \label{couplers}}

The action in \module{httpy} is in the responders, but the couplers are the
workhorses that put responders on the network. \module{httpy} includes three
couplers out of the box. There are multiple Python webservers available, and a
couple Python FastCGI libraries too. I've tried to pick the single most mature
option in each category, and build the couplers on top of those. These decisions
are purely pragmatic.

\begin{classdesc}{CGI}{responder} Constructs a new \class{CGI} coupler object.
\end{classdesc}

\begin{classdesc}{FastCGI}{responder} Constructs a new \class{FastCGI} coupler
object.
\end{classdesc}

\begin{classdesc}{StandAlone}{responder\optional{, argv}} Constructs a new
\class{StandAlone} coupler object. \var{argv} is a sequence of arguments. If
omitted or \class{None}, \code{sys.argv} is used. See \ulink{the chapter on the
\program{httpy} executable}{manual.html} for the available options.
\end{classdesc}

Each coupler takes a (possibly implicit) provider of the \class{IResponder}
interface as its first argument, and each provides the following method:

\begin{methoddesc}{go}{} For \class{CGI} instances, this responds to a single
request. For \class{FastCGI} and \class{StandAlone}, this enters a blocking
loop.\end{methoddesc}

As an example, here is what a CGI script looks like with \module{httpy}:

\begin{verbatim}
#!/usr/local/bin/python
"""This is a CGI script.
"""
import httpy
import myresponder

coupler = httpy.couplers.CGI(myresponder)
coupler.go()
\end{verbatim}

Those wishing to implement new couplers are invited to consult the source code,
and to contact the author.