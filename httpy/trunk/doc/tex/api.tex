\chapter{API}

There are two kinds of websites: publications and applications. They are
differentiated by their organization and interface models. In a
\dfn{publication}, information is organized hierarchically into folders that one
navigates by browsing. An \dfn{application} website, on the other hand,
organizes information non-hierarchically, and presents a non-browsing interface
such as a search box.

The HTML version of this documentation is an example of a publication website: a
few hypertext documents organized into sections. If we weren't using \LaTeX, the
sections would probably be encoded in folders.
\ulink{Gmail}{http://mail.google.com/mail} would be an example of a pure
application website. Most websites, however, are hybrids. For example, some
information might be organized hierarchically, but at points in that hierarchy,
one may meet an application, such as a site search feature, or a threaded
discussion forum.

httpy enables you to build publication, application, and hybrid websites. It
relies on the underlying filesystem to support a publication website's
hierarchical organization. For application websites or sections of websites,
httpy looks for certain Python files in a certain subdirectory of any
directories that you explicitly configured as applications. This is the
\strong{filesystem} aspect of its API.

Once your Python application has successfully made it into httpy's process
space, then httpy's \strong{Python} API comes into play.

Both aspects are described below.

\section{Filesystem}

\section{Python}

\subsection{\module{httpy.Request} ---
            An HTTP Request}

\subsection{\module{httpy.Response} ---
            An HTTP Response}

\subsection{\module{httpy.Server} ---
            An httpy server}

\begin{classdesc}{Server}{config}

foo
\end{classdesc}

\subsection{\module{httpy.app} ---
            The default application used to process requests}

\subsection{\module{httpy.helpers} ---
            Useful mixins for custom applications}

