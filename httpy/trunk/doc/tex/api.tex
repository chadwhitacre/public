\chapter{Package Contents/API \label{api}}

The \module{httpy} package defines classes to represent HTTP Request and
Response messages. It also defines interfaces specifying how to build responders
and couplers, specimens of which are to be found in the \module{responders} and
\module{couplers} subpackages. Finally, the \class{mode} singleton provides an
object-oriented API for the \envvar{HTTPY_MODE} environment variable, and the
\module{utils} subpackage collects some other possibly useful utilities.



\begin{classdesc}{Request}{IRequest}
Constructs a new \class{Request} object. \var{IRequest} is a (probably implicit)
provider of the \class{IRequest} interface. This provision is validated, and the
desired API is transferred from the \class{IRequest} provider to the new object
instance.

Your responder's \method{respond} method will be given instances of this class,
so you will be using it constantly on that basis. However, you would probably
only need to instantiate it directly if you were writing a new coupler.
\end{classdesc}



\begin{classdesc}{Response}{\optional{code} \optional{, body} \optional{,
    headers}}
Constructs a new \class{Response} object. If given, \var{code} must be
an integer; the default is
\ulink{200}{http://www.w3.org/Protocols/rfc2616/rfc2616-sec10.html#sec10.2.1}
(see \citetitle[http://www.w3.org/Protocols/rfc2616/rfc2616-sec10.html]{the HTTP
spec} for other values that will be meaningful to most HTTP clients). \var{body}
may be a string, an iterator over strings, or an \class{httpy.utils.Path}
object. \var{headers} must be a dictionary.

\var{body} is second rather than \var{headers} because one more often wants to
specify a body without headers than vice versa. Also note that
\mailheader{Content-Type} defaults to \code{text/html} for responses where
\var{code} is between 200 and 299, inclusive, but to \code{text/plain} for
non-2xx responses.

This class is likewise central to \module{httpy} programming: its instances are
the payload for \module{httpy}'s overloaded \code{raise} statement. These will
be caught, validated, flattened, and sent out to the wire by whatever coupler is
in use.
\end{classdesc}


\section{\class{Request} Objects \label{request}}

HTTP is a stateless protocol. Any application state that exists in a request is
encoded in one of the following places:

\begin{itemize}
\item{the Request Method}
\item{the path}
\item{a querystring}
\item{another part of the Request-URI}
\item{a \mailheader{Cookie} header}
\item{another header}
\item{a \code{POST} body}
\end{itemize}

httpy doesn't pretend to know the best way for you to encode state in your app's
requests, nor the best way to object-represent this state. Instead, it gives you
the raw HTTP message, and a very minimal secondary API. There are a few tools in
\class{httpy.utils} that might be helpful -- \function{parse_body},
\function{parse_cookie}, \function{parse_query} -- but you are free to translate
the \class{Request} to your application's object model however you like.


\subsection{Raw API}

Instances of \class{httpy.Request} store the raw HTTP Request message in the
following attributes:

\begin{datadesc}{raw}
The entire \ulink{HTTP Request message}{http://www.w3.org/Protocols/rfc2616/rfc2616-sec5.html} exactly as it was found on the wire.
\end{datadesc}

\begin{datadesc}{raw_line}
The raw
\ulink{Request-Line}{http://www.w3.org/Protocols/rfc2616/rfc2616-sec5.html#sec5.1},
 not including the trailing line break.
\end{datadesc}

\begin{datadesc}{raw_headers}
The raw message headers, not including the trailing line breaks.
\end{datadesc}

\begin{datadesc}{raw_body}
The raw message body.
\end{datadesc}



\subsection{Derived API}

Besides providing access to the raw Request message, \class{Request} instances
provide a very minimal derivative API:

\begin{datadesc}{method}
The \ulink{HTTP method}{http://www.w3.org/Protocols/rfc2616/rfc2616-sec9.html} from the Request-Line.
\end{datadesc}

\begin{datadesc}{uri}
The \ulink{Request-URI}{http://www.w3.org/Protocols/rfc2616/rfc2616-sec5.html#sec5.1.2} as a dictionary. The keys of this dictionary are taken from the
names used in the standard library's
\ulink{\module{urlparse}}{http://docs.python.org/lib/module-urlparse.html} module,
namely: \emph{scheme://netloc/path;parameters?query\#fragment}.
\end{datadesc}

\begin{datadesc}{path}
The \emph{path} component of the Request-URI.
\end{datadesc}

\begin{datadesc}{headers}
The headers as an instance of the standard library's
\ulink{\module{email.Message.Message}}{http://docs.python.org/lib/module-email.Message.html}.
\end{datadesc}


\subsection{Differences under CGI/FastCGI}

Whereas the \class{StandAlone} coupler builds \class{Request} objects directly
from the raw request coming off the wire, the \class{CGI} and \class{FastCGI}
couplers must resort to reconstructing \class{Request}s from the environment and
standard input. The resulting object has the same API as above, but since some
attributes could have subtly different meanings, we provide the following
reference. Where called for, \code{\e r\e n} line breaks are used.

\begin{datadesc}{raw}
The concatenation of \member{raw_line}, \member{raw_headers}, and
\member{raw_body} as described below.
\end{datadesc}

\begin{datadesc}{raw_line}
The Request-URI is reconstructed from the \envvar{SCRIPT_NAME},
\envvar{PATH_INFO}, and \envvar{QUERY_STRING} environment variables. This is
combined with the \envvar{REQUEST_METHOD} and \envvar{SERVER_PROTOCOL} variables
to approximate the Request-Line.
\end{datadesc}

\begin{datadesc}{raw_headers}
The headers are reconstructed by taking all environment variables beginning with
"HTTP_" and replacing all underscores with dashes.
\end{datadesc}

\begin{datadesc}{raw_body}
Read from the standard input.
\end{datadesc}

\begin{datadesc}{method}
Equivalent to the \envvar{REQUEST_METHOD} environment variable.
\end{datadesc}

\begin{datadesc}{uri}
The dictionary is based on \member{raw_line} as derived above.
\end{datadesc}

\begin{datadesc}{path}
Equivalent to the \envvar{PATH_INFO} environment variable.
\end{datadesc}

\begin{datadesc}{headers}
The \class{Message} object is constructed from \member{raw_headers} as described
above.
\end{datadesc}


\begin{seealso}
    \seetitle[http://hoohoo.ncsa.uiuc.edu/cgi/interface.html]{The CGI
    Specification}{ The reference for CGI, including use of environment
    variables and standard input.}
\end{seealso}

\section{\class{Response} Objects \label{response}}

Instances of \class{httpy.Response} have the following attributes. Note that
values are only validated in the constructor, so it is currently possible to
raise a malformed \class{Response} by setting instance attributes
post-instantiation.

\begin{datadesc}{code}
The \ulink{HTTP code}{http://www.w3.org/Protocols/rfc2616/rfc2616-sec10.html} as an integer.
\end{datadesc}

\begin{datadesc}{body}
The message body as a string or an iterator over strings.
\end{datadesc}

\begin{datadesc}{headers}
The message headers as an instance of the standard library's
\ulink{\module{email.Message.Message}}{http://docs.python.org/lib/module-email.M
essage.html}.
\end{datadesc}

\section{Responders \label{responders}}

\module{httpy} includes three responders out of the box. The fun of
\module{httpy}, though, is in writing your own responders.

\begin{classdesc}{Static}{\optional{root}\optional{, defaults}} Constructs a new
\class{Static} responder object. \var{root} is the filesystem path of the
directory to use as the publishing root. If not given or \class{None}, then the
value in the class attribute of the same name is used if it does not evaluate to
\class{False}. If it does (and out of the box, it does), then \var{root} is set
to the current working directory. If \var{root} does not point to a directory,
\exception {ValueError} is raised. \var{defaults} is a sequence of names to use
when looking for a default resource. If it is omitted or \class{None}, it is set
to \code{('index.html', 'index.htm')}, after checking the class attribute as
with \var{root}. \end{classdesc}

\begin{classdesc}{Multiple}{\optional{root}} Constructs a new \class{Multiple}
responder object. \var{root} has the same meaning as in
\class{Static}.\end{classdesc}

\begin{classdesc*}{XMLRPC}
The XMLRPC responder is not designed to be instantiated directly, but rather
used as a mixin.
\end{classdesc*}


\subsection{\class{Multiple} Responders \label{multiple}}

The \class{Multiple} responder implements a hybrid website. The URI hierarchy is
taken straight from the filesystem. At any point in the site's filesystem tree,
you can place a file named responder.py, which can either be a responder

, using the filesystem for hierarchical organization. URI hierarchy is
mapped directly are to the filesystem, and chooses different responders based on
the path. The various sub-applications are defined in files named
\code{responder.py}, The Static responder picks up anything that falls through
the cracks.

\subsubsection{Filesystem Layout}

When httpy starts up, it walks the tree rooted in your website's filesystem
root, and it gathers all paths to directories below the root that have a
subdirectory named '__' -- that's two underscores. The directory is said to be
an \dfn{app} or \dfn{application}, and the subdirectory is that application's
\dfn{magic directory}. \footnote{You can override this default behavior. Please
see the section on Configuration.}

For example, consider this filesystem layout:

\begin{verbatim}
/usr/local/www/
/usr/local/www/about/
/usr/local/www/news/
/usr/local/www/news/__
\end{verbatim}

Assuming that \file{/usr/local/www/} is configured as your website root, httpy
will by default find one app: /news. This app's magic directory will be
\file{/usr/local/www/__}.

On each request, httpy looks for an application that matches the beginning of
the requested URI path. So in the above example, requests for \file{/news},
\file{/news/archive}, \file{/news/post}, etc., would be handled by the
\code{/news} app. All other requests would be handled by the default httpy
application, because there is no app configured for the site root.

If you manually configure the \var{apps} setting, and a given app does not have
a magic directory, httpy will raise \class{ConfigError} and exits.



\subsubsection{Expected Python Objects}

When httpy starts up, it attempts to import a module or package named
\module{app} from each application's magic directory. Each \module{app}
module/package must define a class named \class{Application}.
\class{Application}'s constructor must take a single argument, an instance of
\class{AppConfig}. \class{Application} must have at least one other method,
named \code{respond}, which also takes a single argument, an instance of
\class{Request}. If any of these conditions are not met, httpy raises
\class{ConfigError} and exits. Therefore, here is the minimal legal
\file{app.py}:

\begin{verbatim}
class Application:

    def __init__(self, config):
        pass

    def respond(self, request):
        pass

\end{verbatim}

\class{Application.respond} is expected to end the request by raising a
\class{Response} or other \class{Exception}. Therefore, here is the minimal
successful \file{app.py}:

\begin{verbatim}
from httpy.Response import Response

class Application:

    def __init__(self, config):
        pass

    def respond(self, request):
        response = Response(200)
        response.headers['content-type'] = 'text/plain'
        response.body = "Greetings, program!"
        raise response

\end{verbatim}



\subsubsection{Site-Specific Libraries}

If a website root has a magic directory, httpy will look when it starts up for a
subdirectory of this magic directory named \file{site-packages}. It will prepend
this path to \class{sys.path}, with the effect that each website can maintain a
local Python library in \file{<site-root>/__/site-packages}. This library is
global to the applications within a website, however, and only the root's magic
directory is searched for \file{site-packages}.



\subsection{\class{Static} Responders \label{static}}

The \class{Static} responder is designed to be used either by itself or as a
mixin. It provides two attributes and two methods:

\begin{memberdesc}[string]{root}
The filesystem path of the directory to use as the publishing root.
\end{memberdesc}

\begin{memberdesc}[tuple]{defaults}
A tuple of names to interpret as default resources. The first-named is chosen
first.
\end{memberdesc}


\begin{methoddesc}{respond}{request}
This is a pass-through for \method{serve_static}.
\end{methoddesc}

\begin{methoddesc}{serve_static}{request} Serves a static resource from the
filesystem. \var{request} is a \class{Request} object. The URI is translated
using the \function{httpy.utils.translate} function, and the
\mailheader{Content-Type} is set by the standard library's
\ulink{\module{mimetypes}}{http://docs.python.org/lib/module-mimetypes.html} module. In
\code{staging} and \code{deployment} modes, \method{serve_static} supports the
\ulink{\code{304 Not
Modified}}{http://www.w3.org/Protocols/rfc2616/rfc2616-sec10.html#sec10.3.5}
response.
\end{methoddesc}


\subsection{\class{XMLRPC} Responders \label{xmlrpc}}

\ulink{XMLRPC}{http://www.xmlrpc.com/} is a language-neutral protocol for
distributed computing, using XML for serialization and HTTP for transport. The
Python standard library includes \ulink{a well-written XMLRPC
library}{http://docs.python.org/lib/module-xmlrpclib.html}, which forms the
basis for this class. By mixing the \class{XMLRPC} responder into your class,
you can easily make your instance's methods available to clients over the
network. Instances inherit one attribute and two methods:

\begin{memberdesc}[tuple]{protected} A tuple of names of methods of \class{self}
which should never be served via XMLRPC. This class attribute on \class{XMLRPC}
is set to the empty tuple.\end{memberdesc}

\begin{methoddesc}{respond}{request}
This is a pass-through for \method{serve_xmlrpc}.
\end{methoddesc}

\begin{methoddesc}{serve_xmlrpc}{request}
Proxies methods of \class{self} via XMLRPC. \var{request} is a \class{Request}
object. Assuming \var{request} is an XMLRPC request for method \var{name}, the
following conditions trigger an XMLRPC <fault> response with <faultCode> 404:

\begin{enumerate}

\item \var{name} starts with an underscore.

\item \var{name} is \code{respond} or \code{serve_xmlrpc}.

\item \var{name} is named in \code{protected}.

\item The instance does not have a method named \var{name}.

\end{enumerate}
\end{methoddesc}


\subsubsection{An Example}

In a Python shell, create an XMLRPC server like so:

\begin{verbatim}
>>> import httpy
>>> class Responder(httpy.responders.XMLRPC):
...   protected = ['private']
...   def private(self):
...     return 'leave me alone!'
...   def ping(self):
...     return 'pong'
...
>>> responder = Responder()
>>> coupler = httpy.couplers.StandAlone(responder)
>>> coupler.go()
httpy.server     INFO     httpy started on port 8080
\end{verbatim}

Then, in a second shell, you can talk to your server like this:

\begin{verbatim}
>>> import xmlrpclib
>>> server = xmlrpclib.ServerProxy('http://localhost:8080/')
>>> server.ping()
'pong'
>>> server.private()
Traceback (most recent call last):
...
xmlrpclib.Fault: <Fault 404: "method 'private' not found">
>>>
\end{verbatim}



\subsection{Writing a Responder \label{writing-a-responder}}

\section{Couplers \label{couplers}}

The action in \module{httpy} is in the responders, but the couplers are the
workhorses that put responders on the network. \module{httpy} includes three
couplers out of the box. There are multiple Python webservers available, and a
couple Python FastCGI libraries too. I've tried to pick the single most mature
option in each category, and build the couplers on top of those. These decisions
are purely pragmatic.

\begin{classdesc}{CGI}{Responder} Constructs a new \class{CGI} coupler object.
This class is defined in the \module{httpy.couplers.cgi} module.
\end{classdesc}

\begin{classdesc}{FastCGI}{Responder} Constructs a new \class{FastCGI} coupler
object. This class is defined in the \module{httpy.couplers.fastcgi} module.
\end{classdesc}

\begin{classdesc}{StandAlone}{Responder\optional{, argv}} Constructs a new
\class{StandAlone} coupler object. \var{argv} is a sequence of arguments. If
omitted or \class{None}, \code{sys.argv} is used. See \ulink{the chapter on the
\program{httpy} executable}{manual.html} for the available options. This class
is defined in the \module{httpy.couplers.standalone} module. \end{classdesc}

Each coupler takes a class defining a (possibly implicit) provider of the
\class{IResponder} interface as its first argument, and each provides the
following method:

\begin{methoddesc}{go}{} For \class{CGI} instances, this responds to a single
request. For \class{FastCGI} and \class{StandAlone}, this enters a blocking
loop.\end{methoddesc}

As an example, here is what a CGI script looks like with \module{httpy}:

\begin{verbatim}
#!/usr/local/bin/python
"""This is a CGI script.
"""
from httpy.couplers.cgi import CGI
import MyResponder

coupled = CGI(MyResponder)
coupled.go()
\end{verbatim}

Those wishing to implement new couplers are invited to consult the source code,
and to contact the author.
\section{The \class{mode} Object \label{mode}}

Websites and web applications go through a life-cycle involving development,
debugging, staging, and deployment. It is often desirable to alter behavior
throughout the application based on the current stage in this life-cycle, for
example, to use a different database connection string in deployment than in
development or staging.

\module{httpy} models this common requirement via a \class{mode} singleton,
which has the following members:

\begin{memberdesc}[boolean]{IS_DEVELOPMENT}
\memberline[boolean]{IS_DEBUGGING}
\memberline[boolean]{IS_STAGING}
\memberline[boolean]{IS_DEPLOYMENT}
    These constants are boolean values based on the current mode, and only one
    will be \class{True} at any given time. Abbrevations and alternate casings
    are allowed; e.g., \constant{IS_DEV} and \constant{is_develo} are both
    aliases for \constant{IS_DEVELOPMENT}. \constant{IS_DE} is an
    \class{AttributeError}, however.
\end{memberdesc}

\begin{methoddesc}{__repr__}{}
    Returns the current mode as a lowercase string. This will always be one of
    \code{development}, \code{debugging}, \code{staging}, or \code{deployment}.
\end{methoddesc}

\begin{methoddesc}{__str__}{}
    Alias for \method{__repr__}.
\end{methoddesc}

\begin{memberdesc}{default}
    Contains the default mode as a lowercase string. Out of the box, this is
    \code{development}.
\end{memberdesc}


\module{httpy}'s current mode is determined by the environment variable
\envvar{HTTPY_MODE}. Since the mode of an application instance is generally only
defined at start-up, this API is intended to be read-only. However, \class{mode}
checks the environment on each call or attribute access, so if you must change
the mode on the fly, you can.

Other parts of the \module{httpy} package alter their behavior according to the
current mode. Here is a reference:

\begin{description}

\item[\code{development}]
    Internal server errors generate a traceback in the browser.

    On \UNIX{} systems, the \class{StandAlone} coupler monitors the filesystem
    counterparts of all loaded modules, and restarts itself if the modification
    time of any of these files changes.

\item[\code{debugging}]
    Equivalent to \code{development}. Additionally, non-\class{Response}
    \class{Exception}s trigger post-mortem debugging via
    \ulink{\module{pdb}}{http://docs.python.org/lib/module-pdb.html}.

\item[\code{staging}]
    Equivalent to \code{deployment}.

\item[\code{deployment}]
    The \class{Static} responder supports the \ulink{304 Not
    Modified}{http://www.w3.org/Protocols/rfc2616/rfc2616-sec10.html#sec10.3.5}
    response.

\end{description}

\section{Utilities \label{utils}}

The \module{httpy.utils} subpackage collects several tools which may be of value
to those building web applications with \module{httpy}. The \function{translate}
function is actually used by \module{httpy}'s own \class{Static} and
\class{Multiple} responders. The \function{parse_*} functions are not used in
the base package, but even the simplest web applications will need these or
similar methods to extract application state from the \class{Request} object.


\begin{funcdesc}{parse_body}{request}
Given a \class{Request} instance, returns an instance of the standard library's
\ulink{\module{cgi.FieldStorage}}{http://docs.python.org/lib/node471.html} class
representing the request body.
\end{funcdesc}

\begin{funcdesc}{parse_cookie}{request}
Given a \class{Request} instance, returns an instance of the standard library's
\ulink{\module{Cookie.SimpleCookie}}{http://docs.python.org/lib/module-Cookie.html} class
representing the request's cookie.
\end{funcdesc}

\begin{funcdesc}{parse_query}{request}
Given a \class{Request} instance, returns an instance of the standard library's
\ulink{\module{cgi.FieldStorage}}{http://docs.python.org/lib/node471.html} class
representing the request's querystring.
\end{funcdesc}


\begin{funcdesc}{translate}{uri_path, fs_root\optional{, defaults}\optional{, raw}}
Translates a URI path to the filesystem.

\var{uri_path} is the path component of a Request-URI (i.e.,
\class{Request}.\code{path}). \var{fs_root} is the filesystem path of the
directory in which the URI path should be rooted. \var{defaults}, if given, is a
sequence of names that should be considered default resources. If not given,
\var{defaults} is empty. If \var{raw} is given and it evaluates to \class{True},
then \code{translate()} ignores \var{defaults} and performs no validation. If
\var{raw} is \class{False}, then validation proceeds according to this rubric:

\begin{enumerate}

\item If the translated path points to a directory, then the URI must end with a
slash, or \ulink{301 Moved
Permanently}{http://www.w3.org/Protocols/rfc2616/rfc2616-sec10.html#sec10.3.2}
is raised.

\item If the translated path points to a directory, and no default resource is
named or available, then \ulink{403
Forbidden}{http://www.w3.org/Protocols/rfc2616/rfc2616-sec10.html#sec10.4.4} is
raised.

\item If the translated path does not point to a directory, then it must point
to a valid file, or \ulink{404 Not
Found}{http://www.w3.org/Protocols/rfc2616/rfc2616-sec10.html#sec10.4.5} is
raised.

\end{enumerate}

\code{translate()} returns the filesystem path of the requested resource.
\end{funcdesc}

