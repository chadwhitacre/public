\chapter{Package Contents}

The \module{httpy} package defines classes to represent HTTP Request and
Response messages. It also defines interfaces specifying how to build responders
and couplers, specimens of which are to be found in the \module{responders} and
\module{couplers} subpackages. Finally, the \class{mode} singleton provides an
object-oriented API for the \envvar{HTTPY_MODE} environment variable, and the
\module{utils} subpackage collects some other possibly useful utilities.



\begin{classdesc}{Request}{IRequest}

Constructs a new \class{Request} object. \var{IRequest} is a provider of the
\class{IRequest} interface. This provision is validated, and the desired API is
transferred from the \class{IRequest} provider to the new object instance.

Your responder's \method{respond} method will be given instances of this class,
so you will be using it constantly on that basis. However, you would probably
only need to instantiate it directly if you were writing a new coupler.

\end{classdesc}



\begin{classdesc}{Response}{\optional{code} \optional{, body} \optional{, headers}}

Constructs a new \class{Response} object. If given, \var{code} must be an
integer. The default is 200; see
\citetitle[http://www.w3.org/Protocols/rfc2616/rfc2616-sec10.html]{the HTTP
spec} for other values that will be meaningful to most HTTP clients. \var{body}
may be a string, an iterator over strings, or an \class{httpy.utils.Path}
object. \var{headers} must be a dictionary.

\var{body} is second rather than \var{headers} because one more often wants to
specify a body without headers than a header without a body. Also note that
\mailheader{Content-Type} defaults to \code{text/html} for responses where
\var{code} is between 200 and 299, inclusive, but to \code{text/plain} for
non-2xx responses.

This class is likewise central to \module{httpy} programming: its instances are
the payload for \module{httpy}'s overloaded \code{raise} statement. These will
be caught, validated, flattened, and sent out to the wire by whatever coupler is
in use.

\end{classdesc}




\section{\class{Request} Objects}

\subsection{Raw API}

Instances of \class{httpy.Request} store the raw HTTP request in the following
attributes:

\begin{datadesc}{raw}
The entire \ulink{HTTP Request message}{http://www.w3.org/Protocols/rfc2616/rfc2616-sec5.html} exactly as it was found on the wire.
\end{datadesc}

\begin{datadesc}{raw_line}
The raw
\ulink{Request-Line}{http://www.w3.org/Protocols/rfc2616/rfc2616-sec5.html#sec5.1},
 not including the trailing line break.
\end{datadesc}

\begin{datadesc}{raw_headers}
The raw message headers, not including the trailing line breaks.
\end{datadesc}

\begin{datadesc}{raw_body}
The raw message body.
\end{datadesc}



\subsection{Derived API}

Furthermore, \class{Request} instances provide a very minimal derivative API:

\begin{datadesc}{method}
The \ulink{HTTP method}{http://www.w3.org/Protocols/rfc2616/rfc2616-sec9.html} from the Request-Line.
\end{datadesc}

\begin{datadesc}{uri}
The \ulink{Request-URI}{http://www.w3.org/Protocols/rfc2616/rfc2616-sec5.html#sec5.1.2} as a dictionary. The keys of this dictionary are taken from the
names used in the standard library's
\ulink{\module{urlparse}}{http://docs.python.org/lib/module-urlparse.html},
namely: \emph{scheme://netloc/path;parameters?query\#fragment}.
\end{datadesc}

\begin{datadesc}{path}
The \emph{path} component of the Request-URI.
\end{datadesc}

\begin{datadesc}{headers}
the headers as an instance of the standard library's
\ulink{\module{email.Message.Message}}{http://docs.python.org/lib/module-email.Message.html}.
\end{datadesc}


\subsection{Differences under CGI/FastCGI}

Whereas the \class{StandAlone} coupler builds \class{Request} objects directly
from the raw request coming off the wire, the \class{CGI} and \class{FastCGI}
couplers must resort to reconstructing \class{Request}s from the environment and
standard input. The resulting object has the same API as above, but since some
attributes could have subtly different meanings, we provide the following
reference. Where called for, \code{\e r\e n} line breaks are used.

\begin{datadesc}{raw}
The concatenation of raw_line, raw_headers, and raw_body as described below.
\end{datadesc}

\begin{datadesc}{raw_line}
The Request-URI is reconstructed from the \envvar{SCRIPT_NAME},
\envvar{PATH_INFO}, and \envvar{QUERY_STRING} environment variables. This is
combined with the \envvar{REQUEST_METHOD} and \envvar{SERVER_PROTOCOL} variables
to approximate the Request-Line.
\end{datadesc}

\begin{datadesc}{raw_headers}
The headers are reconstructed by taking all environment variables beginning with
"HTTP_" and replacing all underscores with dashes.
\end{datadesc}

\begin{datadesc}{raw_body}
Read from the standard input.
\end{datadesc}

\begin{datadesc}{method}
Equivalent to the \envvar{REQUEST_METHOD} environment variable.
\end{datadesc}

\begin{datadesc}{uri}
The dictionary is based on \var{raw_line} as derived above.
\end{datadesc}

\begin{datadesc}{path}
Equivalent to the \envvar{PATH_INFO} environment variable.
\end{datadesc}

\begin{datadesc}{headers}
The \class{Message} object is constructed from \var{raw_headers} as described
above.
\end{datadesc}


\begin{seealso}
    \seetitle[http://hoohoo.ncsa.uiuc.edu/cgi/interface.html]{The CGI
    Specification}{ The reference for CGI, including use of environment
    variables and standard input.}
\end{seealso}



\section{\class{Response} Objects}

Instances of \class{httpy.Response} have the following attributes. Note that
values are only validated in the constructor, so it is currently possible to
raise a malformed \class{Response} by setting instance attributes
post-instantiation.

\begin{datadesc}{code}
The HTTP code as an integer.
\end{datadesc}

\begin{datadesc}{body}
The message body as a string, an iterator over strings, or an
\class{httpy.utils.Path} object.
\end{datadesc}

\begin{datadesc}{headers}
The message headers as a dictionary.
\end{datadesc}



\section{\class{httpy.couplers}}
\section{\class{httpy.interfaces}}



\section{The \class{mode} Object}

Websites and web applications go through a life-cycle involving development,
debugging, staging, and deployment. It is often desirable to alter behavior
throughout the application based on the current stage in this life-cycle, for
example, to use a different database connection string in deployment than in
development or staging.

\module{httpy} models this common requirement via a \class{mode} singleton,
which provides the following members:

\begin{datadesc}{IS_DEVELOPMENT}
\dataline{IS_DEBUGGING}
\dataline{IS_STAGING}
\dataline{IS_DEPLOYMENT}
    These constants are boolean values based on the current mode, and only one
    will be \class{True} at any given time. Abbrevations and alternate casings
    are allowed: \constant{IS_DEV} and \constant{is_devel} are both aliases for
    \constant{IS_DEVELOPMENT}. \constant{IS_DE} is an \class{AttributeError},
    however.
\end{datadesc}

\begin{datadesc}{default}
    Contains the default mode as a lowercase string. Out of the box, this is
    \code{development}.
\end{datadesc}

\begin{datadesc}{__str__}
\dataline{__repr__}
    Returns the current mode as a lowercase string.
\end{datadesc}


\module{httpy}'s current mode is determined by the environment variable
\envvar{HTTPY_MODE}. Since the mode of an application instance is generally only
defined at start-up, this API is intended to be read-only. However, the instance
checks the environment on each call or attribute access, so if you must change
the mode on the fly, you can.

\module{httpy} itself alters its behavior according to the current mode. Here is
a reference:

\begin{description}

\item[\code{development}]
    {foo}

\item[\code{debugging}]
    {Equivalent to \code{development}, except that non-\class{Response}
    \class{Exception}s will cause the Python process to enter post-mortem
    debugging via
    \ulink{\module{pdb}}{http://docs.python.org/lib/module-pdb.html}.}

\item[\code{staging}]
    {foo}

\item[\code{deployment}]
    {foo}

\end{description}



\section{\class{httpy.responders}}
\section{\class{httpy.utils}}


