\chapter{Customization}

There are two kinds of websites: publications and applications. They are
differentiated by their organization and interface models. A \dfn{publication}
website organizes information into individual pages within a hierarchical folder
structure that one navigates by browsing. In an \dfn{application} website, on
the other hand, data is not organized into hierarchical pages but is dealt with
via a non-browsing interface such as a search box.

The HTML version of this documentation is an example of a publication website: a
number of hypertext documents organized into sections. If we weren't using LaTeX
(or if I knew how to use it better), the sections would probably be encoded in
folders. \ulink{Gmail}{http://mail.google.com/mail} is a pure application
website, one which organizes and presents information non-hierarchically. Most
websites, however, are hybrids. That is, within an overall hierarchical
organization you will find both individual pages of information as well as
applications such as a site search feature, or a threaded discussion forum.

Now, in point of fact, publication websites are a subset of application
websites. An application site can use any interface metaphor; a publication is
an application that uses the familiar folder/page metaphor to organize and
present its information. Therefore, every website is fundamentally an
application. httpy's default application implements the basic publication
scenario: serving static files straight off the filesystem.

httpy also enables you to build custom publication, application, and hybrid
websites. It relies on the underlying filesystem to support a publication or
hybrid website's hierarchical organization. To support applications, httpy looks
for a certain filesystem layout with files that define certain Python objects,
which it incorporates into the HTTP request/response process. These are
described below.



\section{Filesystem Layout}

For application websites or sections of websites, httpy looks for a \dfn{magic
directory} named __ (that's two underscores). certain Python files in a certain
subdirectory of any directories that you explicitly configured as applications.
These Python files are expected to define certain objects, which are
incorporated into the HTTP request/response process.



\section{Expected Python Objects}