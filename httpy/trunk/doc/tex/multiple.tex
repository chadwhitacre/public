\subsection{\class{Multiple} Responders \label{multiple}}

httpy is built around the idea that there are two kinds of websites,
publications and applications, which are differentiated by their organization
and interface models. A \dfn{publication} website organizes information into
individual pages within a hierarchical folder structure that one navigates by
browsing. In an \dfn{application} website, on the other hand, data is not
organized into hierarchical pages but is dealt with via a non-browsing interface
such as a search box.

The HTML version of this documentation is an example of a publication website: a
number of hypertext documents organized into sections. If we weren't using LaTeX
(or if I knew how to use it better), the sections would probably be encoded in
folders. \ulink{Gmail}{http://mail.google.com/mail} is a pure application
website, one which organizes and presents information non-hierarchically. Most
websites, however, are hybrids. That is, within an overall hierarchical
organization you will find both individual pages of information as well as
applications such as a site search feature, or a threaded discussion forum.

Now, in point of fact, publication websites are a subset of application
websites. An application site can use any interface metaphor; a publication is
an application that uses the familiar folder/page metaphor to organize and
present its information. Therefore, every website is fundamentally an
application. httpy's default application implements the basic publication
scenario: serving static files straight off the filesystem.

httpy enables you to build custom publication, application, and hybrid websites.
It privileges the filesystem in supporting a publication or hybrid website's
hierarchical organization. To support applications, httpy looks for a certain
filesystem layout with files that define certain Python objects, which it
incorporates into the HTTP request/response process. These are described below.



\subsubsection{Filesystem Layout}

When httpy starts up, it walks the tree rooted in your website's filesystem
root, and it gathers all paths to directories below the root that have a
subdirectory named '__' -- that's two underscores. The directory is said to be
an \dfn{app} or \dfn{application}, and the subdirectory is that application's
\dfn{magic directory}. \footnote{You can override this default behavior. Please
see the section on Configuration.}

For example, consider this filesystem layout:

\begin{verbatim}
/usr/local/www/
/usr/local/www/about/
/usr/local/www/news/
/usr/local/www/news/__
\end{verbatim}

Assuming that \file{/usr/local/www/} is configured as your website root, httpy
will by default find one app: /news. This app's magic directory will be
\file{/usr/local/www/__}.

On each request, httpy looks for an application that matches the beginning of
the requested URI path. So in the above example, requests for \file{/news},
\file{/news/archive}, \file{/news/post}, etc., would be handled by the
\code{/news} app. All other requests would be handled by the default httpy
application, because there is no app configured for the site root.

If you manually configure the \var{apps} setting, and a given app does not have
a magic directory, httpy will raise \class{ConfigError} and exits.



\subsubsection{Expected Python Objects}

When httpy starts up, it attempts to import a module or package named
\module{app} from each application's magic directory. Each \module{app}
module/package must define a class named \class{Application}.
\class{Application}'s constructor must take a single argument, an instance of
\class{AppConfig}. \class{Application} must have at least one other method,
named \code{respond}, which also takes a single argument, an instance of
\class{Request}. If any of these conditions are not met, httpy raises
\class{ConfigError} and exits. Therefore, here is the minimal legal
\file{app.py}:

\begin{verbatim}
class Application:

    def __init__(self, config):
        pass

    def respond(self, request):
        pass

\end{verbatim}

\class{Application.respond} is expected to end the request by raising a
\class{Response} or other \class{Exception}. Therefore, here is the minimal
successful \file{app.py}:

\begin{verbatim}
from httpy.Response import Response

class Application:

    def __init__(self, config):
        pass

    def respond(self, request):
        response = Response(200)
        response.headers['content-type'] = 'text/plain'
        response.body = "Greetings, program!"
        raise response

\end{verbatim}



\subsubsection{Site-Specific Libraries}

If a website root has a magic directory, httpy will look when it starts up for a
subdirectory of this magic directory named \file{site-packages}. It will prepend
this path to \class{sys.path}, with the effect that each website can maintain a
local Python library in \file{<site-root>/__/site-packages}. This library is
global to the applications within a website, however, and only the root's magic
directory is searched for \file{site-packages}.


