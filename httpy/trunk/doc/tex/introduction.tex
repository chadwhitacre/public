\chapter{Introduction}

httpy exists to bridge your Python application with other HTTP applications. Its
primary use case is as an HTTP origin server for a cluster of Python-based
websites numbering into the hundreds or thousands. Therefore, httpy shares its
design aesthetic with toilet paper: instances of httpy must be instantly
available to solidly perform a single function with little or no configuration,
and then be cast away without a thought.

httpy's job is to get HTTP requests into your Python application, and to get
HTTP responses from your app back onto the network. Here are some things that
are explicitly not httpy's job, along with links to tools that do these jobs
well:

\begin{description}

\item[daemonization, complex error logging, uid/gid manipulations]
    {You want \ulink{Dan Bernstein's
    daemontools}{http://cr.yp.to/daemontools.html}. httpy logs everything to the
    standard output, so use
    \ulink{multilog}{http://cr.yp.to/daemontools/multilog.html} to pick up from
    there, and use \ulink{setuidgid}{http://cr.yp.to/daemontools/setuidgid.html}
    to run httpy under a certain account.}

\item[access logging, ssl encryption, virtual hosting, load-balancing]
    {Use an HTTP proxy server such as \ulink{Pound}{http://www.apsis.ch/pound/}.
    You could also do these things with a general-purpose HTTP server such as
    \ulink{Apache}{http://httpd.apache.org/} or
    \ulink{lighttpd}{http://www.lighttpd.net/}.}

\item[caching]
    {Use a caching proxy such as \ulink{Squid}{http://www.squid-cache.org/}, or,
    again, \ulink{Apache}{http://httpd.apache.org/}. Your application should
    also do its own internal caching, of course.}

\item[authentication, authorization, sessioning, storage, templating, etc.]
    {These are your application's responsibility. There are plenty of
    \ulink{Python packages}{http://cheeseshop.python.org/pypi} available to help
    you.}

\end{description}


For the interested, httpy uses the Zope 3 server under the hood. Thanks to the
Zope team, especially for making Zope 3 so much more modular than Zope 2.
