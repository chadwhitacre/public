\subsection{\class{Static} Responders \label{static}}

The \class{Static} responder implements the basic publication application:
serving public files straight off the filesystem. This responder is designed to
be used either by itself or as a mixin. It provides one attributes and two
methods:

\begin{memberdesc}[string]{root}
The filesystem path of the directory to use as the publishing root.
\end{memberdesc}

\begin{memberdesc}[tuple]{defaults}
A tuple of names to interpret as default resources. The first-named is chosen
first.
\end{memberdesc}


\begin{methoddesc}{respond}{request}
This is a pass-through for \method{serve_static}, and can safely be overriden.
\end{methoddesc}

\begin{methoddesc}{serve_static}{request} Serves a static resource from the
filesystem. \var{request} is a \class{Request} object. The URI is translated
using the \function{httpy.utils.translate} function, and the
\mailheader{Content-Type} is set by the standard library's
\ulink{\module{mimetypes}}{http://docs.python.org/lib/module-mimetypes.html} module. In
\code{staging} and \code{deployment} modes, \method{serve_static} supports the
\ulink{\code{304 Not
Modified}}{http://www.w3.org/Protocols/rfc2616/rfc2616-sec10.html#sec10.3.5}
response.
\end{methoddesc}

