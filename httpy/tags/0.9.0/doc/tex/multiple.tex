\subsection{\class{Multiple} Responders \label{multiple}}

The \class{Multiple} responder implements a hybrid website. It is a
meta-responder: it enables you to tie together multiple responders into a
coherent website, using the filesystem for hierarchical organization. It also
gives you site-wide in- and outbound hooks via a "framework" abstraction. The
\class{Multiple} responder is the most complicated part of \class{httpy} to
explain in reference, but almost everything is optional, and it is fundamentally
intuitive.

\class{Multiple} translates URI paths literally to the filesystem, rooting them
in \var{root}. For example, if \var{root} is:

\begin{quote}
\file{/usr/local/www}
\end{quote}

Then \class{Multiple} would translate a request for:

\begin{verbatim}
http://localhost:8080/foo
\end{verbatim}

To the path:

\begin{quote}
\file{/usr/local/www/foo}
\end{quote}


\subsubsection{Responder Discovery}

When \class{Multiple} is instantiated, it searches the tree under \var{root} for
responders. If a directory contains a file named \file{responder.py}, then
\class{Multiple} tries to import a responder from that file, and uses that
responder to serve all requests for resources at or below that directory. If
\code{responder.py} defines a class named \class{Responder}, then that is used
as the responder. Otherwise \class{Multiple} uses the module itself. In either
case, the responder must provide \ulink{iresponder.html}{the \class{IResponder}
interface}, at least implicitly.

So if you define a responder in:

\begin{quote}
\file{/usr/local/www/foo/responder.py}
\end{quote}

Then \class{Multiple} would route requests for all of the following to that
responder:

\begin{verbatim}
http://localhost:8080/foo
http://localhost:8080/foo/
http://localhost:8080/foo/bar
\end{verbatim}

There must be at least one responder, associated with the root of the site. If
none is defined, then \class{Multiple} uses the \class{Static} responder.


\subsubsection{Special Directories}

Furthermore, \class{Multiple} adds exactly one path to \member{sys.path} for
each responder it finds. First it looks for a subdirectory named
\file{site-packages}, then one named \file{lib}. The first one found is used.

If a directory contains a subdirectory named \file{__} (double-underscore,
referred to as a \file{magic directory}), then \class{Multiple} will use any
responder defined there to serve requests for the parent directory, and the
\member{sys.path} addition logic uses the \file{magic directory} as its base. If
both a \file{magic directory} and its parent define a responder, then the
parent's wins. If the responder comes from the parent, then any \file{magic
directory} is considered for inclusion in \member{sys.path}, after
\file{site-packages} and \file{lib}.

Requests that translate to a responder's \file{magic directory} or
\member{sys.path} addition trigger a \ulink{403
Forbidden}{http://www.w3.org/Protocols/rfc2616/rfc2616-sec10.html#sec10.4.4}
response.


\subsubsection{API Additions}

\class{Multiple} adds the following data attributes to each responder as it is
discovered. Existing attributes are not overriden. For class responders, these
are added before instantiation, so they are available during construction.

\begin{tableii}{l|l}{code}{name}{value}

\lineii{__}
    {the filesystem path of the responder's \file{magic directory}, or
    \class{None} if it does not exist [that's two underscores]}

\lineii{pkg}
    {the path added to \member{sys.path} for this responder}

\lineii{root}
    {the responder's filesystem path}

\lineii{site___}
    {the filesystem path of the \file{magic directory} in the publishing root,
    or \class{None} if it does not exist [that's three underscores]}

\lineii{site_root}
    {the filesystem path of the publishing root}

\lineii{path}
    {the URI path below which requests will go to this responder}

\end{tableii}



\subsubsection{Frameworks}

Responders are mini Python applications scattered about your site hierarchy. In
order to unify these responders into a coherent website, \class{Multiple}
provides a framework abstraction. \class{Multiple} looks for a file named
\file{framework.py} in \var{root} and in \var{root}'s \file{magic directory}, in
that order. The first found is used. Frameworks have no required API, but they
may hook into the HTTP transaction process by defining some or all of the
following callables. These are listed in the order they are called.

\begin{funcdesc}{get_responder}{request}
\var{request} is an \class{httpy.Request} object. This routine must return
a provider of \ulink{\class{IResponder}}{responder.html}.
\end{funcdesc}

\begin{funcdesc}{wrap_request}{responder, request}
\var{responder} is the result of the call to \method{get_responder};
\var{request} is the same \class{Request} object. This routine may return any
object that is meaningful to \var{responder}'s \function{respond} routine.
\end{funcdesc}

\begin{funcdesc}{unwrap_response}{response, responder, request} \var{response}
is the object raised by \var{responder}.\function{respond}. Note that this
object must be a subclass of \class{httpy.Response}, or it will be interpreted
as a \ulink{\code{500 Internal Server
Error}}{http://www.w3.org/Protocols/rfc2616/rfc2616-sec10.html#sec10.5.1}.
\var{request} is the same \class{Request} object.

\end{funcdesc}

\begin{funcdesc}{stop}{}
This method is called when the application terminates normally.
\end{funcdesc}

Note that because \var{request} is passed to each hook, and then discarded
following the transaction, you can safely store state there between hooks.

As with responders, \class{Multiple} adds the following data attributes to your
framework if they are not already present. If your framework is a class, these
are added immediately before instantiation.


\begin{tableii}{l|l}{code}{name}{value}

\lineii{__}
    {the filesystem path of the site's magic directory [that's two underscores]}

\lineii{root}
    {the filesystem path of the site's publishing root}

\end{tableii}
