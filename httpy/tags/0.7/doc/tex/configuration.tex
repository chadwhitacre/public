\chapter{Configuration}

httpy exposes the following configuration parameters:


\begin{tableiii}{l|l|l}{var}{Parameter}{Description}{Default Behavior/Value}

\lineiii{apps}
    {A colon-separated list of paths that should be considered applications. The
    paths should be given as if they were in URL-space, or from the filesystem's
    perspective, as if they were rooted in the website's filesystem root (but
    always using the forward-slash as the path separator). So they should begin
    with a forward-slash but should not include the filesystem path to the root
    of the website. The website root is always implicitly considered an
    application.}
    {Walk the tree rooted in \var{root} and collect all directories with a
    subdirectory named __ (that's two underscores).}

\lineiii{ip}
    {The IP address that httpy should listen on. This must either be a valid
    IPv4 address or a null value, in which case httpy will listen on all
    available addresses.}
    {Listen on all available addresses.}

\lineiii{mode}
    {Either of the strings \code{deployment} or \code{development}. This affects
    various aspects of httpy's behavior.}
    {\code{deployment}}

\lineiii{port}
    {The TCP port that httpy should bind to. This must be an integer between 0
    and 65535, inclusive.}
    {\code{8080}}

\lineiii{root}
    {The path to a directory on the filesystem which will serve as the root of
    the website.}
    {Use the current working directory.}

\lineiii{verbosity}
    {An integer from 0 to 99, inclusive, indicating how much information to
    log.}
    {\code{1}}

\end{tableiii}

The defaults may be overriden in three ways: via command line options, through a
configuration file, and by setting environment variables. This is also the order
in which they take precedence, such that configuration file settings override
environment variables, etc. The subsections below refer to the parameter names
as given in the first column of the above table.




\section{Command Line Options}

httpy uses the Python standard library's \module{optparse} module to represent
its command line arguments. The parameter names may be used for long-named
arguments, or their initial letter may be used for short-named arguments. So for
example, httpy could be started on a custom port, with a custom website root,
using the following command:

\begin{verbatim}
% httpy -p9000 --root ~/www
\end{verbatim}

In addition, the path to any configuration file is specified with the
\programopt{-f}/\longprogramopt{file} argument.

\begin{seealso}
\seetitle [http://www.python.org/doc/lib/module-optparse.html]
          {optparse}
          {httpy uses the standard library's optparse module.}
\end{seealso}




\section{Configuration File Settings}

httpy uses the standard library's \module{RawConfigParser} class to represent
its configuration file. This means that the file is in the style of a Windows
INI file, with section headers in brackets, and \code{name=value} pairs on
successive lines. Use the parameter names as given earlier. The number and names
of any sections are irrelevant, but \module{RawConfigParser} requires that there
be at least one section. As an example, a configuration file with the following
contents would set the verbosity of an httpy instance:

\begin{verbatim}
[main]
verbosity = 99
\end{verbatim}

\begin{seealso}
\seetitle [http://www.python.org/doc/lib/RawConfigParser-objects.html]
          {RawConfigParser}
          {httpy uses the standard library's RawConfigParser class.}
\end{seealso}




\section{Environment Variables}

The names of the environment variables recognized by httpy are derived by
uppercasing the above names, and prepending them with HTTPY_. So, for example,
to configure a server so that all instances of httpy on that server are run in
development mode instead of deployment mode, one would simply set the
environment variable \envvar{HTTPY_MODE} to \code{development}.
